\documentclass{mini}
\usepackage[utf8]{inputenc}
%------------------------------------------------------------------------------%
\title{ Estymacja w modelu Coxa metodą stochastycznego spadku gradientu \newline dla $p >> n$ na przykładzie danych \newline z The Cancer Genome Atlas}
\author{Marcin Kosiński}
\supervisor{\text{prof. ndzw. dr hab. inż Przemysław Biecek}}
\type{magisters}
\discipline{matematyka}
\monthyear{lipiec 2015}
\date{\today}
\album{123123}
%------------------------------------------------------------------------------%
\begin{document}
\maketitle
\tableofcontents

\chapter*{Wprowadzenie}


\chapter{Estymacja metodą największej wiarogodności}
\chapter{Numeryczne metody estymacji}
\subsection{Ogólne pojęcia związane ze zbieżnością algorytmu}
\subsubsection{Warunki stopu itp}
\section{Algorytmy spadku wzdłuż gradientu}
\subsection{Algorytm Cauchy'ego}
\subsection{Algorytm Raphsona-Newtona}
\section{Algorytmy stochastycznego spadku wzdłuż gradientu}
\subsection{Metoda estymacji stochastycznego spadku gradientu I}
\subsubsection{Algorytm SGD}
\subsection{Metoda estymacji stochastycznego spadku gradientu II}

\chapter{Model Coxa}
\section{Estymacja analityczna w oparciu o metodę największej wiarogodności dla funkcji pseudo/sub-wiarogodności}
\section{Estymacja numeryczna w oparciu o metodę stochastycznego spadku gradientu rzędu I dla funkcji pseudo/sub-wiarogodności}


\chapter{A gdzie to $p>>n$?}

\chapter{Zaimplementowany algorytm}
\chapter{Analiza danych genomicznych - model Coxa z estymacją metodą stochastycznego spadku gradientu}
\section{Opis i pobranie danych}
\section{Analiza}

\chapter{Studium przypadku}
\section{Porównanie wyników algorytmu SGD z R-N}


\appendix

\chapter{Kody w R}
\chapter{Dokumentacja pakietu RTCGA}


\makestatement
\end{document}
