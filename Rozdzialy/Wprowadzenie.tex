\chapter*{Wprowadzenie}


+++
Analiza przeżycia.
+++

Najbardziej charakterystyczną cechą typowych danych, jakimi posługuje się w analizie przeżycia, jest obecność obiektów, w których końcowe zdarzenie nastąpiło (wówczas ma się do czynienia z obserwacjami \textit{kompletnymi}), oraz obiektów, w których to zdarzenie (jeszcze) nie nastąpiło (obserwacja \textit{ucięta}). Ta specyficzna postać danych statystycznych doprowadziła do powstania specjalnych metod stosowanych tylko w analizie czasu trwania zjawisk. Jednym z takich modeli jest model proporcjonalnych hazardów Coxa. Jak podaje \cite{assel}, model proporcjonalnych hazardów Coxa jest jednym z najszerzej stosowanych modeli w onkologicznych publikacjach naukowych, ale także jedną z najmniej rozumianych metod statystycznych. Wynika to z łatwego dostępu do pakietów statystycznych zawierających programy do analizy przeżyć, modeli regresji i analiz wielowariantowych, ale prawie nigdy nie zawierających dobrego opisu podstawowych zasad działania modelu Coxa. Dostarczają one wyłącznie instrukcje, jak wprowadzić dane i uruchomić odpowiednie procedury w celu uzyskania wyniku. Poniższy praca zawiera pełny opis metodologii modelu proporcjonalnych hazardów Coxa, w tym wyjaśnienie najważniejszych pojęć. 


++++++++