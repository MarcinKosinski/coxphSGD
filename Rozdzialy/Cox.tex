\chapter{Model Coxa}
\begin{flushright}
\textit{The proportion of my life that I spent working on the \\ 
proportional hazards model is, in fact, very small. I had an \\
 idea of how to solve it but I could not complete the \\
  argument and so it took me about four years on and off... \\
Sir David Cox, An interview with Sir David Cox, 2014.
}
\end{flushright}


W tym rozdziale zostanie przedstawiony model proporcjonalnych hazardów Coxa. Głównym celem tej pracy jest wykorzystanie, nietypowej w tym modelu, numerycznej metody estymacji współczynników metodą stochastycznego spadku gradientu. Więcej o estymacji metodą stochastycznego spadku gradientu napisane jest w rozdziale \ref{SGD}. Definicje i twierdzenia w tym rozdziale oparte są o \cite{cox}, \cite{ther} i \cite{assel}.

%\section{Nomenklatura i podstawy analizy przeżycia}

%Analiza przeżycia to zbiór metod statystycznych badających procesy, w których interesujący jest czas, jaki upłynie do (pierwszego) wystąpienia pewnego zdarzenia.

\section{Wprowadzenie do modelu Coxa i nomenklatura}

Model proporcjonalnych hazardów Coxa \cite{cox} jest obecnie najczęściej stosowaną procedurą do modelowania relacji pomiędzy zmiennymi objaśniającymi a przeżyciem lub innym cenzurowanym rezultatem. Model ten umożliwia analizę wpływu czynników prognostycznych na przeżycie. Sir David Cox opracował tego typu model dla tabeli przeżyć i zilustrował zastosowanie modelu dla przypadku
leukemii, ale model może być stosowany do obliczania
przeżyć w odniesieniu do wszystkich innych chorób, jak
w przypadku przeżyć w chorobach nowotworowych lub
kardiologicznych po transplantacji serca lub zawałach
serca \cite{norwe}. 

\begin{definition}
\textbf{Model Coxa} określa funkcję hazardu dla i-tej obserwacji $X_i$ jako
\begin{equation}
\lambda_i(t) = \lambda_0(t)e^{X_i(t)\beta},
\end{equation}
gdzie $\lambda_0$ to niesprecyzowana nieujemna funkcja nazywana \textit{bazowym hazardem}, a $\beta$ to wektor współczynników o rozmiarze $p$, co odpowiada liczbie zmiennych objaśniających w modelu.
\end{definition}

Wspomnianą funkcję hazardu definiuje się jak następuje:
\begin{definition}
\textbf{Funkcja hazardu} to funkcja, która wyraża się wzorem
\begin{equation}
\begin{align*}
\lambda_j(t) & =  \lim\limits_{h\rightarrow 0}\dfrac{\mathbb{P}(t \leq T^* \leq t +h | T^* \geq t)}{h} & \ \\
 \ & = \lim\limits_{h\rightarrow 0}\dfrac{\mathbb{P}(t \leq T^* \leq t +h )}{h}\cdot\frac{1}{  \mathbb{P}(T^* \geq t)} & \ \\ \ & = \lim\limits_{h\rightarrow 0} \frac{F_j(t+h) - F_j(t) }{h}\cdot\frac{1}{S_j(t)} & =  \frac{f_j(t)}{S_j(t)}.
 \end{align*}
\end{equation}
\end{definition}

W powyższej definicji $T^*$ oznacza czas do wystąpienia zdarzenia. Zakłada się, że wewnątrz każdej grupy $j=1,2,\dots, k, k \in \mathbb{N}$ czasy $T_i^*$ to niezależne zmienne losowe z tego samego rozkładu o zadanej gęstości $f_j(t)$, zaś $S_j(t)$ to \textit{funkcja przeżycia} w grupie $j$, która spełnia 
\begin{equation}
S_j(t) = \mathbb{P}(T^* \geq t )  = 1 - F_j(t), 
\end{equation}
gdzie $F_j(t)$ to dystrybuanta rozkładu zadanego gęstością $f_j(t)$.


Wartość funkcji hazardu w momencie $t$ traktuje się jako chwilowy potencjał pojawiającego się zdarzenia (np. śmierci lub choroby), pod warunkiem że osoba dożyła czasu $t$. Funkcja hazardu nazywana jest również funkcją ryzyka,
intensywnością umieralności (\textit{force of mortality}), umieralnością
chwilową (\textit{instantaneous death rate}) lub chwilową
częstością niepowodzeń (awarii) (\textit{failure rate}). Ostatniego
określenia używa się w teorii odnowy \cite{cox0}, w której analizuje
się awaryjność elementów przemysłowych. 


Model Coxa nazywany jest modelem proporcjonalnych hazardów, gdyż stosunek (proporcja) hazardów dla dwóch obserwacji $X_i$ oraz $X_j$, które mają współczynniki stałe w czasie, jest stały w czasie:  
$$\dfrac{\lambda_i(t)}{\lambda_j(t)} = \dfrac{\lambda_0(t)e^{X_i\beta}}{\lambda_0(t)e^{X_j\beta}} = \dfrac{e^{X_i\beta}}{e^{X_j\beta}}$$

\textbf{Założenia modelu proporcjonalnego ryzyka Coxa.}

\section{Estymacja w modelu Coxa}
Estimation of ß is based on the partiallikelihood function introduced by
Cox [36]. For untied failure time data it has the form
P L(ß) = fI rr { Yi(t)ri(ß, t) }dNi(tl ,
i=l t~O Lj Yj(t}rj(ß, t) (3.2)
where ri(ß, t) is the risk score for subject i, ri(ß, t) = exp[Xi(t)ß] == ri(t).
The log partial likelihood can be written as a sum
I(ß) ~ t, f [Y;( t)X;( t)ß -log ( ~ Y, (t)r, (t») 1 dN; (t), (3.3)
from which we can already fore see the counting process structure.
Although the partiallikelihood is not, in general, a likelihood in the sense
of being proportional to the probability of an observed dataset, nonetheless
it can be treated as a likelihood for purposes of asymptotic inference.
Differentiating the log partial likelihood with respect to ß gives the p x 1
score vector, U(ß):
(3.4)
where x(ß, s) is simply a weighted mean of X, over those observations still
at risk at time s,
-(ß ) = L Yi(s)ri(s)Xi(s) x ,s L Yi(s)ri(s) , (3.5)
with Yi (s )ri (s) as the weights.
The negative second derivative is the p x p information matrix
I(ß) = t 100 V(ß, s)dNi(s),
i=l 0
(3.6)
where V (ß, s) is the weighted variance of X at time s:
V(ß, s) = Li Yi(sh(s)[Xi(s) - x(ß, s)]'[Xi(s) - x(ß, s)]. (3.7)
Li Yi(sh(s)
The maximum partiallikelihood estimator is found by solving the partial
likelihood equation:
U(ß) = O.
The solution 13 is consistent and asymptotically normally distributed with
mean ß, the true parameter vector, and variance {EI(ß)} ~I, the inverse 
3.1 Introduction and notation 41
of the expected information matrix. The expectation requires knowledge of
the censoring distribution even for those observations with observed failures;
that information is typically nonexistent. The inverse of the observed
information matrix I-I (13) is available, has better finite sampIe properties
than the inverse of the expected information, and is used as the variance
of ß.
Eoth SAS and S-Plus use the Newton-Raphson algorithm to solve the
partiallikelihood equation. Starting with an initial guess 13(01, the algorithm
iteratively computes
(3.8)
until convergence, as assessed by stability in the log partial likelihood,
I(ß(nH1) ~ l(ß(n1). This algorithm is incredibly robust for the Cox partial
likelihood. Convergence problems are very rare using the default initial
value of 13(01 = 0 and easily addressed by simple methods such as stephalving.
As a result many packages (e.g., SAS) do not even have an option
to choose alternate starting values. 
\section{Płynne przejście do kolejnej sekcji}

%\section{Estymacja analityczna w oparciu o metodę największej wiarogodności dla funkcji pseudo/sub-wiarogodności}
%\section{Estymacja numeryczna w oparciu o metodę stochastycznego spadku gradientu rzędu I dla funkcji pseudo/sub-wiarogodności}
\newpage
Poszukujemy rozwiazan równosci
$$ \delta ln Ln  / \delta \theta =  0.$$
Tym razem w ogólnym przypadku zwykle nie znajdziemy analitycznego rozwiazania. W
zwiazku z tym jestesmy zdani na metody iteracyjne. Poza tym, byc moze rozwiazanie problemu
nie istnieje albo istnieje ich wiele. Zwykle uzywa sie do tego celu, tj. znalezienia
rozwiazania, metody Newtona, zwykle w literaturze statystycznej w zastosowaniu do tego
problemu, nazywanej metoda Newtona-Raphsona. W efekcie w zasadzie dla kazdego modelu
z osobna nalezy badac własnosci asymptotyczne estymatora najwiekszej wiarygodnosci.