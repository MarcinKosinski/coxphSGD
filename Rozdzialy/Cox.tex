\chapter{Model Coxa}
\begin{flushright}
\textit{The proportion of my life that I spent working on the \\ 
proportional hazards model is, in fact, very small. I had an \\
 idea of how to solve it but I could not complete the \\
  argument and so it took me about four years on and off... \\
Sir David Cox, An interview with Sir David Cox, 2014.
}
\end{flushright}


W tym rozdziale zostanie przedstawiony model proporcjonalnych hazardów Coxa. Głównym celem tej pracy jest wykorzystanie, nietypowej w tym modelu, numerycznej metody estymacji współczynników metodą stochastycznego spadku gradientu. Więcej o estymacji metodą stochastycznego spadku gradientu napisane jest w rozdziale \ref{SGD}. Definicje i twierdzenia w tym rozdziale oparte są o \cite{cox}, \cite{ther}, \cite{assel} i \cite{burzyk1}.

%\section{Nomenklatura i podstawy analizy przeżycia}

%Analiza przeżycia to zbiór metod statystycznych badających procesy, w których interesujący jest czas, jaki upłynie do (pierwszego) wystąpienia pewnego zdarzenia.

\section{Wprowadzenie do modelu Coxa i nomenklatura}

Model proporcjonalnych hazardów Coxa \cite{cox} jest obecnie najczęściej stosowaną procedurą do modelowania relacji pomiędzy zmiennymi objaśniającymi a przeżyciem lub innym cenzurowanym rezultatem. Model ten umożliwia analizę wpływu czynników prognostycznych na przeżycie. Sir David Cox opracował tego typu model dla tabeli przeżyć i zilustrował zastosowanie modelu dla przypadku
leukemii, ale model może być stosowany do obliczania
przeżyć w odniesieniu do wszystkich innych chorób, jak
w przypadku przeżyć w chorobach nowotworowych lub
kardiologicznych po transplantacji serca lub zawałach
serca \cite{norwe}. 

\begin{definition}
\textbf{Model Coxa} określa funkcję hazardu dla i-tej obserwacji $X_i$ jako
\begin{equation}
\lambda_i(t) = \lambda_0(t)e^{X_i(t)'\beta},
\end{equation}
gdzie $\lambda_0$ to niesprecyzowana nieujemna funkcja nazywana \textit{bazowym hazardem}, a $\beta$ to wektor współczynników o rozmiarze $p$, co odpowiada liczbie zmiennych objaśniających w modelu.
\end{definition}

Takie sformułowanie modelu gwarantuje, że funkcja hazardu jest nieujemna. Wspomnianą funkcję hazardu definiuje się jak następuje:
\begin{definition}
\textbf{Funkcja hazardu} to funkcja, która wyraża się wzorem
\begin{equation}
\begin{align*}
\lambda_j(t) & =  \lim\limits_{h\rightarrow 0}\dfrac{\mathbb{P}(t \leq T^* \leq t +h | T^* \geq t)}{h} & \ \\
 \ & = \lim\limits_{h\rightarrow 0}\dfrac{\mathbb{P}(t \leq T^* \leq t +h )}{h}\cdot\frac{1}{  \mathbb{P}(T^* \geq t)} & \ \\ \ & = \lim\limits_{h\rightarrow 0} \frac{F_j(t+h) - F_j(t) }{h}\cdot\frac{1}{S_j(t)} & =  \frac{f_j(t)}{S_j(t)}.
\end{align*}
\end{equation}
\end{definition}

W powyższej definicji $T^*$ oznacza czas do wystąpienia zdarzenia. Zakłada się, że wewnątrz każdej grupy $j=1,2,\dots, k, k \in \mathbb{N}$ czasy $T_i^*$ to niezależne zmienne losowe z tego samego rozkładu o zadanej gęstości $f_j(t)$, zaś $S_j(t)$ to \textit{funkcja przeżycia} w grupie $j$, która spełnia 
\begin{equation}
S_j(t) = \mathbb{P}(T^* \geq t )  = 1 - F_j(t), 
\end{equation}
gdzie $F_j(t)$ to dystrybuanta rozkładu zadanego gęstością $f_j(t)$.


Wartość funkcji hazardu w momencie $t$ traktuje się jako chwilowy potencjał pojawiającego się zdarzenia (np. śmierci lub choroby), pod warunkiem że osoba dożyła czasu $t$. Funkcja hazardu nazywana jest również funkcją ryzyka,
intensywnością umieralności (\textit{force of mortality}), umieralnością
chwilową (\textit{instantaneous death rate}) lub chwilową
częstością niepowodzeń (awarii) (\textit{failure rate}). Ostatniego
określenia używa się w teorii odnowy \cite{cox0}, w której analizuje
się awaryjność elementów przemysłowych. 


Model Coxa nazywany jest modelem proporcjonalnych hazardów, gdyż stosunek (proporcja) hazardów dla dwóch obserwacji $X_i$ oraz $X_j$, które mają współczynniki stałe w czasie, jest stały w czasie:  
$$\dfrac{\lambda_i(t)}{\lambda_j(t)} = \dfrac{\lambda_0(t)e^{X_i\beta}}{\lambda_0(t)e^{X_j\beta}} = \dfrac{e^{X_i\beta}}{e^{X_j\beta}}.$$

Oznacza to, że hazard dla jednej obserwacji można uzyskać poprzez przemnożenie hazardu dla innej obserwacji przez pewną stałą $c$:
$$\lambda_i(t) = \dfrac{e^{X_i'\beta}}{e^{X_j'\beta}} \cdot \lambda_j(t) = c \cdot \lambda_j(t).$$

W modelu proporcjonalnych hazardów istotnym elementem jest estymacja stałej $c$.

\section{Założenia modelu proporcjonalnego ryzyka Coxa.}

Model Coxa oparty jest o założenia, że
\begin{enumerate}
\item Współczynniki modelu $\beta_k, k = 1,\cdots,p$ są stałe w czasie.
\item Stosunek hazardów dla dwóch obserwacji, z racji na stałość współczynników w czasie, jest stały w czasie.
\item Obserwacje są niezależne.
\item Cenzurowanie czasów jest nie-informatywne.
\item Cenzurowanie czasów jest niezależne.
\end{enumerate}

Model Coxa znalazł szerokie zastosowanie z racji na nietypowy rodzaj danych, które jest w stanie wykorzystać do estymacji współczynników w modelu, przekładających się na proporcje hazardów. Z uwagi na aspekt praktyczny podyktowany warunkami technicznymi prób klinicznych i badań biologicznych, zbiory danych klinicznych zawierają cenzurowane czasy zdarzeń. Oznacza to, że w wielu przypadkach niemożliwe jest obserwowanie czasu zdarzeń dla wszystkich obserwacji w zbiorze. Niekiedy jest to uwarunkowane zbyt długim czasem do wystąpienia zdarzenia, czasem jest to związane z zaplanowanym okresem próby klinicznej, który jest krótszy niż czas do zdarzenia dla pacjentów, którzy mogli zostać włączeni do próby klinicznej pod koniec jej trwania i nie udało się dla nich zaobserwować czasów zdarzeń. W wielu przypadkach pacjenci, traktowani jako obserwacje w zbiorze, znikają z pola widzenia w momencie, gdy np. przestają pojawiać się na wizytach kontrolnych. Może być to spowodowane negatywnymi relacjami z lekarzem prowadzącym lub przeprowadzką. W takich sytuacjach wykorzystuje się daną obserwację do momentu jej ostatniej kontroli. Nie rezygnuje się z tej obserwacji w analizie i wykorzystuje się o niej informacje w pełni dla czasu, w którym przebywała pod obserwacją. Jest to ogromna zaleta modelu Coxa.

Z tej przyczyny potrzebne są założenia modelu dotyczące cenzurowania czasów.

\begin{definition}
\textbf{Cenzuorwanie prawostronne} polega na zaobserwowaniu czasu 
$$T= \min(T^*, C),$$
gdzie $T^*$ to prawdziwy czas zdarzenia, zaś $C$ jest nieujemną zmienną losową.
\end{definition}


\begin{definition}
\textbf{Cenzurowanie jest niezależne} jeśli zachodzi
$$ \lim\limits_{h\rightarrow 0}\dfrac{\mathbb{P}(t \leq T^* \leq t +h | T^* \geq t)}{h} =  \lim\limits_{h\rightarrow 0}\dfrac{\mathbb{P}(t \leq T^* \leq t +h | T^* \geq t, Y(t) = 1)}{h},$$
gdzie $Y(t) = 1$ jeśli do chwili $t$ nie wystąpiło zdarzenie ani cenzurowanie, czyli jednostka pozostaje narażona na ryzyko zdarzenia.
\end{definition}

Interpretacja tej definicji jest następująca: jednostka cenzurowana w chwili $t$ jest reprezentatywna dla wszystkich innych narażonych na ryzyko zdarzenia w chwili $t$.

\begin{definition}
\textbf{Cenzurowanie jest nie-informatywne} jeśli zachodzi
$$g(t;\theta, \phi) \equiv g(t;\phi),$$
gdzie $g(t;\theta, \phi)$ jest funkcją gęstości dla cenzurowań $C_i$ wyrażonych jako niezależne zmienne losowe o jednakowym rozkładzie, zaś prawdzie czasy $T^*_i$ są interpretowane jako niezależne zmienne losowe o jednakowym rozkładzie i funkcji gęstości $f(t;\theta)$.
\end{definition}

Oznacza to, że cenzurowanie nie daje informacji o parametrach rozkładu czasów zdarzeń.

Terminu cenzurowanie po raz pierwszy w literaturze użył Hald w 1949 r. \cite{hald}.

\section{Estymacja w modelu Coxa}

Funkcja hazardu jest wykładniczą funkcją zmiennych objaśniających, nieznana jest natomiast
postać bazowej funkcji hazardu, co bez dalszych założeń uniemożliwia estymację standardową
metodą największej wiarygodności. Rozwiązaniem Cox’a jest maksymalizacja tylko tego fragmentu funkcji wiarygodności, który zależy od estymowanych parametrów. W modelu proporcjonalnych hazardów Coxa estymacja współczynników $\beta$ oparta jest o częściową funkcję wiarogodności, którą wprowadził Cox w 1972 r. \cite{cox}. 

Dla konkretnego czasu zdarzenia $t_i$, gdzie w zbiorze obserwowanych jest K czasów zdarzeń, prawdopodobieństwo warunkowe ze względu na liczność zbioru ryzyka w czasie $t_i$, że czas zdarzenia dotyczy $i$-tej jednostki spośród wciąż obserwowanych jest równe
\begin{equation}
\dfrac{e^{X_i'\beta}}{\sum\limits_{l\in \mathscr{R}(t_i)}^{}e^{X_l'\beta}},
\end{equation}
gdzie \textit{zbiór ryzyka} $\mathscr{R}(t_i)$, w chwili $t_i$, rozumiany jest jako zbiór indeksów obserwacji, które są w danym czasie $t_i$ pod obserwacją.

Wtedy dla niezależnego cenzurowania i dla czasów zdarzeń, które nie zaszły jednocześnie \textbf{częściowa funkcja wiarogodności} ma postać:
\begin{equation}
L(\beta) = \prod\limits_{i=1}^{K}\dfrac{e^{X_i'\beta}}{\sum\limits_{l=1}^{n}Y_l(t_i)e^{X_l'\beta}},
\end{equation}
gdzie $Y_l(t_i)$ = 1, gdy obserwacja $X_l$ jest w zbiorze ryzyka w czasie $t_i$, i $Y_l(t_i)$ = 0 w przeciwnym przypadku, a $n$ to liczba obserwacji w zbiorze. Zaletą takiej postaci funkcji częściowej wiarogodności jest to, że w jej wzorze nie występuje funkcja bazowego hazardu, zatem estymacja współczynników może odbywać się bez znajomości jej postaci. Pomimo, że częściowa funkcja wiarogodności, nie jest funkcją wiarogodności w sensie bycia proporcjonalną do prawdopodobieństwa obserwowanego zbioru, można ją traktować jako funkcję wiarogodności dla celów asymptotycznego wnioskowania o współczynnikach modelu.

W takich przypadkach standardowo w celu znalezienia maximum, aby ułatwić obliczenia, można funkcję obłożyć monotoniczną transformacją jaką jest logarytm, by otrzymać \textbf{częściową~funkcję~log-wiarogodności}
\begin{equation}
\ell(\beta) = \sum\limits_{i=1}^{K}X_i'\beta - \sum\limits_{i=1}^{K}\log\Big(\sum\limits_{l\in \mathscr{R}(t_i)}^{}e^{X_l'\beta}\Big).
\end{equation}



