\chapter{Model Coxa}
\section{Estymacja analityczna w oparciu o metodę największej wiarogodności dla funkcji pseudo/sub-wiarogodności}
\section{Estymacja numeryczna w oparciu o metodę stochastycznego spadku gradientu rzędu I dla funkcji pseudo/sub-wiarogodności}
Poszukujemy rozwiazan równosci
$$ \delta ln Ln  / \delta \theta =  0.$$
Tym razem w ogólnym przypadku zwykle nie znajdziemy analitycznego rozwiazania. W
zwiazku z tym jestesmy zdani na metody iteracyjne. Poza tym, byc moze rozwiazanie problemu
nie istnieje albo istnieje ich wiele. Zwykle uzywa sie do tego celu, tj. znalezienia
rozwiazania, metody Newtona, zwykle w literaturze statystycznej w zastosowaniu do tego
problemu, nazywanej metoda Newtona-Raphsona. W efekcie w zasadzie dla kazdego modelu
z osobna nalezy badac własnosci asymptotyczne estymatora najwiekszej wiarygodnosci.