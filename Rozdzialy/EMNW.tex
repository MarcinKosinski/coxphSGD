\chapter{Estymacja metodą największej wiarogodności}
\begin{flushright}
\textit{The making of maximum likelihood was one of \\
the most important developments in 20th century statistics. \\
It was the work of one man but it was no simple process (...). \\
John Aldrich o R. A. Fisher'ze, 1997 \cite{aldrich1}
}
\end{flushright}

\section{Estymacja}
\section{Metoda największej wiarogodności}

Metodę największej wiarogodności wprowadził R. A. Fisher w 1922 r. \cite{fisher2}, dla której po raz pierwszy procedurę numeryczną zaproponował już w 1912 r. \cite{fisher1}. O burzliwym procesie powstawania metody, o zmianach w jej uzasadnieniu, o koncepcjach, które powstały w obrębie tej metody takich jak parametr, statystyka, wiarogodność, dostateczność czy efektywność oraz o podejściach, które Fisher odrzucił tworząc podstawy pod nową teorię można przeczytać w~obszernej pracy dokumentalnej \cite{aldrich1}. 

Metoda ta, jako alternatywa dla metody najmniejszych kwadratów \cite{legendre1}, \cite{gauss1}, była rozwijana i szeroko stosowana później przez wielu statystyków i wciąż znajduje obszerne zastosowania w wielu obszarach estymacji statystycznej, np. \cite{hutch1}, \cite{kenward1}, \cite{millar1}.

Aby zdefiniować estymator oparty o metodę największej wiarogodności, należy najpierw wprowadzić pojęcie funkcji wiarogodności.

\begin{definition}
\textbf{Funkcją wiarogodności} nazywamy funkcję $L : \Theta \rightarrow \mathbb{R}$ daną wzorem $$ L(\Theta|x_1, \dots , x_n) = f(\Theta; x_1, \dots , x_n),$$
którą rozważamy jako funkcję parametru $\theta$ przy ustalonych wartościach obserwacji $x_1, \dots , x_n$, gdzie $$ f(\Theta; x_1, \dots , x_n) = \left\lbrace \begin{align*}
\ & \mathbb{P}_{\Theta}( X_1 = x_1, \dots , X_n = x_n), & \text{dla rozkładów dyskretnych}, & \ \\
\ & f_{\Theta}(x_1, \dots , x_n), & \text{dla rozkładów absolutnie ciągłych.} & \
\end{align*}$$

\end{definition}

Oznacza to, że wiarogodność jest właściwie tym samym, co gęstość prawdopodobieństwa,
ale rozważana jako funkcja parametru $\theta$, przy ustalonych wartościach obserwacji
$x = X(\omega)$.


\begin{definition}
\textbf{Estymatorem największej wiarogodności} parametru $\theta$, oznaczanym ENW($\theta$), nazywamy wartość parametru, w której funkcja
wiarogodności przyjmuje supremum $$L(\hat{\theta}) = \sup_{\theta \in \Theta} L(\theta).$$

\end{definition}

Niektóre pozycje w literaturze, w definicji estymatora największej wiarogodności, supremum zastępują wartością największą \cite{rydl1}, str 14.

\section{Asymptotyczne własności estymatorów największej wiarogodności}

W tym podrozdziale zostanie wykazane, że estymator największej wiarygodności jest
\begin{enumerate}
\item asymptotycznie nieobciążony,
\item zgodny,
\item asymptotycznie normalny.
\end{enumerate}

\subsection{Zgodność estymatorów największej wiarogodności}
\subsection{Asymptotyczna normalność estymatorów największej wiarygodności}