\chapter{Numeryczne metody estymacji}

Przez \textbf{numerykę} rozumie się dziedzinę matematyki
zajmującą się rozwiązywaniem przybliżonych zagadnień algebraicznych. Odkąd zjawiska przyrodnicze zaczęto opisywać przy użyciu formalizmu matematycznego,
pojawiła się potrzeba rozwiązywania zadań analizy matematycznej czy algebry. Dopóki były
one nieskomplikowane, dawały się rozwiązywać analitycznie, tzn. z użyciem pewnych
przekształceń algebraicznych prowadzących do otrzymywania rozwiązań ścisłych danych
problemów. Z czasem jednak, przy powstawaniu coraz to bardziej skomplikowanych teorii
opisujących zjawiska, problemy te stawały się na tyle złożone, iż ich rozwiązywanie ścisłe
było albo bardzo czasochłonne albo też zgoła niemożliwe. Numeryka pozwalała znajdywać
przybliżone rozwiązania z żądaną dokładnością. Ich podstawową zaletą była ogólność tak
formułowanych algorytmów, tzn. w ramach danego zagadnienia nie miało znaczenia czy było
ono proste czy też bardzo skomplikowane (najwyżej wiązało się z większym nakładem pracy
obliczeniowej). Natomiast wadą była czasochłonność. Stąd prawdziwy renesans metod
numerycznych nastąpił wraz z powszechnym użyciem w pracy naukowej maszyn cyfrowych,
a w szczególności mikrokomputerów \cite{milewski}. Dziś dziesiątki żmudnych dla człowieka operacji
arytmetycznych wykonuje komputer, jednak złożoność obliczeniowa algorytmów uczących i modeli statystycznych stała się krytycznym czynnikiem ograniczającym w sytuacjach, gdy rozważane są duże zbiory danych. Te ograniczenia spowodowały, że w uczeniu maszynowym i modelowaniu statystycznym wielkiej skali zaczęto wykorzystywać algorytmy \textbf{stochastycznego spadku gradientu}. W poniższym rozdziale przedstawione są klasyczne algorytmy spadku wzdłuż gradientu Cauchy'ego oraz Raphsona-Newtona. Następnie omówiony jest algorytm stochastycznego spadku wzdłuż gradientu, którego wykorzystanie do estymacji współczynników w modelu Coxa jest kluczowym celem tej pracy. Algorytm stochastycznego spadku gradientu to metoda optymalizacji wzdłuż spadku gradientu wykorzystywana w sytuacjach, gdy rozważaną funkcję można zapisać jako sumę różniczkowalnych składników. Ponadto przedstawiono również zalety algorytmów stochastycznego spadku gradientu, które przemawiają za atrakcyjnością i popularnością tego typu rozwiązania. Ostatecznie przedyskutowano asymptotyczną efektywność estymatorów uzyskanych dzięki jednemu przejściu po zbiorze, zwanym \textit{epoką}. Definicje i pojęcia w tym rozdziale pochodzą z~\cite{bott1},~\cite{bott2},~\cite{kotlowski}.

\newpage
\section{Algorytmy spadku wzdłuż gradientu}

t tekst tekst tekst tekst tekst tekst tekst tekst tekst tekst tekst tekst tekst tekst tekst ekst tekst tekst tekst tekst t tekst tekst tekst tekst tekst tekst tekst tekst tekst tekst tekst tekst tekst tekst tekst ekst tekst tekst tekst tekst t tekst tekst tekst tekst tekst tekst tekst t tekst tekst tekst tekst tekst tekst tekst tekst tekst tekst tekst tekst tekst tekst tekst ekst tekst tekst tekst tekst t tekst tekst tekst tekst tekst tekst tekst tekst tekst tekst tekst tekst tekst tekst tekst ekst tekst tekst tekst tekst t tekst tekst tekst tekst tekst tekst tekst

\begin{center}
\textbf{Metoda spadku wzdłuż gradientu I (Cauchy’ego)}
\end{center}
Minimalizacja funkcji $L(w)$:
\begin{itemize}
\item Zaczynamy od wybranego rozwiązania startowego, np. $w_{0} = 0$.
\item Dla $k = 1, 2, \dots$ aż do zbieżności
	\begin{itemize}
	\item Wyznaczamy gradient w punkcie $w_{k-1}, \nabla_{L}(w_{k-1})$.
	\item Robimy krok wzdłuż negatywnego gradientu: $$w_{k} = w_{k-1} - \alpha_{k}\nabla_{L}(w_{k-1}). $$
	\end{itemize}
\end{itemize}

\begin{center}
\textbf{Metoda spadku wzdłuż gradientu II (Newtona-Raphsona)}
\end{center}
Minimalizacja funkcji $L(w)$:
\begin{itemize}
\item Zaczynamy od wybranego rozwiązania startowego, np. $w_{0} = 0$.
\item Dla $k = 1, 2, \dots$ aż do zbieżności
	\begin{itemize}
	\item Wyznaczamy gradient w punkcie $w_{k-1}, \nabla_{L}(w_{k-1})$ \\ i odwrotność Hesjanu $(D_{L}^{2}(w_{k-1}))^{-1}$.
	\item Robimy krok wzdłuż negatywnego gradientu z zadanym krokiem przez Hesjan: $$w_{k} = w_{k-1} - (D_{L}^{2}(w_{k-1}))^{-1}\nabla_{L}(w_{k-1}). $$
	\end{itemize}
\end{itemize}
\section{Algorytm stochastycznego spadku wzdłuż gradientu I}\label{SGD}
\begin{center}
\textbf{Metoda stochastycznego spadku wzdłuż gradientu I}
\end{center}
Minimalizacja funkcji $L(w)$:
\begin{itemize}
\item Zaczynamy od wybranego rozwiązania startowego, np. $w_{0} = 0$.
\item Dla $k = 1, 2, \dots$ aż do zbieżności
	\begin{itemize}
	\item Wylosuj $i \in \{1,\dots,n\}$
	\item Wyznaczamy gradient funkcji $\ell_{i}$ w punkcie $w_{k-1}, \nabla_{\ell_{i}}(w_{k-1})$.
	\item Robimy krok wzdłuż negatywnego gradientu: $$w_{k} = w_{k-1} - \alpha_{k}\nabla_{\ell_{i}}(w_{k-1}).$$
	\end{itemize}
\end{itemize}


Stochastyczny spadek gradientu to popularny algorytm wykorzystywany do estymacji współczynników w szerokiej gamie modeli uczenia maszynowego takich jak maszyny wektorów podpierających (\textit{ang. Support Vector Machines}), regresja logistyczna czy modele graficzne~\cite{finkel}. W~połączeniu z algorytmem propagacji wstecznej jest standardowym algorytmem w~trenowaniu sztucznych sieci neuronowych. Algorytm stochastycznego spadku gradientu był używany już od 1960 przy estymacji współczynników w modelu regresji liniowej, oryginalnie znany pod nazwą \textit{ADALINE} \cite{ADALINE}. Kolejnym znanym popularnym algorytmem wykorzystującym stochastyczny spadek gradientu jest filtr adaptacyjny najmniejszych średnich kwadratów \cite{widrow2} (\textit{ang.~least mean squares (LMS) adaptive filter}), który również został wynaleziony w 1960 przez Bernard Widrowa, który jest także twórcą \textit{ADALINE}.

Idea algorytmu stochastycznego spadku gradientu jest następująca: zamiast obliczać gradient na całej funkcji $L$, w danym kroku oblicz
gradient tylko na pojedynczym elemencie $\ell_{i}$. Nazwa \textit{stochastyczny} bierze się stąd, iż oryginalnie wybiera
się element $\ell_{i}$ losowo. W praktyce zwykle przechodzi się po całym zbiorze danych w losowej kolejności lub, o ile to możliwe, w~kolejności chronologicznej obserwacji.



\subsection{Właściwości stochastycznego spadku wzdłuż gradientu}

\subsubsection{Zalety}
\begin{itemize}
\item \textcolor{orange}{Szybkość}: obliczenie gradientu wymaga wzięcia tylko jednej
obserwacji.
\item \textcolor{orange}{Skalowalność}: cały zbiór danych nie musi nawet znajdować się
w pamięci operacyjnej.
\item \textcolor{orange}{Prostota}: gradient funkcji  $\ell_{i}$ daje bardzo prosty wzór na
modyfikacje wag.
\end{itemize}

\subsubsection{Wady}
\begin{itemize}
\item \textcolor{orange}{Wolna zbieżność}: czasem gradient stochastyczny zbiega wolno
i wymaga wielu iteracji po zbiorze uczącym.
\item \textcolor{orange}{Problem z ustaleniem długości kroku $k$}: wyznaczenie $k$
przez przeszukiwanie liniowe nie przynosi dobrych rezultatów,
ponieważ optymalizujemy oryginalnej funkcji $L$ tylko jej jeden
składnik $\ell_{i}$.
\end{itemize}

\subsubsection{Stochastyczny gradient w praktyce}

\begin{itemize}
\item Zwykle nie losuje się obserwacji, ale przechodzi się po zbiorze
danych w losowej kolejności.
\item Zbieżność wymaga często przejścia parokrotnie po całym
zbiorze danych (jednokrotne przejście nazywa się epoka).
\item Metody ustalania współczynników długości kroku $k$:
\begin{itemize}
\item Ustalamy \textcolor{orange}{stała wartość} $\alpha_{k} = \alpha$ \\ Zwykle tak się robi w praktyce, działa dobrze ale wymaga ustalenia $\alpha_{k}$ metodą prób i błędów.
\item Bierzemy wartość kroku malejącą jak \textcolor{orange}{$\sim \frac{1}{\sqrt{k}}: \alpha_{k} = \frac{\alpha}{\sqrt{k}}$} \\
 Zapewniona zbieżność, ale czasem może zbiegać zbyt wolno.
\end{itemize}
\end{itemize}