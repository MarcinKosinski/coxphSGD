\chapter{Numeryczne metody estymacji}
Poszukujemy rozwiazan równosci
$$ \delta ln Ln  / \delta \theta =  0.$$
Tym razem w ogólnym przypadku zwykle nie znajdziemy analitycznego rozwiazania. W
zwiazku z tym jestesmy zdani na metody iteracyjne. Poza tym, byc moze rozwiazanie problemu
nie istnieje albo istnieje ich wiele. Zwykle uzywa sie do tego celu, tj. znalezienia
rozwiazania, metody Newtona, zwykle w literaturze statystycznej w zastosowaniu do tego
problemu, nazywanej metoda Newtona-Raphsona. W efekcie w zasadzie dla kazdego modelu
z osobna nalezy badac własnosci asymptotyczne estymatora najwiekszej wiarygodnosci.
\subsection{Ogólne pojęcia związane ze zbieżnością algorytmu}
\subsubsection{Warunki stopu itp}
\section{Algorytmy spadku wzdłuż gradientu}
\subsection{Algorytm Cauchy'ego}
\subsection{Algorytm Raphsona-Newtona}\label{R-N}
\section{Algorytmy stochastycznego spadku wzdłuż gradientu}\label{SGD}
\subsection{Metoda estymacji stochastycznego spadku gradientu I}
\subsubsection{Algorytm SGD}
\subsection{Metoda estymacji stochastycznego spadku gradientu II}
