\begin{thebibliography}{99}

\bibitem[1]{aldrich1} Aldrich J., (1997) \textit{R. A. Fisher and the Making of Maximum Likelihood 1912 – 1922}, Statistical Science
1997, Vol. 12, No. 3, 162-176.

\bibitem[2]{biecek1} Biecek P., (2011) \textit{Przewodnik po pakiecie R}, Rozprawa doktorska, Oficyna Wydawnicza GiS, wydanie II.

\bibitem[3]{bott1} Bottou L., (2010) \textit{Large-Scale Machine Learning with Stochastic Gradient Descent}.

\bibitem[4]{bott1} Bottou L., (2012) \textit{Stochastic Gradient Descent Tricks}.

\bibitem[5]{fisher1} Fisher R. A., (1912) \textit{An absolute criterion for fitting frequency curves}. 

\bibitem[6]{fisher2} Fisher R. A., (1922) \textit{On the mathematical foundations of theoretical statistics}, Philos. Trans. Roy. Soc. London Ser. A 222 309-368.


\bibitem[7]{gauss1} Gauss C. F., (1809) \textit{Theoria Motus Corporum Coelestium}.

\bibitem[8]{gagol1} Gągolewski M., (2014) \textit{Programowanie w języku R}, Wydawnictwo Naukowe PWN.


\bibitem[9]{hutch1} Hutchinson J. B., (1928) \textit{The Application of the "Method of Maximum Likelihood" to the Estimation of Linkage}, Genetics. 1929 Nov; 14(6): 519–537.


\bibitem[10]{kenward1} Kenward M. G., Lesaffre E. and Molenberghs G., (1994) \textit{An Application of Maximum Likelihood and Generalized Estimating Equations to the Analysis of Ordinal Data from a Longitudinal Study with Cases Missing at Random}, Biometrics
Vol. 50, No. 4 (Dec., 1994), pp. 945-953.

\bibitem[11]{legendre1} Legendre A. M., (1804) \textit{Nouvelles m´ethods pour la d´etermination des orbites des com`etes}.

\bibitem[12]{millar1} Millar R. B., (2011) \textit{Maximum Likelihood Estimation and Inference: With Examples in R, SAS and ADMB, chapter 6. Some Widely Used Applications of Maximum Likelihood}, John Wiley \& Sons, Ltd.

\bibitem[13]{rydl1} Rydlewski J., (2009) \textit{Estymatory Największej Wiarogodności w Uogólnionych Modelach Regresji Nieliniowej}, Rozprawa doktorska.

 



\end{thebibliography}