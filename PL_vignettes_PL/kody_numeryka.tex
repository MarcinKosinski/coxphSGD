\documentclass[]{article}
\usepackage{lmodern}
\usepackage{amssymb,amsmath}
\usepackage{ifxetex,ifluatex}
\usepackage{fixltx2e} % provides \textsubscript
\ifnum 0\ifxetex 1\fi\ifluatex 1\fi=0 % if pdftex
  \usepackage[T1]{fontenc}
  \usepackage[utf8]{inputenc}
\else % if luatex or xelatex
  \ifxetex
    \usepackage{mathspec}
    \usepackage{xltxtra,xunicode}
  \else
    \usepackage{fontspec}
  \fi
  \defaultfontfeatures{Mapping=tex-text,Scale=MatchLowercase}
  \newcommand{\euro}{€}
\fi
% use upquote if available, for straight quotes in verbatim environments
\IfFileExists{upquote.sty}{\usepackage{upquote}}{}
% use microtype if available
\IfFileExists{microtype.sty}{%
\usepackage{microtype}
\UseMicrotypeSet[protrusion]{basicmath} % disable protrusion for tt fonts
}{}
\usepackage[margin=1in]{geometry}
\usepackage{color}
\usepackage{fancyvrb}
\newcommand{\VerbBar}{|}
\newcommand{\VERB}{\Verb[commandchars=\\\{\}]}
\DefineVerbatimEnvironment{Highlighting}{Verbatim}{commandchars=\\\{\}}
% Add ',fontsize=\small' for more characters per line
\usepackage{framed}
\definecolor{shadecolor}{RGB}{248,248,248}
\newenvironment{Shaded}{\begin{snugshade}}{\end{snugshade}}
\newcommand{\KeywordTok}[1]{\textcolor[rgb]{0.13,0.29,0.53}{\textbf{{#1}}}}
\newcommand{\DataTypeTok}[1]{\textcolor[rgb]{0.13,0.29,0.53}{{#1}}}
\newcommand{\DecValTok}[1]{\textcolor[rgb]{0.00,0.00,0.81}{{#1}}}
\newcommand{\BaseNTok}[1]{\textcolor[rgb]{0.00,0.00,0.81}{{#1}}}
\newcommand{\FloatTok}[1]{\textcolor[rgb]{0.00,0.00,0.81}{{#1}}}
\newcommand{\CharTok}[1]{\textcolor[rgb]{0.31,0.60,0.02}{{#1}}}
\newcommand{\StringTok}[1]{\textcolor[rgb]{0.31,0.60,0.02}{{#1}}}
\newcommand{\CommentTok}[1]{\textcolor[rgb]{0.56,0.35,0.01}{\textit{{#1}}}}
\newcommand{\OtherTok}[1]{\textcolor[rgb]{0.56,0.35,0.01}{{#1}}}
\newcommand{\AlertTok}[1]{\textcolor[rgb]{0.94,0.16,0.16}{{#1}}}
\newcommand{\FunctionTok}[1]{\textcolor[rgb]{0.00,0.00,0.00}{{#1}}}
\newcommand{\RegionMarkerTok}[1]{{#1}}
\newcommand{\ErrorTok}[1]{\textbf{{#1}}}
\newcommand{\NormalTok}[1]{{#1}}
\ifxetex
  \usepackage[setpagesize=false, % page size defined by xetex
              unicode=false, % unicode breaks when used with xetex
              xetex]{hyperref}
\else
  \usepackage[unicode=true]{hyperref}
\fi
\hypersetup{breaklinks=true,
            bookmarks=true,
            pdfauthor={Marcin Kosinski},
            pdftitle={Untitled},
            colorlinks=true,
            citecolor=blue,
            urlcolor=blue,
            linkcolor=magenta,
            pdfborder={0 0 0}}
\urlstyle{same}  % don't use monospace font for urls
\setlength{\parindent}{0pt}
\setlength{\parskip}{6pt plus 2pt minus 1pt}
\setlength{\emergencystretch}{3em}  % prevent overfull lines
\setcounter{secnumdepth}{0}

%%% Use protect on footnotes to avoid problems with footnotes in titles
\let\rmarkdownfootnote\footnote%
\def\footnote{\protect\rmarkdownfootnote}

%%% Change title format to be more compact
\usepackage{titling}

% Create subtitle command for use in maketitle
\newcommand{\subtitle}[1]{
  \posttitle{
    \begin{center}\large#1\end{center}
    }
}

\setlength{\droptitle}{-2em}
  \title{Untitled}
  \pretitle{\vspace{\droptitle}\centering\huge}
  \posttitle{\par}
  \author{Marcin Kosinski}
  \preauthor{\centering\large\emph}
  \postauthor{\par}
  \predate{\centering\large\emph}
  \postdate{\par}
  \date{25.10.2015}



\begin{document}

\maketitle


\begin{Shaded}
\begin{Highlighting}[]
\NormalTok{logitGD <-}\StringTok{ }\NormalTok{function(y, x, }\DataTypeTok{optim.method =} \StringTok{"GDI"}\NormalTok{, }\DataTypeTok{eps =} \FloatTok{10e-4}\NormalTok{,}
                    \DataTypeTok{max.iter =} \DecValTok{100}\NormalTok{, }\DataTypeTok{alpha =} \NormalTok{function(t)\{}\DecValTok{1}\NormalTok{/t\}, }\DataTypeTok{beta_0 =} \KeywordTok{c}\NormalTok{(}\DecValTok{0}\NormalTok{,}\DecValTok{0}\NormalTok{))\{}
  \KeywordTok{stopifnot}\NormalTok{(}\KeywordTok{length}\NormalTok{(y) ==}\StringTok{ }\KeywordTok{length}\NormalTok{(x) &}\StringTok{ }\NormalTok{optim.method %in%}\StringTok{ }\KeywordTok{c}\NormalTok{(}\StringTok{"GDI"}\NormalTok{, }\StringTok{"GDII"}\NormalTok{, }\StringTok{"SGDI"}\NormalTok{)}
            \NormalTok{&}\StringTok{ }\KeywordTok{is.numeric}\NormalTok{(}\KeywordTok{c}\NormalTok{(max.iter, eps, x)) &}\StringTok{ }\KeywordTok{all}\NormalTok{(}\KeywordTok{c}\NormalTok{(eps, max.iter) >}\StringTok{ }\DecValTok{0}\NormalTok{) &}
\StringTok{              }\KeywordTok{is.function}\NormalTok{(alpha))}
  \NormalTok{iter <-}\StringTok{ }\DecValTok{0}
  \NormalTok{err <-}\StringTok{ }\KeywordTok{list}\NormalTok{()}
  \NormalTok{err[[iter}\DecValTok{+1}\NormalTok{]] <-}\StringTok{ }\NormalTok{eps}\DecValTok{+1}
  \NormalTok{w_old <-}\StringTok{ }\NormalTok{beta_0}

  \NormalTok{res <-}\KeywordTok{list}\NormalTok{()}
  \NormalTok{while(iter <}\StringTok{ }\NormalTok{max.iter &&}\StringTok{ }\NormalTok{(}\KeywordTok{abs}\NormalTok{(err[[}\KeywordTok{ifelse}\NormalTok{(iter==}\DecValTok{0}\NormalTok{,}\DecValTok{1}\NormalTok{,iter)]]) >}\StringTok{ }\NormalTok{eps))\{}

    \NormalTok{iter <-}\StringTok{ }\NormalTok{iter +}\StringTok{ }\DecValTok{1}
    \NormalTok{if (optim.method ==}\StringTok{ "GDI"}\NormalTok{)\{}
      \NormalTok{w_new <-}\StringTok{ }\NormalTok{w_old +}\StringTok{ }\KeywordTok{alpha}\NormalTok{(iter)*}\KeywordTok{updateWeightsGDI}\NormalTok{(y, x, w_old)}
    \NormalTok{\}}
    \NormalTok{if (optim.method ==}\StringTok{ "GDII"}\NormalTok{)\{}
      \NormalTok{w_new <-}\StringTok{ }\NormalTok{w_old +}\StringTok{ }\KeywordTok{as.vector}\NormalTok{(}\KeywordTok{inverseHessianGDII}\NormalTok{(x, w_old)%*%}
\StringTok{                                   }\KeywordTok{updateWeightsGDI}\NormalTok{(y, x, w_old))}
    \NormalTok{\}}
    \NormalTok{if (optim.method ==}\StringTok{ "SGDI"}\NormalTok{)\{}
      \NormalTok{w_new <-}\StringTok{ }\NormalTok{w_old +}\StringTok{ }\KeywordTok{alpha}\NormalTok{(iter)*}\KeywordTok{updateWeightsSGDI}\NormalTok{(y[iter], x[iter], w_old)}
    \NormalTok{\}}
    \NormalTok{res[[iter]] <-}\StringTok{ }\NormalTok{w_new}
    \NormalTok{err[[iter]] <-}\StringTok{ }\KeywordTok{sqrt}\NormalTok{(}\KeywordTok{sum}\NormalTok{((w_new -}\StringTok{ }\NormalTok{w_old)^}\DecValTok{2}\NormalTok{))}

    \NormalTok{w_old <-}\StringTok{ }\NormalTok{w_new}

  \NormalTok{\}}
  \KeywordTok{return}\NormalTok{(}\KeywordTok{list}\NormalTok{(}\DataTypeTok{steps =} \KeywordTok{c}\NormalTok{(}\KeywordTok{list}\NormalTok{(beta_0),res), }\DataTypeTok{errors =} \KeywordTok{c}\NormalTok{(}\KeywordTok{list}\NormalTok{(}\KeywordTok{c}\NormalTok{(}\DecValTok{0}\NormalTok{,}\DecValTok{0}\NormalTok{)),err)))}
\NormalTok{\}}

\NormalTok{updateWeightsGDI <-}\StringTok{ }\NormalTok{function(y, x, w_old)\{}
  \NormalTok{(}\DecValTok{1}\NormalTok{/}\KeywordTok{length}\NormalTok{(y))*}\KeywordTok{c}\NormalTok{(}\KeywordTok{sum}\NormalTok{(y-}\KeywordTok{p}\NormalTok{(w_old, x)), }\KeywordTok{sum}\NormalTok{(x*(y-}\KeywordTok{p}\NormalTok{(w_old, x))))}
  \CommentTok{#c(sum(y-p(w_old, x)), sum(x*(y-p(w_old, x))))}
\NormalTok{\}}

\NormalTok{updateWeightsSGDI <-}\StringTok{ }\NormalTok{function(y_i, x_i, w_old)\{}
  \KeywordTok{c}\NormalTok{(y_i-}\KeywordTok{p}\NormalTok{(w_old, x_i), x_i*(y_i-}\KeywordTok{p}\NormalTok{(w_old, x_i)))}
\NormalTok{\}}

\NormalTok{p <-}\StringTok{ }\NormalTok{function(w_old, x_i)\{}
  \DecValTok{1}\NormalTok{/(}\DecValTok{1}\NormalTok{+}\KeywordTok{exp}\NormalTok{(-w_old[}\DecValTok{1}\NormalTok{]-w_old[}\DecValTok{2}\NormalTok{]*x_i))}
\NormalTok{\}}

\NormalTok{inverseHessianGDII <-}\StringTok{ }\NormalTok{function(x, w_old)\{}
  \KeywordTok{solve}\NormalTok{(}
    \KeywordTok{matrix}\NormalTok{(}\KeywordTok{c}\NormalTok{(}
      \KeywordTok{sum}\NormalTok{(}\KeywordTok{p}\NormalTok{(w_old, x)*(}\DecValTok{1}\NormalTok{-}\KeywordTok{p}\NormalTok{(w_old, x))),}
      \KeywordTok{sum}\NormalTok{(x*}\KeywordTok{p}\NormalTok{(w_old, x)*(}\DecValTok{1}\NormalTok{-}\KeywordTok{p}\NormalTok{(w_old, x))),}
      \KeywordTok{sum}\NormalTok{(x*}\KeywordTok{p}\NormalTok{(w_old, x)*(}\DecValTok{1}\NormalTok{-}\KeywordTok{p}\NormalTok{(w_old, x))),}
      \KeywordTok{sum}\NormalTok{(x*x*}\KeywordTok{p}\NormalTok{(w_old, x)*(}\DecValTok{1}\NormalTok{-}\KeywordTok{p}\NormalTok{(w_old, x)))}
    \NormalTok{),}
    \DataTypeTok{nrow =}\DecValTok{2} \NormalTok{)}
  \NormalTok{)}
\NormalTok{\}}
\end{Highlighting}
\end{Shaded}

\begin{Shaded}
\begin{Highlighting}[]
                    \CommentTok{# wstępna inicjalizacja parametrów}
\NormalTok{eps =}\StringTok{ }\FloatTok{1e-5}                               \CommentTok{# warunek stopu.}

\NormalTok{n =}\StringTok{ }\KeywordTok{length}\NormalTok{(data)                         }\CommentTok{# data jest listą ramek danych.}

\NormalTok{diff =}\StringTok{ }\NormalTok{eps +}\StringTok{ }\DecValTok{1}                           \CommentTok{# różnice w oszacowaniach parametrów}
                                         \CommentTok{# między kolejnymi krokami.}

\NormalTok{learningRates =}\StringTok{ }\NormalTok{function(x) }\DecValTok{1}\NormalTok{/x          }\CommentTok{# długości kroku algorytmu.}

\NormalTok{beta_old =}\StringTok{ }\KeywordTok{numeric}\NormalTok{(}\DecValTok{0}\NormalTok{, }\DataTypeTok{length =} \NormalTok{k)        }\CommentTok{# punkt startowy dlugosci k,}
                                         \CommentTok{# gdzie k to liczba zmiennych}
                                         \CommentTok{# objaśniających w modelu.}

                              \CommentTok{# estymacja}
\NormalTok{i =}\StringTok{ }\DecValTok{1}                                    \CommentTok{# iterator kroku algorytmu}
\NormalTok{while(i <=}\StringTok{ }\NormalTok{n |}\StringTok{ }\NormalTok{diff <}\StringTok{ }\NormalTok{eps) do            }\CommentTok{# do zbieżności lub wyczerpania zbiorów}
  \NormalTok{batch =}\StringTok{ }\NormalTok{data[[i]]}
  
  \NormalTok{beta_new =}\StringTok{ }\NormalTok{beta_old -}\StringTok{ }\KeywordTok{learningRates}\NormalTok{(i) *}\StringTok{ }\KeywordTok{U_Batch}\NormalTok{(batch) }
                                         \CommentTok{# U_Batch to częściowa funkcja}
                                         \CommentTok{# log-wiarogdności dla zaobserwowanego}
                                         \CommentTok{# zbioru `batch`}
  
  \NormalTok{diff =}\StringTok{ }\KeywordTok{euclidean_dist}\NormalTok{(beta_new, beta_old) }\CommentTok{# odległość euklidesowa}
  
  \NormalTok{beta_old =}\StringTok{ }\NormalTok{beta_new }
  
  \NormalTok{i =}\StringTok{ }\NormalTok{i +}\StringTok{ }\DecValTok{1}
\NormalTok{end while}
\NormalTok{return beta_new}
\end{Highlighting}
\end{Shaded}

\begin{Shaded}
\begin{Highlighting}[]
\NormalTok{coxphSGD <-}\StringTok{ }\NormalTok{function(formula, data, }
                    \DataTypeTok{learningRates =} \NormalTok{function(x)\{}\DecValTok{1}\NormalTok{/x\},}
                    \DataTypeTok{beta_0 =} \DecValTok{0}\NormalTok{, }\DataTypeTok{epsilon =} \FloatTok{1e-5} \NormalTok{) \{}
  \KeywordTok{checkArguments}\NormalTok{(formula, data, learningRates,}
                  \NormalTok{beta_0, epsilon) ->}\StringTok{ }\NormalTok{beta_old }\CommentTok{# check arguments}

  \NormalTok{n <-}\StringTok{ }\KeywordTok{length}\NormalTok{(data)}
  \NormalTok{diff <-}\StringTok{ }\NormalTok{epsilon +}\StringTok{ }\DecValTok{1}
  \NormalTok{i <-}\StringTok{ }\DecValTok{1}
  \NormalTok{beta_new <-}\StringTok{ }\KeywordTok{list}\NormalTok{() }\CommentTok{# steps are saved in a list so that they can}
                     \CommentTok{# be tracked in the future}
  \CommentTok{# estimate}
  \NormalTok{while(i <=}\StringTok{ }\NormalTok{n &}\StringTok{ }\NormalTok{diff >}\StringTok{ }\NormalTok{epsilon) \{}
    \CommentTok{#tryCatch(\{}
    \NormalTok{beta_new[[i]] <-}\StringTok{ }\KeywordTok{coxphSGD_batch}\NormalTok{(}\DataTypeTok{formula =} \NormalTok{formula, }\DataTypeTok{data =} \NormalTok{data[[i]],}
                                    \DataTypeTok{learningRate =} \KeywordTok{learningRates}\NormalTok{(i),}
                                    \DataTypeTok{beta =} \NormalTok{beta_old)}
    
    \NormalTok{diff <-}\StringTok{ }\KeywordTok{sqrt}\NormalTok{(}\KeywordTok{sum}\NormalTok{((beta_new[[i]] -}\StringTok{ }\NormalTok{beta_old)^}\DecValTok{2}\NormalTok{))}
    \NormalTok{beta_old <-}\StringTok{ }\NormalTok{beta_new[[i]]}
    \NormalTok{i <-}\StringTok{ }\NormalTok{i +}\StringTok{ }\DecValTok{1}  
    \CommentTok{#\}, error = function(cond) \{i <<- n + 1\})}
  \NormalTok{\}}
  
  \CommentTok{# return results}
  \NormalTok{fit <-}\StringTok{ }\KeywordTok{list}\NormalTok{()}
  \NormalTok{fit$Call <-}\StringTok{ }\KeywordTok{match.call}\NormalTok{()}
  \NormalTok{fit$coefficients <-}\StringTok{ }\NormalTok{beta_new}
  \NormalTok{fit$epsilon <-}\StringTok{ }\NormalTok{epsilon}
  \NormalTok{fit$learningRates <-}\StringTok{ }\NormalTok{learningRates}
  \NormalTok{fit$steps <-}\StringTok{ }\NormalTok{i}
  \KeywordTok{class}\NormalTok{(fit) <-}\StringTok{ "coxphSGD"}
  \NormalTok{fit}
\NormalTok{\}}

\NormalTok{coxphSGD_batch <-}\StringTok{ }\NormalTok{function(formula, data, learningRate, beta)\{}
  
  \CommentTok{# Parameter identification as in  `survival::coxph()`.}
  \NormalTok{Call <-}\StringTok{ }\KeywordTok{match.call}\NormalTok{()}
  \NormalTok{indx <-}\StringTok{ }\KeywordTok{match}\NormalTok{(}\KeywordTok{c}\NormalTok{(}\StringTok{"formula"}\NormalTok{, }\StringTok{"data"}\NormalTok{),}
                \KeywordTok{names}\NormalTok{(Call), }\DataTypeTok{nomatch =} \DecValTok{0}\NormalTok{)}
  \NormalTok{if (indx[}\DecValTok{1}\NormalTok{] ==}\StringTok{ }\DecValTok{0}\NormalTok{) }
      \KeywordTok{stop}\NormalTok{(}\StringTok{"A formula argument is required"}\NormalTok{)}
  \NormalTok{temp <-}\StringTok{ }\NormalTok{Call[}\KeywordTok{c}\NormalTok{(}\DecValTok{1}\NormalTok{, indx)]}
  \NormalTok{temp[[}\DecValTok{1}\NormalTok{]] <-}\StringTok{ }\KeywordTok{as.name}\NormalTok{(}\StringTok{"model.frame"}\NormalTok{)}
  
  \NormalTok{mf <-}\StringTok{ }\KeywordTok{eval}\NormalTok{(temp, }\KeywordTok{parent.frame}\NormalTok{())}
  \NormalTok{Y <-}\StringTok{ }\KeywordTok{model.extract}\NormalTok{(mf, }\StringTok{"response"}\NormalTok{)}
  
  \NormalTok{if (!}\KeywordTok{inherits}\NormalTok{(Y, }\StringTok{"Surv"}\NormalTok{)) }
      \KeywordTok{stop}\NormalTok{(}\StringTok{"Response must be a survival object"}\NormalTok{)}
  \NormalTok{type <-}\StringTok{ }\KeywordTok{attr}\NormalTok{(Y, }\StringTok{"type"}\NormalTok{)}
  
  \NormalTok{if (type !=}\StringTok{ "right"} \NormalTok{&&}\StringTok{ }\NormalTok{type !=}\StringTok{ "counting"}\NormalTok{) }
      \KeywordTok{stop}\NormalTok{(}\KeywordTok{paste}\NormalTok{(}\StringTok{"Cox model doesn't support }\CharTok{\textbackslash{}"}\StringTok{"}\NormalTok{, type, }\StringTok{"}\CharTok{\textbackslash{}"}\StringTok{ survival data"}\NormalTok{, }
          \DataTypeTok{sep =} \StringTok{""}\NormalTok{))}
  
  \CommentTok{# collect times, status, variables and reorder samples }
  \CommentTok{# to make the algorithm more clear to read and track}
  \KeywordTok{cbind}\NormalTok{(}\DataTypeTok{not_censored =} \DecValTok{1} \NormalTok{-}\StringTok{ }\KeywordTok{unclass}\NormalTok{(Y)[,}\DecValTok{2}\NormalTok{],}
        \DataTypeTok{times =} \KeywordTok{unclass}\NormalTok{(Y)[,}\DecValTok{1}\NormalTok{],}
        \NormalTok{mf[, -}\DecValTok{1}\NormalTok{]) %>%}
\StringTok{    }\KeywordTok{arrange}\NormalTok{(times) ->}\StringTok{ }\NormalTok{batchData}
  
  \CommentTok{# calculate the log-likelihood for this batch sample}
  \NormalTok{partial_sum <-}\StringTok{ }\KeywordTok{list}\NormalTok{()}
  
  \NormalTok{for(k in }\DecValTok{1}\NormalTok{:}\KeywordTok{nrow}\NormalTok{(batchData)) \{}
    
    \CommentTok{# risk set for current time/observation}
    \NormalTok{risk_set <-}\StringTok{ }\NormalTok{batchData %>%}
\StringTok{      }\KeywordTok{filter}\NormalTok{(times <=}\StringTok{ }\NormalTok{batchData$times[k])}
    
    \NormalTok{nominator <-}\StringTok{ }\KeywordTok{apply}\NormalTok{(risk_set[, -}\KeywordTok{c}\NormalTok{(}\DecValTok{1}\NormalTok{,}\DecValTok{2}\NormalTok{)], }\DataTypeTok{MARGIN =} \DecValTok{1}\NormalTok{, function(element)\{}
      \NormalTok{element *}\StringTok{ }\KeywordTok{exp}\NormalTok{(element *}\StringTok{ }\NormalTok{beta)}
    \NormalTok{\}) %>%}
\StringTok{      }\NormalTok{t %>%}
\StringTok{      }\KeywordTok{colSums}\NormalTok{()}
    
    \NormalTok{denominator <-}\StringTok{ }\KeywordTok{apply}\NormalTok{(risk_set[, -}\KeywordTok{c}\NormalTok{(}\DecValTok{1}\NormalTok{,}\DecValTok{2}\NormalTok{)], }\DataTypeTok{MARGIN =} \DecValTok{1}\NormalTok{, function(element)\{}
      \KeywordTok{exp}\NormalTok{(element *}\StringTok{ }\NormalTok{beta)}
    \NormalTok{\}) %>%}
\StringTok{      }\NormalTok{t %>%}
\StringTok{      }\KeywordTok{colSums}\NormalTok{()}
    
    \NormalTok{partial_sum[[k]] <-}\StringTok{ }
\StringTok{      }\NormalTok{batchData[k, }\StringTok{"not_censored"}\NormalTok{] *}\StringTok{ }\NormalTok{(batchData[k, -}\KeywordTok{c}\NormalTok{(}\DecValTok{1}\NormalTok{,}\DecValTok{2}\NormalTok{)] -}\StringTok{ }\NormalTok{nominator/denominator)}
    
  \NormalTok{\}}
  
  \KeywordTok{do.call}\NormalTok{(rbind, partial_sum) %>%}
\StringTok{    }\KeywordTok{colSums}\NormalTok{() ->}\StringTok{ }\NormalTok{U_batch}
  
  \NormalTok{beta_out <-}\StringTok{ }\NormalTok{beta +}\StringTok{ }\NormalTok{learningRate *}\StringTok{ }\NormalTok{U_batch}
  
  \KeywordTok{return}\NormalTok{(beta_out)}
\NormalTok{\}}
  
\NormalTok{checkArguments <-}\StringTok{ }\NormalTok{function(formula, data, learningRates,}
                             \NormalTok{beta_0, epsilon) \{}
  \KeywordTok{assert_that}\NormalTok{(}\KeywordTok{is.list}\NormalTok{(data) &}\StringTok{ }\KeywordTok{length}\NormalTok{(data) >}\StringTok{ }\DecValTok{0}\NormalTok{)}
  \KeywordTok{assert_that}\NormalTok{(}\KeywordTok{length}\NormalTok{(}\KeywordTok{unique}\NormalTok{(}\KeywordTok{unlist}\NormalTok{(}\KeywordTok{lapply}\NormalTok{(data, ncol)))) ==}\StringTok{ }\DecValTok{1}\NormalTok{)}
  \CommentTok{# + check names and types for every variables}
  \KeywordTok{assert_that}\NormalTok{(}\KeywordTok{is.function}\NormalTok{(learningRates))}
  \KeywordTok{assert_that}\NormalTok{(}\KeywordTok{is.numeric}\NormalTok{(epsilon))}
  \KeywordTok{assert_that}\NormalTok{(}\KeywordTok{is.numeric}\NormalTok{(beta_0))}
  
    \CommentTok{# check length of the start parameter}
  \NormalTok{if (}\KeywordTok{length}\NormalTok{(beta_0) ==}\StringTok{ }\DecValTok{1}\NormalTok{) \{}
    \NormalTok{beta_0 <-}\StringTok{ }\KeywordTok{rep}\NormalTok{(beta_0, }\KeywordTok{as.character}\NormalTok{(formula)[}\DecValTok{3}\NormalTok{] %>%}
\StringTok{                    }\KeywordTok{strsplit}\NormalTok{(}\StringTok{"}\CharTok{\textbackslash{}\textbackslash{}}\StringTok{+"}\NormalTok{) %>%}
\StringTok{                    }\NormalTok{unlist %>%}
\StringTok{                    }\NormalTok{length)}
  \NormalTok{\}}
  \KeywordTok{return}\NormalTok{(beta_0)}
\NormalTok{\}}
\end{Highlighting}
\end{Shaded}

\begin{Shaded}
\begin{Highlighting}[]
\NormalTok{x <-}\StringTok{ }\KeywordTok{rnorm}\NormalTok{(}\DecValTok{1000}\NormalTok{)}
\NormalTok{z <-}\StringTok{ }\DecValTok{2} \NormalTok{+}\StringTok{ }\DecValTok{3}\NormalTok{*x}
\NormalTok{pr <-}\StringTok{ }\DecValTok{1}\NormalTok{/(}\DecValTok{1}\NormalTok{+}\KeywordTok{exp}\NormalTok{(-z))}
\NormalTok{y <-}\StringTok{ }\KeywordTok{rbinom}\NormalTok{(}\DecValTok{1000}\NormalTok{,}\DecValTok{1}\NormalTok{,pr)}


\KeywordTok{logitGD}\NormalTok{(y, x, }\DataTypeTok{optim.method =} \StringTok{"GDI"}\NormalTok{, }\DataTypeTok{eps =} \FloatTok{10e-5}\NormalTok{, }\DataTypeTok{max.iter =} \DecValTok{500}\NormalTok{)$steps ->}\StringTok{ }\NormalTok{GDI}
\KeywordTok{logitGD}\NormalTok{(y, x, }\DataTypeTok{optim.method =} \StringTok{"GDII"}\NormalTok{, }\DataTypeTok{eps =} \FloatTok{10e-5}\NormalTok{, }\DataTypeTok{max.iter =} \DecValTok{500}\NormalTok{)$steps ->}\StringTok{ }\NormalTok{GDII}

\NormalTok{ind <-}\StringTok{ }\KeywordTok{sample}\NormalTok{(}\KeywordTok{length}\NormalTok{(y))}
\KeywordTok{logitGD}\NormalTok{(y[ind], x[ind], }\DataTypeTok{optim.method =} \StringTok{"SGDI"}\NormalTok{,}
        \DataTypeTok{max.iter =} \DecValTok{500}\NormalTok{, }\DataTypeTok{eps =} \FloatTok{10e-5}\NormalTok{)$steps ->}\StringTok{ }\NormalTok{SGDI}\FloatTok{.1}
\NormalTok{ind2 <-}\StringTok{ }\KeywordTok{sample}\NormalTok{(}\KeywordTok{length}\NormalTok{(y))}
\KeywordTok{logitGD}\NormalTok{(y[ind2], x[ind2], }\DataTypeTok{optim.method =} \StringTok{"SGDI"}\NormalTok{,}
        \DataTypeTok{max.iter =} \DecValTok{500}\NormalTok{, }\DataTypeTok{eps =} \FloatTok{10e-5}\NormalTok{)$steps ->}\StringTok{ }\NormalTok{SGDI}\FloatTok{.2}
\NormalTok{ind3 <-}\StringTok{ }\KeywordTok{sample}\NormalTok{(}\KeywordTok{length}\NormalTok{(y))}
\KeywordTok{logitGD}\NormalTok{(y[ind3], x[ind3], }\DataTypeTok{optim.method =} \StringTok{"SGDI"}\NormalTok{,}
        \DataTypeTok{max.iter =} \DecValTok{500}\NormalTok{, }\DataTypeTok{eps =} \FloatTok{10e-5}\NormalTok{)$steps ->}\StringTok{ }\NormalTok{SGDI}\FloatTok{.3}
\NormalTok{ind4 <-}\StringTok{ }\KeywordTok{sample}\NormalTok{(}\KeywordTok{length}\NormalTok{(y))}
\KeywordTok{logitGD}\NormalTok{(y[ind4], x[ind4], }\DataTypeTok{optim.method =} \StringTok{"SGDI"}\NormalTok{,}
        \DataTypeTok{max.iter =} \DecValTok{500}\NormalTok{, }\DataTypeTok{eps =} \FloatTok{10e-5}\NormalTok{)$steps ->}\StringTok{ }\NormalTok{SGDI}\FloatTok{.4}
\NormalTok{ind5 <-}\StringTok{ }\KeywordTok{sample}\NormalTok{(}\KeywordTok{length}\NormalTok{(y))}
\KeywordTok{logitGD}\NormalTok{(y[ind5], x[ind5], }\DataTypeTok{optim.method =} \StringTok{"SGDI"}\NormalTok{,}
        \DataTypeTok{max.iter =} \DecValTok{500}\NormalTok{, }\DataTypeTok{eps =} \FloatTok{10e-5}\NormalTok{)$steps ->}\StringTok{ }\NormalTok{SGDI}\FloatTok{.5}

\KeywordTok{do.call}\NormalTok{(rbind, }\KeywordTok{c}\NormalTok{(GDI, GDII, SGDI}\FloatTok{.1}\NormalTok{, SGDI}\FloatTok{.2}\NormalTok{, SGDI}\FloatTok{.3}\NormalTok{, SGDI}\FloatTok{.4}\NormalTok{, SGDI}\FloatTok{.5}\NormalTok{)) ->}\StringTok{ }\NormalTok{coeffs}
\KeywordTok{unlist}\NormalTok{(}\KeywordTok{lapply}\NormalTok{(}\KeywordTok{list}\NormalTok{(GDI, GDII, SGDI}\FloatTok{.1}\NormalTok{, SGDI}\FloatTok{.2}\NormalTok{, SGDI}\FloatTok{.3}\NormalTok{, SGDI}\FloatTok{.4}\NormalTok{, SGDI}\FloatTok{.5}\NormalTok{), length)) ->}\StringTok{ }\NormalTok{algorithm}
\NormalTok{data2viz <-}\StringTok{ }\KeywordTok{cbind}\NormalTok{(}\KeywordTok{as.data.frame}\NormalTok{(coeffs),}
      \DataTypeTok{algorithm =} \KeywordTok{unlist}\NormalTok{(}\KeywordTok{mapply}\NormalTok{(rep, }\KeywordTok{c}\NormalTok{(}\StringTok{"GDI"}\NormalTok{, }\StringTok{"GDII"}\NormalTok{, }\StringTok{"SGDI.1"}\NormalTok{, }\StringTok{"SGDI.2"}\NormalTok{, }\StringTok{"SGDI.3"}\NormalTok{, }\StringTok{"SGDI.4"}\NormalTok{, }\StringTok{"SGDI.5"}\NormalTok{), algorithm)))}
\KeywordTok{names}\NormalTok{(data2viz)[}\DecValTok{1}\NormalTok{:}\DecValTok{2}\NormalTok{] <-}\StringTok{ }\KeywordTok{c}\NormalTok{(}\StringTok{"Intercept"}\NormalTok{, }\StringTok{"X"}\NormalTok{)}
\KeywordTok{library}\NormalTok{(ggplot2); }\KeywordTok{library}\NormalTok{(ggthemes)}
\KeywordTok{ggplot}\NormalTok{(data2viz) +}
\StringTok{  }\KeywordTok{geom_point}\NormalTok{(}\KeywordTok{aes}\NormalTok{(}\DataTypeTok{x =} \NormalTok{X, }\DataTypeTok{y =} \NormalTok{Intercept, }\DataTypeTok{col =} \NormalTok{algorithm)) +}
\StringTok{  }\KeywordTok{geom_line}\NormalTok{(}\KeywordTok{aes}\NormalTok{(}\DataTypeTok{x =} \NormalTok{X, }\DataTypeTok{y =} \NormalTok{Intercept, }\DataTypeTok{col =} \NormalTok{algorithm,}
                \DataTypeTok{group =} \NormalTok{algorithm)) +}
\StringTok{  }\KeywordTok{theme_tufte}\NormalTok{(}\DataTypeTok{base_size =} \DecValTok{20}\NormalTok{)}
\end{Highlighting}
\end{Shaded}

\texttt{logitGD()} asda \texttt{graphSGD()}

\begin{Shaded}
\begin{Highlighting}[]
\KeywordTok{graphSGD}\NormalTok{(}\KeywordTok{c}\NormalTok{(}\DecValTok{0}\NormalTok{,}\DecValTok{0}\NormalTok{), y, x)}
\KeywordTok{graphSGD}\NormalTok{(}\KeywordTok{c}\NormalTok{(}\FloatTok{3.1}\NormalTok{,}\FloatTok{2.1}\NormalTok{), y, x)}
\KeywordTok{graphSGD}\NormalTok{(}\KeywordTok{c}\NormalTok{(}\DecValTok{4}\NormalTok{,}\DecValTok{3}\NormalTok{), y, x)}
\KeywordTok{graphSGD}\NormalTok{(}\KeywordTok{c}\NormalTok{(}\DecValTok{1}\NormalTok{,}\DecValTok{2}\NormalTok{), y, x)}
\end{Highlighting}
\end{Shaded}

\begin{Shaded}
\begin{Highlighting}[]
\NormalTok{dataCox <-}\StringTok{ }\NormalTok{function(N, lambda, rho, x, beta, censRate)\{}
  
  \CommentTok{# real Weibull times}
  \NormalTok{u <-}\StringTok{ }\KeywordTok{runif}\NormalTok{(N)}
  \NormalTok{Treal <-}\StringTok{ }\NormalTok{(-}\StringTok{ }\KeywordTok{log}\NormalTok{(u) /}\StringTok{ }\NormalTok{(lambda *}\StringTok{ }\KeywordTok{exp}\NormalTok{(x *}\StringTok{ }\NormalTok{beta)))^(}\DecValTok{1} \NormalTok{/}\StringTok{ }\NormalTok{rho)}
  
  \CommentTok{# censoring times}
  \NormalTok{Censoring <-}\StringTok{ }\KeywordTok{rexp}\NormalTok{(N, censRate)}
  
  \CommentTok{# follow-up times and event indicators}
  \NormalTok{time <-}\StringTok{ }\KeywordTok{pmin}\NormalTok{(Treal, Censoring)}
  \NormalTok{status <-}\StringTok{ }\KeywordTok{as.numeric}\NormalTok{(Treal <=}\StringTok{ }\NormalTok{Censoring)}
  
  \CommentTok{# data set}
  \KeywordTok{data.frame}\NormalTok{(}\DataTypeTok{id=}\DecValTok{1}\NormalTok{:N, }\DataTypeTok{time=}\NormalTok{time, }\DataTypeTok{status=}\NormalTok{status, }\DataTypeTok{x=}\NormalTok{x)}
\NormalTok{\}}

\NormalTok{x <-}\StringTok{ }\KeywordTok{sample}\NormalTok{(}\DecValTok{0}\NormalTok{:}\DecValTok{1}\NormalTok{, }\DataTypeTok{size =} \DecValTok{20}\NormalTok{, }\DataTypeTok{prob =}  \KeywordTok{rep}\NormalTok{(}\FloatTok{0.5}\NormalTok{,}\DecValTok{2}\NormalTok{), }\DataTypeTok{replace =} \OtherTok{TRUE}\NormalTok{)}

\KeywordTok{head}\NormalTok{(}\KeywordTok{dataCox}\NormalTok{(}\DecValTok{20}\NormalTok{, }\DecValTok{3}\NormalTok{, }\DecValTok{2}\NormalTok{, x, }\DecValTok{2}\NormalTok{, }\DecValTok{5}\NormalTok{))}
\end{Highlighting}
\end{Shaded}

\begin{verbatim}
  id       time status x
1  1 0.34275395      1 0
2  2 0.25465547      1 0
3  3 0.33424171      1 0
4  4 0.01337494      0 0
5  5 0.15110077      1 1
6  6 0.12959737      0 0
\end{verbatim}

\end{document}
