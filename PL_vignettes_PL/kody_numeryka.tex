\documentclass[]{article}
\usepackage{lmodern}
\usepackage{amssymb,amsmath}
\usepackage{ifxetex,ifluatex}
\usepackage{fixltx2e} % provides \textsubscript
\ifnum 0\ifxetex 1\fi\ifluatex 1\fi=0 % if pdftex
  \usepackage[T1]{fontenc}
  \usepackage[utf8]{inputenc}
\else % if luatex or xelatex
  \ifxetex
    \usepackage{mathspec}
    \usepackage{xltxtra,xunicode}
  \else
    \usepackage{fontspec}
  \fi
  \defaultfontfeatures{Mapping=tex-text,Scale=MatchLowercase}
  \newcommand{\euro}{€}
\fi
% use upquote if available, for straight quotes in verbatim environments
\IfFileExists{upquote.sty}{\usepackage{upquote}}{}
% use microtype if available
\IfFileExists{microtype.sty}{%
\usepackage{microtype}
\UseMicrotypeSet[protrusion]{basicmath} % disable protrusion for tt fonts
}{}
\usepackage[margin=1in]{geometry}
\usepackage{color}
\usepackage{fancyvrb}
\newcommand{\VerbBar}{|}
\newcommand{\VERB}{\Verb[commandchars=\\\{\}]}
\DefineVerbatimEnvironment{Highlighting}{Verbatim}{commandchars=\\\{\}}
% Add ',fontsize=\small' for more characters per line
\usepackage{framed}
\definecolor{shadecolor}{RGB}{248,248,248}
\newenvironment{Shaded}{\begin{snugshade}}{\end{snugshade}}
\newcommand{\KeywordTok}[1]{\textcolor[rgb]{0.13,0.29,0.53}{\textbf{{#1}}}}
\newcommand{\DataTypeTok}[1]{\textcolor[rgb]{0.13,0.29,0.53}{{#1}}}
\newcommand{\DecValTok}[1]{\textcolor[rgb]{0.00,0.00,0.81}{{#1}}}
\newcommand{\BaseNTok}[1]{\textcolor[rgb]{0.00,0.00,0.81}{{#1}}}
\newcommand{\FloatTok}[1]{\textcolor[rgb]{0.00,0.00,0.81}{{#1}}}
\newcommand{\CharTok}[1]{\textcolor[rgb]{0.31,0.60,0.02}{{#1}}}
\newcommand{\StringTok}[1]{\textcolor[rgb]{0.31,0.60,0.02}{{#1}}}
\newcommand{\CommentTok}[1]{\textcolor[rgb]{0.56,0.35,0.01}{\textit{{#1}}}}
\newcommand{\OtherTok}[1]{\textcolor[rgb]{0.56,0.35,0.01}{{#1}}}
\newcommand{\AlertTok}[1]{\textcolor[rgb]{0.94,0.16,0.16}{{#1}}}
\newcommand{\FunctionTok}[1]{\textcolor[rgb]{0.00,0.00,0.00}{{#1}}}
\newcommand{\RegionMarkerTok}[1]{{#1}}
\newcommand{\ErrorTok}[1]{\textbf{{#1}}}
\newcommand{\NormalTok}[1]{{#1}}
\ifxetex
  \usepackage[setpagesize=false, % page size defined by xetex
              unicode=false, % unicode breaks when used with xetex
              xetex]{hyperref}
\else
  \usepackage[unicode=true]{hyperref}
\fi
\hypersetup{breaklinks=true,
            bookmarks=true,
            pdfauthor={Marcin Kosinski},
            pdftitle={Untitled},
            colorlinks=true,
            citecolor=blue,
            urlcolor=blue,
            linkcolor=magenta,
            pdfborder={0 0 0}}
\urlstyle{same}  % don't use monospace font for urls
\setlength{\parindent}{0pt}
\setlength{\parskip}{6pt plus 2pt minus 1pt}
\setlength{\emergencystretch}{3em}  % prevent overfull lines
\setcounter{secnumdepth}{0}

%%% Use protect on footnotes to avoid problems with footnotes in titles
\let\rmarkdownfootnote\footnote%
\def\footnote{\protect\rmarkdownfootnote}

%%% Change title format to be more compact
\usepackage{titling}

% Create subtitle command for use in maketitle
\newcommand{\subtitle}[1]{
  \posttitle{
    \begin{center}\large#1\end{center}
    }
}

\setlength{\droptitle}{-2em}
  \title{Untitled}
  \pretitle{\vspace{\droptitle}\centering\huge}
  \posttitle{\par}
  \author{Marcin Kosinski}
  \preauthor{\centering\large\emph}
  \postauthor{\par}
  \predate{\centering\large\emph}
  \postdate{\par}
  \date{25.10.2015}



\begin{document}

\maketitle


\begin{Shaded}
\begin{Highlighting}[]
\NormalTok{logitGD <-}\StringTok{ }\NormalTok{function(y, x, }\DataTypeTok{optim.method =} \StringTok{"GDI"}\NormalTok{, }\DataTypeTok{eps =} \FloatTok{10e-4}\NormalTok{,}
                    \DataTypeTok{max.iter =} \DecValTok{100}\NormalTok{, }\DataTypeTok{alpha =} \NormalTok{function(t)\{}\DecValTok{1}\NormalTok{/t\}, }\DataTypeTok{beta_0 =} \KeywordTok{c}\NormalTok{(}\DecValTok{0}\NormalTok{,}\DecValTok{0}\NormalTok{))\{}
  \KeywordTok{stopifnot}\NormalTok{(}\KeywordTok{length}\NormalTok{(y) ==}\StringTok{ }\KeywordTok{length}\NormalTok{(x) &}\StringTok{ }\NormalTok{optim.method %in%}\StringTok{ }\KeywordTok{c}\NormalTok{(}\StringTok{"GDI"}\NormalTok{, }\StringTok{"GDII"}\NormalTok{, }\StringTok{"SGDI"}\NormalTok{)}
            \NormalTok{&}\StringTok{ }\KeywordTok{is.numeric}\NormalTok{(}\KeywordTok{c}\NormalTok{(max.iter, eps, x)) &}\StringTok{ }\KeywordTok{all}\NormalTok{(}\KeywordTok{c}\NormalTok{(eps, max.iter) >}\StringTok{ }\DecValTok{0}\NormalTok{) &}
\StringTok{              }\KeywordTok{is.function}\NormalTok{(alpha))}
  \NormalTok{iter <-}\StringTok{ }\DecValTok{0}
  \NormalTok{err <-}\StringTok{ }\KeywordTok{list}\NormalTok{()}
  \NormalTok{err[[iter}\DecValTok{+1}\NormalTok{]] <-}\StringTok{ }\NormalTok{eps}\DecValTok{+1}
  \NormalTok{w_old <-}\StringTok{ }\NormalTok{beta_0}

  \NormalTok{res <-}\KeywordTok{list}\NormalTok{()}
  \NormalTok{while(iter <}\StringTok{ }\NormalTok{max.iter &&}\StringTok{ }\NormalTok{(}\KeywordTok{abs}\NormalTok{(err[[}\KeywordTok{ifelse}\NormalTok{(iter==}\DecValTok{0}\NormalTok{,}\DecValTok{1}\NormalTok{,iter)]]) >}\StringTok{ }\NormalTok{eps))\{}

    \NormalTok{iter <-}\StringTok{ }\NormalTok{iter +}\StringTok{ }\DecValTok{1}
    \NormalTok{if (optim.method ==}\StringTok{ "GDI"}\NormalTok{)\{}
      \NormalTok{w_new <-}\StringTok{ }\NormalTok{w_old +}\StringTok{ }\KeywordTok{alpha}\NormalTok{(iter)*}\KeywordTok{updateWeightsGDI}\NormalTok{(y, x, w_old)}
    \NormalTok{\}}
    \NormalTok{if (optim.method ==}\StringTok{ "GDII"}\NormalTok{)\{}
      \NormalTok{w_new <-}\StringTok{ }\NormalTok{w_old +}\StringTok{ }\KeywordTok{as.vector}\NormalTok{(}\KeywordTok{inverseHessianGDII}\NormalTok{(x, w_old)%*%}
\StringTok{                                   }\KeywordTok{updateWeightsGDI}\NormalTok{(y, x, w_old))}
    \NormalTok{\}}
    \NormalTok{if (optim.method ==}\StringTok{ "SGDI"}\NormalTok{)\{}
      \NormalTok{w_new <-}\StringTok{ }\NormalTok{w_old +}\StringTok{ }\KeywordTok{alpha}\NormalTok{(iter)*}\KeywordTok{updateWeightsSGDI}\NormalTok{(y[iter], x[iter], w_old)}
    \NormalTok{\}}
    \NormalTok{res[[iter]] <-}\StringTok{ }\NormalTok{w_new}
    \NormalTok{err[[iter]] <-}\StringTok{ }\KeywordTok{sqrt}\NormalTok{(}\KeywordTok{sum}\NormalTok{((w_new -}\StringTok{ }\NormalTok{w_old)^}\DecValTok{2}\NormalTok{))}

    \NormalTok{w_old <-}\StringTok{ }\NormalTok{w_new}

  \NormalTok{\}}
  \KeywordTok{return}\NormalTok{(}\KeywordTok{list}\NormalTok{(}\DataTypeTok{steps =} \KeywordTok{c}\NormalTok{(}\KeywordTok{list}\NormalTok{(beta_0),res), }\DataTypeTok{errors =} \KeywordTok{c}\NormalTok{(}\KeywordTok{list}\NormalTok{(}\KeywordTok{c}\NormalTok{(}\DecValTok{0}\NormalTok{,}\DecValTok{0}\NormalTok{)),err)))}
\NormalTok{\}}

\NormalTok{updateWeightsGDI <-}\StringTok{ }\NormalTok{function(y, x, w_old)\{}
  \NormalTok{(}\DecValTok{1}\NormalTok{/}\KeywordTok{length}\NormalTok{(y))*}\KeywordTok{c}\NormalTok{(}\KeywordTok{sum}\NormalTok{(y-}\KeywordTok{p}\NormalTok{(w_old, x)), }\KeywordTok{sum}\NormalTok{(x*(y-}\KeywordTok{p}\NormalTok{(w_old, x))))}
  \CommentTok{#c(sum(y-p(w_old, x)), sum(x*(y-p(w_old, x))))}
\NormalTok{\}}

\NormalTok{updateWeightsSGDI <-}\StringTok{ }\NormalTok{function(y_i, x_i, w_old)\{}
  \KeywordTok{c}\NormalTok{(y_i-}\KeywordTok{p}\NormalTok{(w_old, x_i), x_i*(y_i-}\KeywordTok{p}\NormalTok{(w_old, x_i)))}
\NormalTok{\}}

\NormalTok{p <-}\StringTok{ }\NormalTok{function(w_old, x_i)\{}
  \DecValTok{1}\NormalTok{/(}\DecValTok{1}\NormalTok{+}\KeywordTok{exp}\NormalTok{(-w_old[}\DecValTok{1}\NormalTok{]-w_old[}\DecValTok{2}\NormalTok{]*x_i))}
\NormalTok{\}}

\NormalTok{inverseHessianGDII <-}\StringTok{ }\NormalTok{function(x, w_old)\{}
  \KeywordTok{solve}\NormalTok{(}
    \KeywordTok{matrix}\NormalTok{(}\KeywordTok{c}\NormalTok{(}
      \KeywordTok{sum}\NormalTok{(}\KeywordTok{p}\NormalTok{(w_old, x)*(}\DecValTok{1}\NormalTok{-}\KeywordTok{p}\NormalTok{(w_old, x))),}
      \KeywordTok{sum}\NormalTok{(x*}\KeywordTok{p}\NormalTok{(w_old, x)*(}\DecValTok{1}\NormalTok{-}\KeywordTok{p}\NormalTok{(w_old, x))),}
      \KeywordTok{sum}\NormalTok{(x*}\KeywordTok{p}\NormalTok{(w_old, x)*(}\DecValTok{1}\NormalTok{-}\KeywordTok{p}\NormalTok{(w_old, x))),}
      \KeywordTok{sum}\NormalTok{(x*x*}\KeywordTok{p}\NormalTok{(w_old, x)*(}\DecValTok{1}\NormalTok{-}\KeywordTok{p}\NormalTok{(w_old, x)))}
    \NormalTok{),}
    \DataTypeTok{nrow =}\DecValTok{2} \NormalTok{)}
  \NormalTok{)}
\NormalTok{\}}
\end{Highlighting}
\end{Shaded}

\begin{Shaded}
\begin{Highlighting}[]
                    \CommentTok{# wstępna inicjalizacja parametrów}
\NormalTok{eps =}\StringTok{ }\FloatTok{1e-5}                               \CommentTok{# warunek stopu.}

\NormalTok{n =}\StringTok{ }\KeywordTok{length}\NormalTok{(data)                         }\CommentTok{# data jest listą ramek danych.}

\NormalTok{diff =}\StringTok{ }\NormalTok{eps +}\StringTok{ }\DecValTok{1}                           \CommentTok{# różnice w oszacowaniach parametrów}
                                         \CommentTok{# między kolejnymi krokami.}

\NormalTok{learningRates =}\StringTok{ }\NormalTok{function(x) }\DecValTok{1}\NormalTok{/x          }\CommentTok{# długości kroku algorytmu.}

\NormalTok{beta_old =}\StringTok{ }\KeywordTok{numeric}\NormalTok{(}\DecValTok{0}\NormalTok{, }\DataTypeTok{length =} \NormalTok{k)        }\CommentTok{# punkt startowy dlugosci k,}
                                         \CommentTok{# gdzie k to liczba zmiennych}
                                         \CommentTok{# objaśniających w modelu.}

\NormalTok{max.iter =}\StringTok{ }\DecValTok{500}                           \CommentTok{# maksymalna liczba kroków.}
\NormalTok{ź}
                              \CommentTok{# estymacja}
\NormalTok{i =}\StringTok{ }\DecValTok{1}                                    \CommentTok{# iterator kroku algorytmu.}
\NormalTok{while(i <=}\StringTok{ }\NormalTok{max.iter |}\StringTok{ }\NormalTok{diff <}\StringTok{ }\NormalTok{eps) do     }
  \NormalTok{iter =}\StringTok{ }\KeywordTok{ifelse}\NormalTok{(i mod n ==}\StringTok{ }\DecValTok{0}\NormalTok{, n, i mod n)}\CommentTok{# wybierz kolejny podzbiór batch.}
  \NormalTok{batch =}\StringTok{ }\NormalTok{data[[iter]] }
  \NormalTok{beta_new =}\StringTok{ }\NormalTok{beta_old -}\StringTok{ }\KeywordTok{learningRates}\NormalTok{(i) *}\StringTok{ }\KeywordTok{U_Batch}\NormalTok{(batch) }
                                         \CommentTok{# U_Batch to częściowa funkcja}
                                         \CommentTok{# log-wiarogdności dla zaobserwowanego}
                                         \CommentTok{# zbioru `batch`}
  \NormalTok{diff =}\StringTok{ }\KeywordTok{euclidean_dist}\NormalTok{(beta_new, beta_old) }\CommentTok{# odległość euklidesowa}
  \NormalTok{beta_old =}\StringTok{ }\NormalTok{beta_new }
  \NormalTok{i =}\StringTok{ }\NormalTok{i +}\StringTok{ }\DecValTok{1}
\NormalTok{end while}
\NormalTok{return beta_new}
\end{Highlighting}
\end{Shaded}

\begin{Shaded}
\begin{Highlighting}[]
\NormalTok{coxphSGD <-}\StringTok{ }\NormalTok{function(formula, data, }\DataTypeTok{learningRates =} \NormalTok{function(x)\{}\DecValTok{1}\NormalTok{/x\},}
                    \DataTypeTok{beta_0 =} \DecValTok{0}\NormalTok{, }\DataTypeTok{epsilon =} \FloatTok{1e-5}\NormalTok{, }\DataTypeTok{max.iter =} \DecValTok{500} \NormalTok{) \{}
  \KeywordTok{checkArguments}\NormalTok{(formula, data, learningRates,}
                  \NormalTok{beta_0, epsilon) ->}\StringTok{ }\NormalTok{beta_start }\CommentTok{# check arguments}
  \NormalTok{n <-}\StringTok{ }\KeywordTok{length}\NormalTok{(data)}
  \NormalTok{diff <-}\StringTok{ }\NormalTok{epsilon +}\StringTok{ }\DecValTok{1}
  \NormalTok{i <-}\StringTok{ }\DecValTok{1}
  \NormalTok{beta_new <-}\StringTok{ }\KeywordTok{list}\NormalTok{()     }\CommentTok{# steps are saved in a list so that they can}
  \NormalTok{beta_old <-}\StringTok{ }\NormalTok{beta_start }\CommentTok{# be tracked in the future}
  \CommentTok{# estimate}
  \NormalTok{while(i <=}\StringTok{ }\NormalTok{max.iter &}\StringTok{ }\NormalTok{diff >}\StringTok{ }\NormalTok{epsilon) \{}
    \NormalTok{beta_new[[i]] <-}\StringTok{ }\KeywordTok{coxphSGD_batch}\NormalTok{(}\DataTypeTok{formula =} \NormalTok{formula, }\DataTypeTok{beta =} \NormalTok{beta_old,}
        \DataTypeTok{learningRate =} \KeywordTok{learningRates}\NormalTok{(i), }\DataTypeTok{data =} \NormalTok{data[[}\KeywordTok{ifelse}\NormalTok{(i%%n==}\DecValTok{0}\NormalTok{,n,i%%n)]])}
    
    \NormalTok{diff <-}\StringTok{ }\KeywordTok{sqrt}\NormalTok{(}\KeywordTok{sum}\NormalTok{((beta_new[[i]] -}\StringTok{ }\NormalTok{beta_old)^}\DecValTok{2}\NormalTok{))}
    \NormalTok{beta_old <-}\StringTok{ }\NormalTok{beta_new[[i]]}
    \NormalTok{i <-}\StringTok{ }\NormalTok{i +}\StringTok{ }\DecValTok{1}  
  \NormalTok{\}}
  \CommentTok{# return results}
  \KeywordTok{list}\NormalTok{(}\DataTypeTok{Call =} \KeywordTok{match.call}\NormalTok{(), }\DataTypeTok{epsilon =} \NormalTok{epsilon, }\DataTypeTok{learningRates =} \NormalTok{learningRates,}
       \DataTypeTok{steps =} \NormalTok{i, }\DataTypeTok{coefficients =} \KeywordTok{c}\NormalTok{(}\KeywordTok{list}\NormalTok{(beta_start), beta_new))}
\NormalTok{\}}

\NormalTok{coxphSGD_batch <-}\StringTok{ }\NormalTok{function(formula, data, learningRate, beta)\{}
  \CommentTok{# collect times, status, variables and reorder samples }
  \CommentTok{# to make the algorithm more clear to read and track}
  \NormalTok{batchData <-}\StringTok{ }\KeywordTok{prepareBatch}\NormalTok{(}\DataTypeTok{formula =} \NormalTok{formula, }\DataTypeTok{data =} \NormalTok{data)}
  
  \CommentTok{# calculate the log-likelihood for this batch sample}
  \NormalTok{partial_sum <-}\StringTok{ }\KeywordTok{list}\NormalTok{()}
  
  \NormalTok{for(k in }\DecValTok{1}\NormalTok{:}\KeywordTok{nrow}\NormalTok{(batchData)) \{}
    \CommentTok{# risk set for current time/observation}
    \NormalTok{risk_set <-}\StringTok{ }\NormalTok{batchData %>%}\StringTok{ }\KeywordTok{filter}\NormalTok{(times >=}\StringTok{ }\NormalTok{batchData$times[k])}
    
    \NormalTok{nominator <-}\StringTok{ }\KeywordTok{apply}\NormalTok{(risk_set[, -}\KeywordTok{c}\NormalTok{(}\DecValTok{1}\NormalTok{,}\DecValTok{2}\NormalTok{)], }\DataTypeTok{MARGIN =} \DecValTok{1}\NormalTok{, function(element)\{}
      \NormalTok{element *}\StringTok{ }\KeywordTok{exp}\NormalTok{(element *}\StringTok{ }\NormalTok{beta)}
    \NormalTok{\}) %>%}\StringTok{ }\KeywordTok{rowSums}\NormalTok{()}
      
    \NormalTok{denominator <-}\StringTok{ }\KeywordTok{apply}\NormalTok{(risk_set[, -}\KeywordTok{c}\NormalTok{(}\DecValTok{1}\NormalTok{,}\DecValTok{2}\NormalTok{)], }\DataTypeTok{MARGIN =} \DecValTok{1}\NormalTok{, function(element)\{}
      \KeywordTok{exp}\NormalTok{(element *}\StringTok{ }\NormalTok{beta)}
    \NormalTok{\}) %>%}\StringTok{ }\KeywordTok{rowSums}\NormalTok{()}
      
    \NormalTok{partial_sum[[k]] <-}\StringTok{ }
\StringTok{      }\NormalTok{batchData[k, }\StringTok{"event"}\NormalTok{] *}\StringTok{ }\NormalTok{(batchData[k, -}\KeywordTok{c}\NormalTok{(}\DecValTok{1}\NormalTok{,}\DecValTok{2}\NormalTok{)] -}\StringTok{ }\NormalTok{nominator/denominator)}
  \NormalTok{\}}
  \KeywordTok{do.call}\NormalTok{(rbind, partial_sum) %>%}
\StringTok{    }\KeywordTok{colSums}\NormalTok{() ->}\StringTok{ }\NormalTok{U_batch}
  
  \KeywordTok{return}\NormalTok{(beta +}\StringTok{ }\NormalTok{learningRate *}\StringTok{ }\NormalTok{U_batch)}
\NormalTok{\}}
  
\NormalTok{checkArguments <-}\StringTok{ }\NormalTok{function(formula, data, learningRates,}
                             \NormalTok{beta_0, epsilon) \{}
  \KeywordTok{assert_that}\NormalTok{(}\KeywordTok{is.list}\NormalTok{(data) &}\StringTok{ }\KeywordTok{length}\NormalTok{(data) >}\StringTok{ }\DecValTok{0}\NormalTok{)}
  \KeywordTok{assert_that}\NormalTok{(}\KeywordTok{length}\NormalTok{(}\KeywordTok{unique}\NormalTok{(}\KeywordTok{unlist}\NormalTok{(}\KeywordTok{lapply}\NormalTok{(data, ncol)))) ==}\StringTok{ }\DecValTok{1}\NormalTok{)}
  \CommentTok{# + check names and types for every variables}
  \KeywordTok{assert_that}\NormalTok{(}\KeywordTok{is.function}\NormalTok{(learningRates))}
  \KeywordTok{assert_that}\NormalTok{(}\KeywordTok{is.numeric}\NormalTok{(epsilon))}
  \KeywordTok{assert_that}\NormalTok{(}\KeywordTok{is.numeric}\NormalTok{(beta_0))}
  
    \CommentTok{# check length of the start parameter}
  \NormalTok{if (}\KeywordTok{length}\NormalTok{(beta_0) ==}\StringTok{ }\DecValTok{1}\NormalTok{) \{}
    \NormalTok{beta_0 <-}\StringTok{ }\KeywordTok{rep}\NormalTok{(beta_0, }\KeywordTok{as.character}\NormalTok{(formula)[}\DecValTok{3}\NormalTok{] %>%}
\StringTok{                    }\KeywordTok{strsplit}\NormalTok{(}\StringTok{"}\CharTok{\textbackslash{}\textbackslash{}}\StringTok{+"}\NormalTok{) %>%}
\StringTok{                    }\NormalTok{unlist %>%}
\StringTok{                    }\NormalTok{length)}
  \NormalTok{\}}
  \KeywordTok{return}\NormalTok{(beta_0)}
\NormalTok{\}}
\end{Highlighting}
\end{Shaded}

\begin{Shaded}
\begin{Highlighting}[]
\NormalTok{x <-}\StringTok{ }\KeywordTok{runif}\NormalTok{(}\DecValTok{1000}\NormalTok{)}
\NormalTok{z <-}\StringTok{ }\DecValTok{2} \NormalTok{+}\StringTok{ }\DecValTok{3}\NormalTok{*x}
\NormalTok{pr <-}\StringTok{ }\DecValTok{1}\NormalTok{/(}\DecValTok{1}\NormalTok{+}\KeywordTok{exp}\NormalTok{(-z))}
\NormalTok{y <-}\StringTok{ }\KeywordTok{rbinom}\NormalTok{(}\DecValTok{1000}\NormalTok{,}\DecValTok{1}\NormalTok{,pr)}


\KeywordTok{logitGD}\NormalTok{(y, x, }\DataTypeTok{optim.method =} \StringTok{"GDI"}\NormalTok{, }\DataTypeTok{eps =} \FloatTok{10e-5}\NormalTok{, }\DataTypeTok{max.iter =} \DecValTok{500}\NormalTok{)$steps ->}\StringTok{ }\NormalTok{GDI}
\KeywordTok{logitGD}\NormalTok{(y, x, }\DataTypeTok{optim.method =} \StringTok{"GDII"}\NormalTok{, }\DataTypeTok{eps =} \FloatTok{10e-5}\NormalTok{, }\DataTypeTok{max.iter =} \DecValTok{500}\NormalTok{)$steps ->}\StringTok{ }\NormalTok{GDII}

\NormalTok{ind <-}\StringTok{ }\KeywordTok{sample}\NormalTok{(}\KeywordTok{length}\NormalTok{(y))}
\KeywordTok{logitGD}\NormalTok{(y[ind], x[ind], }\DataTypeTok{optim.method =} \StringTok{"SGDI"}\NormalTok{,}
        \DataTypeTok{max.iter =} \DecValTok{500}\NormalTok{, }\DataTypeTok{eps =} \FloatTok{10e-5}\NormalTok{)$steps ->}\StringTok{ }\NormalTok{SGDI}\FloatTok{.1}
\NormalTok{ind2 <-}\StringTok{ }\KeywordTok{sample}\NormalTok{(}\KeywordTok{length}\NormalTok{(y))}
\KeywordTok{logitGD}\NormalTok{(y[ind2], x[ind2], }\DataTypeTok{optim.method =} \StringTok{"SGDI"}\NormalTok{,}
        \DataTypeTok{max.iter =} \DecValTok{500}\NormalTok{, }\DataTypeTok{eps =} \FloatTok{10e-5}\NormalTok{)$steps ->}\StringTok{ }\NormalTok{SGDI}\FloatTok{.2}
\NormalTok{ind3 <-}\StringTok{ }\KeywordTok{sample}\NormalTok{(}\KeywordTok{length}\NormalTok{(y))}
\KeywordTok{logitGD}\NormalTok{(y[ind3], x[ind3], }\DataTypeTok{optim.method =} \StringTok{"SGDI"}\NormalTok{,}
        \DataTypeTok{max.iter =} \DecValTok{500}\NormalTok{, }\DataTypeTok{eps =} \FloatTok{10e-5}\NormalTok{)$steps ->}\StringTok{ }\NormalTok{SGDI}\FloatTok{.3}
\NormalTok{ind4 <-}\StringTok{ }\KeywordTok{sample}\NormalTok{(}\KeywordTok{length}\NormalTok{(y))}
\KeywordTok{logitGD}\NormalTok{(y[ind4], x[ind4], }\DataTypeTok{optim.method =} \StringTok{"SGDI"}\NormalTok{,}
        \DataTypeTok{max.iter =} \DecValTok{500}\NormalTok{, }\DataTypeTok{eps =} \FloatTok{10e-5}\NormalTok{)$steps ->}\StringTok{ }\NormalTok{SGDI}\FloatTok{.4}
\NormalTok{ind5 <-}\StringTok{ }\KeywordTok{sample}\NormalTok{(}\KeywordTok{length}\NormalTok{(y))}
\KeywordTok{logitGD}\NormalTok{(y[ind5], x[ind5], }\DataTypeTok{optim.method =} \StringTok{"SGDI"}\NormalTok{,}
        \DataTypeTok{max.iter =} \DecValTok{500}\NormalTok{, }\DataTypeTok{eps =} \FloatTok{10e-5}\NormalTok{)$steps ->}\StringTok{ }\NormalTok{SGDI}\FloatTok{.5}

\KeywordTok{do.call}\NormalTok{(rbind, }\KeywordTok{c}\NormalTok{(GDI, GDII, SGDI}\FloatTok{.1}\NormalTok{, SGDI}\FloatTok{.2}\NormalTok{, SGDI}\FloatTok{.3}\NormalTok{, SGDI}\FloatTok{.4}\NormalTok{, SGDI}\FloatTok{.5}\NormalTok{)) ->}\StringTok{ }\NormalTok{coeffs}
\KeywordTok{unlist}\NormalTok{(}\KeywordTok{lapply}\NormalTok{(}\KeywordTok{list}\NormalTok{(GDI, GDII, SGDI}\FloatTok{.1}\NormalTok{, SGDI}\FloatTok{.2}\NormalTok{, SGDI}\FloatTok{.3}\NormalTok{, SGDI}\FloatTok{.4}\NormalTok{, SGDI}\FloatTok{.5}\NormalTok{), length)) ->}\StringTok{ }\NormalTok{algorithm}
\NormalTok{data2viz <-}\StringTok{ }\KeywordTok{cbind}\NormalTok{(}\KeywordTok{as.data.frame}\NormalTok{(coeffs),}
      \DataTypeTok{algorithm =} \KeywordTok{unlist}\NormalTok{(}\KeywordTok{mapply}\NormalTok{(rep, }\KeywordTok{c}\NormalTok{(}\StringTok{"GDI"}\NormalTok{, }\StringTok{"GDII"}\NormalTok{, }\StringTok{"SGDI.1"}\NormalTok{, }\StringTok{"SGDI.2"}\NormalTok{, }\StringTok{"SGDI.3"}\NormalTok{, }\StringTok{"SGDI.4"}\NormalTok{, }\StringTok{"SGDI.5"}\NormalTok{), algorithm)))}
\KeywordTok{names}\NormalTok{(data2viz)[}\DecValTok{1}\NormalTok{:}\DecValTok{2}\NormalTok{] <-}\StringTok{ }\KeywordTok{c}\NormalTok{(}\StringTok{"Intercept"}\NormalTok{, }\StringTok{"X"}\NormalTok{)}
\KeywordTok{library}\NormalTok{(ggplot2); }\KeywordTok{library}\NormalTok{(ggthemes)}
\KeywordTok{ggplot}\NormalTok{(data2viz) +}
\StringTok{  }\KeywordTok{geom_point}\NormalTok{(}\KeywordTok{aes}\NormalTok{(}\DataTypeTok{x =} \NormalTok{X, }\DataTypeTok{y =} \NormalTok{Intercept, }\DataTypeTok{col =} \NormalTok{algorithm)) +}
\StringTok{  }\KeywordTok{geom_line}\NormalTok{(}\KeywordTok{aes}\NormalTok{(}\DataTypeTok{x =} \NormalTok{X, }\DataTypeTok{y =} \NormalTok{Intercept, }\DataTypeTok{col =} \NormalTok{algorithm,}
                \DataTypeTok{group =} \NormalTok{algorithm)) +}
\StringTok{  }\KeywordTok{theme_tufte}\NormalTok{(}\DataTypeTok{base_size =} \DecValTok{20}\NormalTok{)}
\end{Highlighting}
\end{Shaded}

\texttt{logitGD()} asda \texttt{graphSGD()}

\begin{Shaded}
\begin{Highlighting}[]
\KeywordTok{graphSGD}\NormalTok{(}\KeywordTok{c}\NormalTok{(}\DecValTok{0}\NormalTok{,}\DecValTok{0}\NormalTok{), y, x)}
\KeywordTok{graphSGD}\NormalTok{(}\KeywordTok{c}\NormalTok{(}\FloatTok{3.1}\NormalTok{,}\FloatTok{2.1}\NormalTok{), y, x)}
\KeywordTok{graphSGD}\NormalTok{(}\KeywordTok{c}\NormalTok{(}\DecValTok{4}\NormalTok{,}\DecValTok{3}\NormalTok{), y, x)}
\KeywordTok{graphSGD}\NormalTok{(}\KeywordTok{c}\NormalTok{(}\DecValTok{1}\NormalTok{,}\DecValTok{2}\NormalTok{), y, x)}
\end{Highlighting}
\end{Shaded}

\begin{Shaded}
\begin{Highlighting}[]
\NormalTok{dataCox <-}\StringTok{ }\NormalTok{function(N, lambda, rho, x, beta, censRate)\{}
  
  \CommentTok{# real Weibull times}
  \NormalTok{u <-}\StringTok{ }\KeywordTok{runif}\NormalTok{(N)}
  \NormalTok{Treal <-}\StringTok{ }\NormalTok{(-}\StringTok{ }\KeywordTok{log}\NormalTok{(u) /}\StringTok{ }\NormalTok{(lambda *}\StringTok{ }\KeywordTok{exp}\NormalTok{(x %*%}\StringTok{ }\NormalTok{beta)))^(}\DecValTok{1} \NormalTok{/}\StringTok{ }\NormalTok{rho)}
  
  \CommentTok{# censoring times}
  \NormalTok{Censoring <-}\StringTok{ }\KeywordTok{rexp}\NormalTok{(N, censRate)}
  
  \CommentTok{# follow-up times and event indicators}
  \NormalTok{time <-}\StringTok{ }\KeywordTok{pmin}\NormalTok{(Treal, Censoring)}
  \NormalTok{status <-}\StringTok{ }\KeywordTok{as.numeric}\NormalTok{(Treal <=}\StringTok{ }\NormalTok{Censoring)}
  
  \CommentTok{# data set}
  \KeywordTok{data.frame}\NormalTok{(}\DataTypeTok{id=}\DecValTok{1}\NormalTok{:N, }\DataTypeTok{time=}\NormalTok{time, }\DataTypeTok{status=}\NormalTok{status, }\DataTypeTok{x=}\NormalTok{x)}
\NormalTok{\}}

\NormalTok{x <-}\StringTok{ }\KeywordTok{matrix}\NormalTok{(}\KeywordTok{sample}\NormalTok{(}\DecValTok{0}\NormalTok{:}\DecValTok{1}\NormalTok{, }\DataTypeTok{size =} \DecValTok{40}\NormalTok{, }\DataTypeTok{replace =} \OtherTok{TRUE}\NormalTok{), }\DataTypeTok{ncol =} \DecValTok{2}\NormalTok{)}

\KeywordTok{head}\NormalTok{(}\KeywordTok{dataCox}\NormalTok{(}\DecValTok{20}\NormalTok{, }\DecValTok{3}\NormalTok{, }\DecValTok{2}\NormalTok{, x, }\DataTypeTok{beta =} \KeywordTok{c}\NormalTok{(}\DecValTok{2}\NormalTok{,}\DecValTok{3}\NormalTok{), }\DecValTok{5}\NormalTok{))}
\end{Highlighting}
\end{Shaded}

\begin{verbatim}
  id       time status x.1 x.2
1  1 0.07591466      1   0   1
2  2 0.21533677      1   0   1
3  3 0.09686408      0   1   0
4  4 0.01157519      0   1   1
5  5 0.07779189      0   0   1
6  6 0.11807808      0   1   0
\end{verbatim}

\begin{Shaded}
\begin{Highlighting}[]
\KeywordTok{graphSGD}\NormalTok{(}\KeywordTok{c}\NormalTok{(}\DecValTok{0}\NormalTok{,}\DecValTok{0}\NormalTok{), y, x, }\DecValTok{4561}\NormalTok{);}\KeywordTok{graphSGD}\NormalTok{(}\KeywordTok{c}\NormalTok{(}\DecValTok{0}\NormalTok{,}\DecValTok{0}\NormalTok{), y, x, }\DecValTok{456}\NormalTok{)}
\KeywordTok{graphSGD}\NormalTok{(}\KeywordTok{c}\NormalTok{(}\DecValTok{2}\NormalTok{,}\DecValTok{1}\NormalTok{), y, x, }\DecValTok{4561}\NormalTok{);}\KeywordTok{graphSGD}\NormalTok{(}\KeywordTok{c}\NormalTok{(}\DecValTok{2}\NormalTok{,}\DecValTok{1}\NormalTok{), y, x, }\DecValTok{456}\NormalTok{);}
\KeywordTok{graphSGD}\NormalTok{(}\KeywordTok{c}\NormalTok{(}\DecValTok{1}\NormalTok{,}\DecValTok{0}\NormalTok{), y, x, }\DecValTok{4561}\NormalTok{);}\KeywordTok{graphSGD}\NormalTok{(}\KeywordTok{c}\NormalTok{(}\DecValTok{1}\NormalTok{,}\DecValTok{0}\NormalTok{), y, x, }\DecValTok{456}\NormalTok{);}
\KeywordTok{graphSGD}\NormalTok{(}\KeywordTok{c}\NormalTok{(}\FloatTok{2.1}\NormalTok{,}\FloatTok{3.1}\NormalTok{), y, x, }\DecValTok{4561}\NormalTok{)}\KeywordTok{graphSGD}\NormalTok{(}\KeywordTok{c}\NormalTok{(}\FloatTok{2.1}\NormalTok{,}\FloatTok{3.1}\NormalTok{), y, x, }\DecValTok{456}\NormalTok{)}
\end{Highlighting}
\end{Shaded}

\begin{Shaded}
\begin{Highlighting}[]
\NormalTok{x <-}\StringTok{ }\KeywordTok{matrix}\NormalTok{(}\KeywordTok{sample}\NormalTok{(}\DecValTok{0}\NormalTok{:}\DecValTok{1}\NormalTok{, }\DataTypeTok{size =} \DecValTok{20000}\NormalTok{, }\DataTypeTok{replace =} \OtherTok{TRUE}\NormalTok{), }\DataTypeTok{ncol =} \DecValTok{2}\NormalTok{)}
\NormalTok{dCox <-}\StringTok{ }\KeywordTok{dataCox}\NormalTok{(}\DecValTok{10}\NormalTok{^}\DecValTok{4}\NormalTok{, }\DataTypeTok{lambda =} \DecValTok{3}\NormalTok{, }\DataTypeTok{rho =} \DecValTok{2}\NormalTok{, x, }\DataTypeTok{beta =} \KeywordTok{c}\NormalTok{(}\DecValTok{1}\NormalTok{,}\DecValTok{3}\NormalTok{), }\DataTypeTok{censRate =} \DecValTok{5}\NormalTok{) }
\KeywordTok{vizCoxSGD}\NormalTok{(dCox)}
\end{Highlighting}
\end{Shaded}

\newpage

\begin{Shaded}
\begin{Highlighting}[]
\NormalTok{coxphSGD <-}\StringTok{ }\NormalTok{function(formula, data, }\DataTypeTok{learningRates =} \NormalTok{function(x)\{}\DecValTok{1}\NormalTok{/x\},}
                    \DataTypeTok{beta_0 =} \DecValTok{0}\NormalTok{, }\DataTypeTok{epsilon =} \FloatTok{1e-5}\NormalTok{, }\DataTypeTok{max.iter =} \DecValTok{500} \NormalTok{) \{}
  \KeywordTok{checkArguments}\NormalTok{(formula, data, learningRates,}
                  \NormalTok{beta_0, epsilon) ->}\StringTok{ }\NormalTok{beta_start }\CommentTok{# check arguments}
  \NormalTok{n <-}\StringTok{ }\KeywordTok{length}\NormalTok{(data)}
  \NormalTok{diff <-}\StringTok{ }\NormalTok{epsilon +}\StringTok{ }\DecValTok{1}
  \NormalTok{i <-}\StringTok{ }\DecValTok{1}
  \NormalTok{beta_new <-}\StringTok{ }\KeywordTok{list}\NormalTok{()     }\CommentTok{# steps are saved in a list so that they can}
  \NormalTok{beta_old <-}\StringTok{ }\NormalTok{beta_start }\CommentTok{# be traced in the future}
  \CommentTok{# estimate}
  \NormalTok{while(i <=}\StringTok{ }\NormalTok{max.iter &}\StringTok{ }\NormalTok{diff >}\StringTok{ }\NormalTok{epsilon) \{}
    \NormalTok{beta_new[[i]] <-}\StringTok{ }\KeywordTok{coxphSGD_batch}\NormalTok{(}\DataTypeTok{formula =} \NormalTok{formula, }\DataTypeTok{beta =} \NormalTok{beta_old,}
        \DataTypeTok{learningRate =} \KeywordTok{learningRates}\NormalTok{(i), }\DataTypeTok{data =} \NormalTok{data[[}\KeywordTok{ifelse}\NormalTok{(i%%n==}\DecValTok{0}\NormalTok{,n,i%%n)]]) %>%}
\StringTok{      }\NormalTok{unlist}
    \NormalTok{diff <-}\StringTok{ }\KeywordTok{sqrt}\NormalTok{(}\KeywordTok{sum}\NormalTok{((beta_new[[i]] -}\StringTok{ }\NormalTok{beta_old)^}\DecValTok{2}\NormalTok{))}
    \NormalTok{beta_old <-}\StringTok{ }\NormalTok{beta_new[[i]]}
    \NormalTok{i <-}\StringTok{ }\NormalTok{i +}\StringTok{ }\DecValTok{1}  \NormalTok{; }\KeywordTok{cat}\NormalTok{(}\StringTok{"}\CharTok{\textbackslash{}r}\StringTok{ iteration: "}\NormalTok{, i, }\StringTok{"}\CharTok{\textbackslash{}r}\StringTok{"}\NormalTok{)}
  \NormalTok{\}  }\CommentTok{# return results}
  \KeywordTok{list}\NormalTok{(}\DataTypeTok{Call =} \KeywordTok{match.call}\NormalTok{(), }\DataTypeTok{epsilon =} \NormalTok{epsilon, }\DataTypeTok{learningRates =} \NormalTok{learningRates,}
       \DataTypeTok{steps =} \NormalTok{i, }\DataTypeTok{coefficients =} \KeywordTok{c}\NormalTok{(}\KeywordTok{list}\NormalTok{(beta_start), beta_new))}
\NormalTok{\}}

\NormalTok{coxphSGD_batch <-}\StringTok{ }\NormalTok{function(formula, data, learningRate, beta)\{}
  \CommentTok{# collect times, status, variables and reorder samples }
  \CommentTok{# to make the algorithm more clear to read and track}
  \NormalTok{batchData <-}\StringTok{ }\KeywordTok{prepareBatch}\NormalTok{(}\DataTypeTok{formula =} \NormalTok{formula, }\DataTypeTok{data =} \NormalTok{data)}
  \CommentTok{# calculate the log-likelihood for this batch sample}
  \NormalTok{partial_sum <-}\StringTok{ }\KeywordTok{list}\NormalTok{()}
  \KeywordTok{foreach}\NormalTok{(}\DataTypeTok{k =} \DecValTok{1}\NormalTok{:}\KeywordTok{nrow}\NormalTok{(batchData)) %do%}\StringTok{ }\NormalTok{\{}
    \CommentTok{# risk set for current time/observation}
    \NormalTok{risk_set <-}\StringTok{ }\NormalTok{batchData %>%}\StringTok{ }\KeywordTok{filter}\NormalTok{(times >=}\StringTok{ }\NormalTok{batchData$times[k])}
    
    \NormalTok{nominator <-}\StringTok{ }\KeywordTok{apply}\NormalTok{(risk_set[, -}\KeywordTok{c}\NormalTok{(}\DecValTok{1}\NormalTok{,}\DecValTok{2}\NormalTok{)], }\DataTypeTok{MARGIN =} \DecValTok{1}\NormalTok{, function(element)\{}
      \NormalTok{element *}\StringTok{ }\KeywordTok{exp}\NormalTok{(element *}\StringTok{ }\NormalTok{beta)}
    \NormalTok{\}) %>%}\StringTok{ }\KeywordTok{rowSums}\NormalTok{()}
      
    \NormalTok{denominator <-}\StringTok{ }\KeywordTok{apply}\NormalTok{(risk_set[, -}\KeywordTok{c}\NormalTok{(}\DecValTok{1}\NormalTok{,}\DecValTok{2}\NormalTok{)], }\DataTypeTok{MARGIN =} \DecValTok{1}\NormalTok{, function(element)\{}
      \KeywordTok{exp}\NormalTok{(element *}\StringTok{ }\NormalTok{beta)}
    \NormalTok{\}) %>%}\StringTok{ }\KeywordTok{rowSums}\NormalTok{()}
      
    \NormalTok{partial_sum[[k]] <-}\StringTok{ }
\StringTok{      }\NormalTok{batchData[k, }\StringTok{"event"}\NormalTok{] *}\StringTok{ }\NormalTok{(batchData[k, -}\KeywordTok{c}\NormalTok{(}\DecValTok{1}\NormalTok{,}\DecValTok{2}\NormalTok{)] -}\StringTok{ }\NormalTok{nominator/denominator)}
  \NormalTok{\}}
  \KeywordTok{do.call}\NormalTok{(rbind, partial_sum) %>%}
\StringTok{    }\KeywordTok{colSums}\NormalTok{() ->}\StringTok{ }\NormalTok{U_batch}
  
  \KeywordTok{return}\NormalTok{(beta +}\StringTok{ }\NormalTok{learningRate *}\StringTok{ }\NormalTok{U_batch)}
\NormalTok{\}}
\end{Highlighting}
\end{Shaded}

\begin{Shaded}
\begin{Highlighting}[]
\NormalTok{prepareBatch <-}\StringTok{ }\NormalTok{function(formula, data) \{}
  \CommentTok{# Parameter identification as in  `survival::coxph()`.}
  \NormalTok{Call <-}\StringTok{ }\KeywordTok{match.call}\NormalTok{()}
  \NormalTok{indx <-}\StringTok{ }\KeywordTok{match}\NormalTok{(}\KeywordTok{c}\NormalTok{(}\StringTok{"formula"}\NormalTok{, }\StringTok{"data"}\NormalTok{),}
                \KeywordTok{names}\NormalTok{(Call), }\DataTypeTok{nomatch =} \DecValTok{0}\NormalTok{)}
  \NormalTok{if (indx[}\DecValTok{1}\NormalTok{] ==}\StringTok{ }\DecValTok{0}\NormalTok{) }
      \KeywordTok{stop}\NormalTok{(}\StringTok{"A formula argument is required"}\NormalTok{)}
  \NormalTok{temp <-}\StringTok{ }\NormalTok{Call[}\KeywordTok{c}\NormalTok{(}\DecValTok{1}\NormalTok{, indx)]}
  \NormalTok{temp[[}\DecValTok{1}\NormalTok{]] <-}\StringTok{ }\KeywordTok{as.name}\NormalTok{(}\StringTok{"model.frame"}\NormalTok{)}
  
  \NormalTok{mf <-}\StringTok{ }\KeywordTok{eval}\NormalTok{(temp, }\KeywordTok{parent.frame}\NormalTok{())}
  \NormalTok{Y <-}\StringTok{ }\KeywordTok{model.extract}\NormalTok{(mf, }\StringTok{"response"}\NormalTok{)}
  
  \NormalTok{if (!}\KeywordTok{inherits}\NormalTok{(Y, }\StringTok{"Surv"}\NormalTok{)) }
      \KeywordTok{stop}\NormalTok{(}\StringTok{"Response must be a survival object"}\NormalTok{)}
  \NormalTok{type <-}\StringTok{ }\KeywordTok{attr}\NormalTok{(Y, }\StringTok{"type"}\NormalTok{)}
  
  \NormalTok{if (type !=}\StringTok{ "right"} \NormalTok{&&}\StringTok{ }\NormalTok{type !=}\StringTok{ "counting"}\NormalTok{) }
      \KeywordTok{stop}\NormalTok{(}\KeywordTok{paste}\NormalTok{(}\StringTok{"Cox model doesn't support }\CharTok{\textbackslash{}"}\StringTok{"}\NormalTok{, type, }\StringTok{"}\CharTok{\textbackslash{}"}\StringTok{ survival data"}\NormalTok{, }
          \DataTypeTok{sep =} \StringTok{""}\NormalTok{))}
  
  \CommentTok{# collect times, status, variables and reorder samples }
  \CommentTok{# to make the algorithm more clear to read and track}
  \KeywordTok{cbind}\NormalTok{(}\DataTypeTok{event =} \KeywordTok{unclass}\NormalTok{(Y)[,}\DecValTok{2}\NormalTok{], }\CommentTok{# 1 indicates event, 0 indicates cens}
        \DataTypeTok{times =} \KeywordTok{unclass}\NormalTok{(Y)[,}\DecValTok{1}\NormalTok{],}
        \NormalTok{mf[, -}\DecValTok{1}\NormalTok{]) %>%}
\StringTok{    }\KeywordTok{arrange}\NormalTok{(times) }
\NormalTok{\}}
\end{Highlighting}
\end{Shaded}

\begin{Shaded}
\begin{Highlighting}[]
\KeywordTok{simulateCoxSGD}\NormalTok{(dCox, }\DataTypeTok{learningRates =} \NormalTok{function(x)\{}\DecValTok{1}\NormalTok{/(}\DecValTok{100}\NormalTok{*}\KeywordTok{sqrt}\NormalTok{(x))\},}
               \DataTypeTok{max.iter =} \DecValTok{10}\NormalTok{, }\DataTypeTok{epsilon =} \FloatTok{1e-5}\NormalTok{, }\DataTypeTok{beta_0 =} \KeywordTok{c}\NormalTok{(}\DecValTok{2}\NormalTok{,}\DecValTok{2}\NormalTok{))}
\end{Highlighting}
\end{Shaded}

\begin{Shaded}
\begin{Highlighting}[]
\NormalTok{logitGD <-}\StringTok{ }\NormalTok{function(y, x, }\DataTypeTok{optim.method =} \StringTok{"GDI"}\NormalTok{, }\DataTypeTok{eps =} \FloatTok{10e-4}\NormalTok{,}
                    \DataTypeTok{max.iter =} \DecValTok{100}\NormalTok{, }\DataTypeTok{alpha =} \NormalTok{function(t)\{}\DecValTok{1}\NormalTok{/t\}, }\DataTypeTok{beta_0 =} \KeywordTok{c}\NormalTok{(}\DecValTok{0}\NormalTok{,}\DecValTok{0}\NormalTok{))\{}
  \KeywordTok{stopifnot}\NormalTok{(}\KeywordTok{length}\NormalTok{(y) ==}\StringTok{ }\KeywordTok{length}\NormalTok{(x) &}\StringTok{ }\NormalTok{optim.method %in%}\StringTok{ }\KeywordTok{c}\NormalTok{(}\StringTok{"GDI"}\NormalTok{, }\StringTok{"GDII"}\NormalTok{, }\StringTok{"SGDI"}\NormalTok{)}
            \NormalTok{&}\StringTok{ }\KeywordTok{is.numeric}\NormalTok{(}\KeywordTok{c}\NormalTok{(max.iter, eps, x)) &}\StringTok{ }\KeywordTok{all}\NormalTok{(}\KeywordTok{c}\NormalTok{(eps, max.iter) >}\StringTok{ }\DecValTok{0}\NormalTok{) &}
\StringTok{              }\KeywordTok{is.function}\NormalTok{(alpha))}
  \NormalTok{iter <-}\StringTok{ }\DecValTok{0}
  \NormalTok{err <-}\StringTok{ }\KeywordTok{list}\NormalTok{()}
  \NormalTok{err[[iter}\DecValTok{+1}\NormalTok{]] <-}\StringTok{ }\NormalTok{eps}\DecValTok{+1}
  \NormalTok{w_old <-}\StringTok{ }\NormalTok{beta_0}

  \NormalTok{res <-}\KeywordTok{list}\NormalTok{()}
  \NormalTok{while(iter <}\StringTok{ }\NormalTok{max.iter &&}\StringTok{ }\NormalTok{(}\KeywordTok{abs}\NormalTok{(err[[}\KeywordTok{ifelse}\NormalTok{(iter==}\DecValTok{0}\NormalTok{,}\DecValTok{1}\NormalTok{,iter)]]) >}\StringTok{ }\NormalTok{eps))\{}

    \NormalTok{iter <-}\StringTok{ }\NormalTok{iter +}\StringTok{ }\DecValTok{1}
    \NormalTok{if (optim.method ==}\StringTok{ "GDI"}\NormalTok{)\{}
      \NormalTok{w_new <-}\StringTok{ }\NormalTok{w_old +}\StringTok{ }\KeywordTok{alpha}\NormalTok{(iter)*}\KeywordTok{updateWeightsGDI}\NormalTok{(y, x, w_old)}
    \NormalTok{\}}
    \NormalTok{if (optim.method ==}\StringTok{ "GDII"}\NormalTok{)\{}
      \NormalTok{w_new <-}\StringTok{ }\NormalTok{w_old +}\StringTok{ }\KeywordTok{as.vector}\NormalTok{(}\KeywordTok{inverseHessianGDII}\NormalTok{(x, w_old)%*%}
\StringTok{                                   }\KeywordTok{updateWeightsGDI}\NormalTok{(y, x, w_old))}
    \NormalTok{\}}
    \NormalTok{if (optim.method ==}\StringTok{ "SGDI"}\NormalTok{)\{}
      \NormalTok{w_new <-}\StringTok{ }\NormalTok{w_old +}\StringTok{ }\KeywordTok{alpha}\NormalTok{(iter)*}\KeywordTok{updateWeightsSGDI}\NormalTok{(y[iter], x[iter], w_old)}
    \NormalTok{\}}
    \NormalTok{res[[iter]] <-}\StringTok{ }\NormalTok{w_new}
    \NormalTok{err[[iter]] <-}\StringTok{ }\KeywordTok{sqrt}\NormalTok{(}\KeywordTok{sum}\NormalTok{((w_new -}\StringTok{ }\NormalTok{w_old)^}\DecValTok{2}\NormalTok{))}

    \NormalTok{w_old <-}\StringTok{ }\NormalTok{w_new}

  \NormalTok{\}}
  \KeywordTok{return}\NormalTok{(}\KeywordTok{list}\NormalTok{(}\DataTypeTok{steps =} \KeywordTok{c}\NormalTok{(}\KeywordTok{list}\NormalTok{(beta_0),res), }\DataTypeTok{errors =} \KeywordTok{c}\NormalTok{(}\KeywordTok{list}\NormalTok{(}\KeywordTok{c}\NormalTok{(}\DecValTok{0}\NormalTok{,}\DecValTok{0}\NormalTok{)),err)))}
\NormalTok{\}}

\NormalTok{updateWeightsGDI <-}\StringTok{ }\NormalTok{function(y, x, w_old)\{}
  \CommentTok{#(1/length(y))*c(sum(y-p(w_old, x)), sum(x*(y-p(w_old, x))))}
  \KeywordTok{c}\NormalTok{(}\KeywordTok{sum}\NormalTok{(y-}\KeywordTok{p}\NormalTok{(w_old, x)), }\KeywordTok{sum}\NormalTok{(x*(y-}\KeywordTok{p}\NormalTok{(w_old, x))))}
\NormalTok{\}}

\NormalTok{updateWeightsSGDI <-}\StringTok{ }\NormalTok{function(y_i, x_i, w_old)\{}
  \KeywordTok{c}\NormalTok{(y_i-}\KeywordTok{p}\NormalTok{(w_old, x_i), x_i*(y_i-}\KeywordTok{p}\NormalTok{(w_old, x_i)))}
\NormalTok{\}}

\NormalTok{p <-}\StringTok{ }\NormalTok{function(w_old, x_i)\{}
  \DecValTok{1}\NormalTok{/(}\DecValTok{1}\NormalTok{+}\KeywordTok{exp}\NormalTok{(-w_old[}\DecValTok{1}\NormalTok{]-w_old[}\DecValTok{2}\NormalTok{]*x_i))}
\NormalTok{\}}

\NormalTok{inverseHessianGDII <-}\StringTok{ }\NormalTok{function(x, w_old)\{}
  \KeywordTok{solve}\NormalTok{(}
    \KeywordTok{matrix}\NormalTok{(}\KeywordTok{c}\NormalTok{(}
      \KeywordTok{sum}\NormalTok{(}\KeywordTok{p}\NormalTok{(w_old, x)*(}\DecValTok{1}\NormalTok{-}\KeywordTok{p}\NormalTok{(w_old, x))),}
      \KeywordTok{sum}\NormalTok{(x*}\KeywordTok{p}\NormalTok{(w_old, x)*(}\DecValTok{1}\NormalTok{-}\KeywordTok{p}\NormalTok{(w_old, x))),}
      \KeywordTok{sum}\NormalTok{(x*}\KeywordTok{p}\NormalTok{(w_old, x)*(}\DecValTok{1}\NormalTok{-}\KeywordTok{p}\NormalTok{(w_old, x))),}
      \KeywordTok{sum}\NormalTok{(x*x*}\KeywordTok{p}\NormalTok{(w_old, x)*(}\DecValTok{1}\NormalTok{-}\KeywordTok{p}\NormalTok{(w_old, x)))}
    \NormalTok{),}
    \DataTypeTok{nrow =}\DecValTok{2} \NormalTok{)}
  \NormalTok{)}
\NormalTok{\}}
\KeywordTok{set.seed}\NormalTok{(}\DecValTok{1283}\NormalTok{)}
\NormalTok{x <-}\StringTok{ }\KeywordTok{runif}\NormalTok{(}\DecValTok{10000}\NormalTok{)}
\NormalTok{z <-}\StringTok{ }\DecValTok{2} \NormalTok{+}\StringTok{ }\DecValTok{3}\NormalTok{*x}
\NormalTok{pr <-}\StringTok{ }\DecValTok{1}\NormalTok{/(}\DecValTok{1}\NormalTok{+}\KeywordTok{exp}\NormalTok{(-z))}
\NormalTok{y <-}\StringTok{ }\KeywordTok{rbinom}\NormalTok{(}\DecValTok{10000}\NormalTok{,}\DecValTok{1}\NormalTok{,pr)}


\NormalTok{global_loglog <-}\StringTok{ }\NormalTok{function(beta1, beta2, xX, yY)\{}
  \KeywordTok{sum}\NormalTok{(yY*(beta1+beta2*xX)-}\KeywordTok{log}\NormalTok{(}\DecValTok{1}\NormalTok{+}\KeywordTok{exp}\NormalTok{(beta1+beta2*xX)))}
\NormalTok{\}}



\NormalTok{calculate_outer <-}\StringTok{ }\NormalTok{function(x, y)\{}
  \NormalTok{## contours}
  \NormalTok{outer_res <-}\StringTok{ }\KeywordTok{outer}\NormalTok{(}\KeywordTok{seq}\NormalTok{(}\DecValTok{0}\NormalTok{,}\DecValTok{4}\NormalTok{, }\DataTypeTok{length =} \DecValTok{100}\NormalTok{),}
                     \KeywordTok{seq}\NormalTok{(}\DecValTok{0}\NormalTok{,}\DecValTok{5}\NormalTok{, }\DataTypeTok{length =} \DecValTok{100}\NormalTok{),}
                     \KeywordTok{Vectorize}\NormalTok{( function(beta1,beta2)\{}
                       \KeywordTok{global_loglog}\NormalTok{(beta1, beta2, }\DataTypeTok{xX =} \NormalTok{x, }\DataTypeTok{yY =} \NormalTok{y)}
                     \NormalTok{\} )}
  \NormalTok{)}

  \NormalTok{outer_res_melted <-}\StringTok{ }\KeywordTok{melt}\NormalTok{(outer_res)}


  \NormalTok{outer_res_melted$Var1 <-}\StringTok{ }\KeywordTok{as.factor}\NormalTok{(outer_res_melted$Var1)}
  \KeywordTok{levels}\NormalTok{(outer_res_melted$Var1) <-}\StringTok{ }\KeywordTok{as.character}\NormalTok{(}\KeywordTok{seq}\NormalTok{(}\DecValTok{0}\NormalTok{,}\DecValTok{4}\NormalTok{, }\DataTypeTok{length =} \DecValTok{100}\NormalTok{))}
  \NormalTok{outer_res_melted$Var2 <-}\StringTok{ }\KeywordTok{as.factor}\NormalTok{(outer_res_melted$Var2)}
  \KeywordTok{levels}\NormalTok{(outer_res_melted$Var2) <-}\StringTok{ }\KeywordTok{as.character}\NormalTok{(}\KeywordTok{seq}\NormalTok{(}\DecValTok{0}\NormalTok{,}\DecValTok{5}\NormalTok{, }\DataTypeTok{length =} \DecValTok{100}\NormalTok{))}
  \NormalTok{outer_res_melted$Var1 <-}\StringTok{ }\KeywordTok{as.numeric}\NormalTok{(}\KeywordTok{as.character}\NormalTok{(outer_res_melted$Var1))}
  \NormalTok{outer_res_melted$Var2 <-}\StringTok{ }\KeywordTok{as.numeric}\NormalTok{(}\KeywordTok{as.character}\NormalTok{(outer_res_melted$Var2))}
  \KeywordTok{return}\NormalTok{(outer_res_melted)}
\NormalTok{\}}



\KeywordTok{library}\NormalTok{(ggplot2); }\KeywordTok{library}\NormalTok{(ggthemes); }\KeywordTok{library}\NormalTok{(reshape2)}
\NormalTok{graphSGD <-}\StringTok{ }\NormalTok{function(beta, y, x, }\DataTypeTok{seed =} \DecValTok{4561}\NormalTok{, }\DataTypeTok{outerBounds =} \KeywordTok{calculate_outer}\NormalTok{(x,y))\{}
  \KeywordTok{set.seed}\NormalTok{(seed)}

  \NormalTok{beta <-}\StringTok{ }\KeywordTok{rev}\NormalTok{(beta)}

  \KeywordTok{logitGD}\NormalTok{(y, x, }\DataTypeTok{optim.method =} \StringTok{"GDI"}\NormalTok{,}\DataTypeTok{beta_0 =} \NormalTok{beta,}
          \DataTypeTok{eps =} \FloatTok{10e-4}\NormalTok{, }\DataTypeTok{max.iter =} \DecValTok{10000}\NormalTok{,}
          \DataTypeTok{alpha =} \NormalTok{function(t)\{}\DecValTok{1}\NormalTok{/(}\DecValTok{1000}\NormalTok{*}\KeywordTok{sqrt}\NormalTok{(t))\})$steps ->}\StringTok{ }\NormalTok{GDI.S}

  \KeywordTok{logitGD}\NormalTok{(y, x, }\DataTypeTok{optim.method =} \StringTok{"GDII"}\NormalTok{, }\DataTypeTok{beta_0 =} \NormalTok{beta,}
          \DataTypeTok{eps =} \FloatTok{10e-4}\NormalTok{, }\DataTypeTok{max.iter =} \DecValTok{5000}\NormalTok{)$steps ->}\StringTok{ }\NormalTok{GDII}

  \NormalTok{ind2 <-}\StringTok{ }\KeywordTok{sample}\NormalTok{(}\KeywordTok{length}\NormalTok{(y))}
  \KeywordTok{logitGD}\NormalTok{(y[ind2], x[ind2], }\DataTypeTok{optim.method =} \StringTok{"SGDI"}\NormalTok{, }\DataTypeTok{beta_0 =} \NormalTok{beta,}
          \DataTypeTok{max.iter =} \DecValTok{10000}\NormalTok{, }\DataTypeTok{eps =} \FloatTok{10e-4}\NormalTok{,}
          \DataTypeTok{alpha =} \NormalTok{function(t)\{}\DecValTok{1}\NormalTok{/}\KeywordTok{sqrt}\NormalTok{(t)\})$steps ->}\StringTok{ }\NormalTok{SGDI}\FloatTok{.1}\NormalTok{.S}
  \NormalTok{ind3 <-}\StringTok{ }\KeywordTok{sample}\NormalTok{(}\KeywordTok{length}\NormalTok{(y))}
  \KeywordTok{logitGD}\NormalTok{(y[ind3], x[ind3], }\DataTypeTok{optim.method =} \StringTok{"SGDI"}\NormalTok{, }\DataTypeTok{beta_0 =} \NormalTok{beta,}
          \DataTypeTok{max.iter =} \DecValTok{10000}\NormalTok{, }\DataTypeTok{eps =} \FloatTok{10e-4}\NormalTok{,}
          \DataTypeTok{alpha =} \NormalTok{function(t)\{}\DecValTok{5}\NormalTok{/}\KeywordTok{sqrt}\NormalTok{(t)\})$steps ->}\StringTok{ }\NormalTok{SGDI}\FloatTok{.5}\NormalTok{.S}
  \NormalTok{ind4 <-}\StringTok{ }\KeywordTok{sample}\NormalTok{(}\KeywordTok{length}\NormalTok{(y))}
  \KeywordTok{logitGD}\NormalTok{(y[ind4], x[ind4], }\DataTypeTok{optim.method =} \StringTok{"SGDI"}\NormalTok{, }\DataTypeTok{beta_0 =} \NormalTok{beta,}
          \DataTypeTok{max.iter =} \DecValTok{10000}\NormalTok{, }\DataTypeTok{eps =} \FloatTok{10e-4}\NormalTok{,}
          \DataTypeTok{alpha =} \NormalTok{function(t)\{}\DecValTok{6}\NormalTok{/}\KeywordTok{sqrt}\NormalTok{(t)\})$steps ->}\StringTok{ }\NormalTok{SGDI}\FloatTok{.6}\NormalTok{.S}

  \KeywordTok{do.call}\NormalTok{(rbind, }\KeywordTok{c}\NormalTok{(GDI.S, GDII, SGDI}\FloatTok{.1}\NormalTok{.S, SGDI}\FloatTok{.5}\NormalTok{.S, SGDI}\FloatTok{.6}\NormalTok{.S)) ->}\StringTok{ }\NormalTok{coeffs}
  \KeywordTok{unlist}\NormalTok{(}\KeywordTok{lapply}\NormalTok{(}\KeywordTok{list}\NormalTok{(GDI.S, GDII, SGDI}\FloatTok{.1}\NormalTok{.S, SGDI}\FloatTok{.5}\NormalTok{.S, SGDI}\FloatTok{.6}\NormalTok{.S),}
                \NormalTok{length)) ->}\StringTok{ }\NormalTok{algorithm}
  \NormalTok{data2viz <-}\StringTok{ }\KeywordTok{cbind}\NormalTok{(}\KeywordTok{as.data.frame}\NormalTok{(coeffs),}
  \DataTypeTok{algorithm =} \KeywordTok{unlist}\NormalTok{(}\KeywordTok{mapply}\NormalTok{(rep,}
                              \KeywordTok{c}\NormalTok{(}\KeywordTok{paste}\NormalTok{(}\StringTok{"GDI"}\NormalTok{, }\KeywordTok{length}\NormalTok{(GDI.S), }\StringTok{"steps"}\NormalTok{),}
                              \KeywordTok{paste}\NormalTok{(}\StringTok{"GDII"}\NormalTok{, }\KeywordTok{length}\NormalTok{(GDII), }\StringTok{"steps"}\NormalTok{),}
                              \KeywordTok{paste}\NormalTok{(}\StringTok{"SGDI.1"}\NormalTok{, }\KeywordTok{length}\NormalTok{(SGDI}\FloatTok{.1}\NormalTok{.S), }\StringTok{"steps"}\NormalTok{),}
                              \KeywordTok{paste}\NormalTok{(}\StringTok{"SGDI.5"}\NormalTok{, }\KeywordTok{length}\NormalTok{(SGDI}\FloatTok{.5}\NormalTok{.S), }\StringTok{"steps"}\NormalTok{),}
                              \KeywordTok{paste}\NormalTok{(}\StringTok{"SGDI.6"}\NormalTok{, }\KeywordTok{length}\NormalTok{(SGDI}\FloatTok{.6}\NormalTok{.S), }\StringTok{"steps"}\NormalTok{)),}
                            \NormalTok{algorithm)))}
  \KeywordTok{names}\NormalTok{(data2viz)[}\DecValTok{1}\NormalTok{:}\DecValTok{2}\NormalTok{] <-}\StringTok{ }\KeywordTok{c}\NormalTok{(}\StringTok{"Intercept"}\NormalTok{, }\StringTok{"X"}\NormalTok{)}
  \NormalTok{data2viz$algorithm <-}\StringTok{ }\KeywordTok{factor}\NormalTok{(data2viz$algorithm, }\DataTypeTok{levels =} \KeywordTok{rev}\NormalTok{(}\KeywordTok{levels}\NormalTok{(data2viz$algorithm)))}
  \NormalTok{beta[}\DecValTok{2}\NormalTok{] ->}\StringTok{ }\NormalTok{XX}
  \NormalTok{beta[}\DecValTok{1}\NormalTok{] ->}\StringTok{ }\NormalTok{YY}


  \KeywordTok{ggplot}\NormalTok{()+}
\StringTok{    }\KeywordTok{geom_path}\NormalTok{(}\KeywordTok{aes}\NormalTok{(}\DataTypeTok{x =} \NormalTok{data2viz$X,}
                  \DataTypeTok{y =} \NormalTok{data2viz$Intercept,}
                  \DataTypeTok{col =} \NormalTok{data2viz$algorithm,}
                  \DataTypeTok{group =} \NormalTok{data2viz$algorithm), }\DataTypeTok{size =} \DecValTok{1}\NormalTok{) +}
\StringTok{    }\KeywordTok{geom_point}\NormalTok{(}\KeywordTok{aes}\NormalTok{(}\KeywordTok{as.vector}\NormalTok{(}\KeywordTok{round}\NormalTok{(}\KeywordTok{coefficients}\NormalTok{(}\KeywordTok{glm}\NormalTok{(y~x,}
                          \DataTypeTok{family =} \StringTok{'binomial'}\NormalTok{)), }\DecValTok{2}\NormalTok{)[}\DecValTok{2}\NormalTok{]),}
                   \KeywordTok{as.vector}\NormalTok{(}\KeywordTok{round}\NormalTok{(}\KeywordTok{coefficients}\NormalTok{(}\KeywordTok{glm}\NormalTok{(y~x,}
                          \DataTypeTok{family =} \StringTok{'binomial'}\NormalTok{)), }\DecValTok{2}\NormalTok{)[}\DecValTok{1}\NormalTok{])),}
               \DataTypeTok{col =} \StringTok{"black"}\NormalTok{, }\DataTypeTok{size =} \DecValTok{4}\NormalTok{, }\DataTypeTok{shape =} \DecValTok{15}\NormalTok{) +}
\StringTok{    }\KeywordTok{geom_point}\NormalTok{(}\KeywordTok{aes}\NormalTok{(}\DataTypeTok{x=}\NormalTok{XX, }\DataTypeTok{y=}\NormalTok{YY),}
               \DataTypeTok{col =} \StringTok{"black"}\NormalTok{, }\DataTypeTok{size =} \DecValTok{4}\NormalTok{, }\DataTypeTok{shape =} \DecValTok{17}\NormalTok{) +}
\StringTok{    }\KeywordTok{theme_bw}\NormalTok{(}\DataTypeTok{base_size =} \DecValTok{20}\NormalTok{) +}
\StringTok{    }\KeywordTok{theme}\NormalTok{(}\DataTypeTok{panel.border =} \KeywordTok{element_blank}\NormalTok{(),}
          \DataTypeTok{legend.key =} \KeywordTok{element_blank}\NormalTok{()) +}
\StringTok{    }\KeywordTok{scale_colour_brewer}\NormalTok{(}\DataTypeTok{palette=}\StringTok{"Set1"}\NormalTok{, }\DataTypeTok{name =} \StringTok{'Algorithm'}\NormalTok{) +}
\StringTok{    }\KeywordTok{xlab}\NormalTok{(}\StringTok{'X'}\NormalTok{) +}
\StringTok{    }\KeywordTok{ylab}\NormalTok{(}\StringTok{'Intercept'}\NormalTok{)  ->}\StringTok{ }\NormalTok{pl_g}

  \KeywordTok{return}\NormalTok{(pl_g)}
\NormalTok{\}}
\end{Highlighting}
\end{Shaded}

\begin{Shaded}
\begin{Highlighting}[]
\NormalTok{full_cox_loglik <-}\StringTok{ }\NormalTok{function(beta1, beta2, x1, x2, censored)\{}
  \KeywordTok{sum}\NormalTok{(}\KeywordTok{rev}\NormalTok{(censored)*(beta1*}\KeywordTok{rev}\NormalTok{(x1) +}\StringTok{ }\NormalTok{beta2*}\KeywordTok{rev}\NormalTok{(x2) -}
\StringTok{                       }\KeywordTok{log}\NormalTok{(}\KeywordTok{cumsum}\NormalTok{(}\KeywordTok{exp}\NormalTok{(beta1*}\KeywordTok{rev}\NormalTok{(x1) +}\StringTok{ }\NormalTok{beta2*}\KeywordTok{rev}\NormalTok{(x2))))))}
\NormalTok{\}}


\NormalTok{calculate_outer_cox <-}\StringTok{ }\NormalTok{function(x1, x2, censored)\{}
  \NormalTok{## contours}
  \NormalTok{outer_res <-}\StringTok{ }\KeywordTok{outer}\NormalTok{(}\KeywordTok{seq}\NormalTok{(-}\DecValTok{1}\NormalTok{,}\DecValTok{3}\NormalTok{, }\DataTypeTok{length =} \DecValTok{100}\NormalTok{),}
           \KeywordTok{seq}\NormalTok{(}\DecValTok{0}\NormalTok{,}\DecValTok{4}\NormalTok{, }\DataTypeTok{length =} \DecValTok{100}\NormalTok{),}
           \KeywordTok{Vectorize}\NormalTok{( function(beta1,beta2)\{}
             \KeywordTok{full_cox_loglik}\NormalTok{(beta1, beta2, }\DataTypeTok{x1 =} \NormalTok{x1, }\DataTypeTok{x2 =} \NormalTok{x2, }\DataTypeTok{censored =} \NormalTok{censored)}
           \NormalTok{\} )}
  \NormalTok{)}
  \NormalTok{outer_res_melted <-}\StringTok{ }\KeywordTok{melt}\NormalTok{(outer_res)}
  \NormalTok{outer_res_melted$Var1 <-}\StringTok{ }\KeywordTok{as.factor}\NormalTok{(outer_res_melted$Var1)}
  \KeywordTok{levels}\NormalTok{(outer_res_melted$Var1) <-}\StringTok{ }\KeywordTok{as.character}\NormalTok{(}\KeywordTok{seq}\NormalTok{(-}\DecValTok{1}\NormalTok{,}\DecValTok{3}\NormalTok{, }\DataTypeTok{length =} \DecValTok{100}\NormalTok{))}
  \NormalTok{outer_res_melted$Var2 <-}\StringTok{ }\KeywordTok{as.factor}\NormalTok{(outer_res_melted$Var2)}
  \KeywordTok{levels}\NormalTok{(outer_res_melted$Var2) <-}\StringTok{ }\KeywordTok{as.character}\NormalTok{(}\KeywordTok{seq}\NormalTok{(}\DecValTok{0}\NormalTok{,}\DecValTok{4}\NormalTok{, }\DataTypeTok{length =} \DecValTok{100}\NormalTok{))}
  \NormalTok{outer_res_melted$Var1 <-}\StringTok{ }\KeywordTok{as.numeric}\NormalTok{(}\KeywordTok{as.character}\NormalTok{(outer_res_melted$Var1))}
  \NormalTok{outer_res_melted$Var2 <-}\StringTok{ }\KeywordTok{as.numeric}\NormalTok{(}\KeywordTok{as.character}\NormalTok{(outer_res_melted$Var2))}
  \KeywordTok{return}\NormalTok{(outer_res_melted)}
\NormalTok{\}}
\NormalTok{simulateCoxSGD <-}\StringTok{ }\NormalTok{function(}\DataTypeTok{dCox =} \NormalTok{dCox, }\DataTypeTok{learningRates =} \NormalTok{function(x)\{}\DecValTok{1}\NormalTok{/x\},}
                      \DataTypeTok{epsilon =} \FloatTok{1e-03}\NormalTok{, }\DataTypeTok{beta_0 =} \KeywordTok{c}\NormalTok{(}\DecValTok{0}\NormalTok{,}\DecValTok{0}\NormalTok{), }\DataTypeTok{max.iter =} \DecValTok{100}\NormalTok{)\{}

  \KeywordTok{sample}\NormalTok{(}\DecValTok{1}\NormalTok{:}\DecValTok{90}\NormalTok{, }\DataTypeTok{size =} \DecValTok{10}\NormalTok{^}\DecValTok{4}\NormalTok{, }\DataTypeTok{replace =} \OtherTok{TRUE}\NormalTok{) ->}\StringTok{ }\NormalTok{group}
  \KeywordTok{split}\NormalTok{(dCox, group) ->}\StringTok{ }\NormalTok{dCox_splitted}
  \KeywordTok{coxphSGD}\NormalTok{(}\KeywordTok{Surv}\NormalTok{(time, status)~x}\FloatTok{.1}\NormalTok{+x}\FloatTok{.2}\NormalTok{, }\DataTypeTok{data =} \NormalTok{dCox_splitted,}
           \DataTypeTok{epsilon =} \NormalTok{epsilon, }\DataTypeTok{learningRates =} \NormalTok{learningRates, }
           \DataTypeTok{beta_0 =} \NormalTok{beta_0, }\DataTypeTok{max.iter =} \NormalTok{max.iter*}\DecValTok{90}\NormalTok{) ->}\StringTok{ }\NormalTok{estimates}

  \KeywordTok{sample}\NormalTok{(}\DecValTok{1}\NormalTok{:}\DecValTok{60}\NormalTok{, }\DataTypeTok{size =} \DecValTok{10}\NormalTok{^}\DecValTok{4}\NormalTok{, }\DataTypeTok{replace =} \OtherTok{TRUE}\NormalTok{) ->}\StringTok{ }\NormalTok{group}
  \KeywordTok{split}\NormalTok{(dCox, group) ->}\StringTok{ }\NormalTok{dCox_splitted}
  \KeywordTok{coxphSGD}\NormalTok{(}\KeywordTok{Surv}\NormalTok{(time, status)~x}\FloatTok{.1}\NormalTok{+x}\FloatTok{.2}\NormalTok{, }\DataTypeTok{data =} \NormalTok{dCox_splitted,}
           \DataTypeTok{epsilon =} \NormalTok{epsilon, }\DataTypeTok{learningRates =} \NormalTok{learningRates,}
           \DataTypeTok{beta_0 =} \NormalTok{beta_0, }\DataTypeTok{max.iter =} \NormalTok{max.iter*}\DecValTok{60}\NormalTok{) ->}\StringTok{ }\NormalTok{estimates2}

  \KeywordTok{sample}\NormalTok{(}\DecValTok{1}\NormalTok{:}\DecValTok{120}\NormalTok{, }\DataTypeTok{size =} \DecValTok{10}\NormalTok{^}\DecValTok{4}\NormalTok{, }\DataTypeTok{replace =} \OtherTok{TRUE}\NormalTok{) ->}\StringTok{ }\NormalTok{group}
  \KeywordTok{split}\NormalTok{(dCox, group) ->}\StringTok{ }\NormalTok{dCox_splitted}
  \KeywordTok{coxphSGD}\NormalTok{(}\KeywordTok{Surv}\NormalTok{(time, status)~x}\FloatTok{.1}\NormalTok{+x}\FloatTok{.2}\NormalTok{, }\DataTypeTok{data =} \NormalTok{dCox_splitted,}
           \DataTypeTok{epsilon =} \NormalTok{epsilon, }\DataTypeTok{learningRates =} \NormalTok{learningRates,}
           \DataTypeTok{beta_0 =} \NormalTok{beta_0, }\DataTypeTok{max.iter =} \NormalTok{max.iter*}\DecValTok{120}\NormalTok{) ->}\StringTok{ }\NormalTok{estimates3}

  \KeywordTok{sample}\NormalTok{(}\DecValTok{1}\NormalTok{:}\DecValTok{200}\NormalTok{, }\DataTypeTok{size =} \DecValTok{10}\NormalTok{^}\DecValTok{4}\NormalTok{, }\DataTypeTok{replace =} \OtherTok{TRUE}\NormalTok{) ->}\StringTok{ }\NormalTok{group}
  \KeywordTok{split}\NormalTok{(dCox, group) ->}\StringTok{ }\NormalTok{dCox_splitted}
  \KeywordTok{coxphSGD}\NormalTok{(}\KeywordTok{Surv}\NormalTok{(time, status)~x}\FloatTok{.1}\NormalTok{+x}\FloatTok{.2}\NormalTok{, }\DataTypeTok{data =} \NormalTok{dCox_splitted,}
           \DataTypeTok{epsilon =} \NormalTok{epsilon, }\DataTypeTok{learningRates =} \NormalTok{learningRates,}
           \DataTypeTok{beta_0 =} \NormalTok{beta_0, }\DataTypeTok{max.iter =} \NormalTok{max.iter*}\DecValTok{200}\NormalTok{) ->}\StringTok{ }\NormalTok{estimates4}


  \KeywordTok{sample}\NormalTok{(}\DecValTok{1}\NormalTok{:}\DecValTok{30}\NormalTok{, }\DataTypeTok{size =} \DecValTok{10}\NormalTok{^}\DecValTok{4}\NormalTok{, }\DataTypeTok{replace =} \OtherTok{TRUE}\NormalTok{) ->}\StringTok{ }\NormalTok{group}
  \KeywordTok{split}\NormalTok{(dCox, group) ->}\StringTok{ }\NormalTok{dCox_splitted}
  \KeywordTok{coxphSGD}\NormalTok{(}\KeywordTok{Surv}\NormalTok{(time, status)~x}\FloatTok{.1}\NormalTok{+x}\FloatTok{.2}\NormalTok{, }\DataTypeTok{data =} \NormalTok{dCox_splitted,}
           \DataTypeTok{epsilon =} \NormalTok{epsilon, }\DataTypeTok{learningRates =} \NormalTok{learningRates,}
           \DataTypeTok{beta_0 =} \NormalTok{beta_0, }\DataTypeTok{max.iter =} \NormalTok{max.iter*}\DecValTok{30}\NormalTok{) ->}\StringTok{ }\NormalTok{estimates5}


  \KeywordTok{sample}\NormalTok{(}\DecValTok{1}\NormalTok{:}\DecValTok{10}\NormalTok{, }\DataTypeTok{size =} \DecValTok{10}\NormalTok{^}\DecValTok{4}\NormalTok{, }\DataTypeTok{replace =} \OtherTok{TRUE}\NormalTok{) ->}\StringTok{ }\NormalTok{group}
  \KeywordTok{split}\NormalTok{(dCox, group) ->}\StringTok{ }\NormalTok{dCox_splitted}
  \KeywordTok{coxphSGD}\NormalTok{(}\KeywordTok{Surv}\NormalTok{(time, status)~x}\FloatTok{.1}\NormalTok{+x}\FloatTok{.2}\NormalTok{, }\DataTypeTok{data =} \NormalTok{dCox_splitted,}
           \DataTypeTok{epsilon =} \NormalTok{epsilon, }\DataTypeTok{learningRates =} \NormalTok{learningRates,}
           \DataTypeTok{beta_0 =} \NormalTok{beta_0, }\DataTypeTok{max.iter =} \NormalTok{max.iter*}\DecValTok{10}\NormalTok{) ->}\StringTok{ }\NormalTok{estimates6}

  \KeywordTok{t}\NormalTok{(}\KeywordTok{simplify2array}\NormalTok{(estimates$coefficients)) %>%}
\StringTok{    }\KeywordTok{as.data.frame}\NormalTok{() ->}\StringTok{ }\NormalTok{df1}
  \KeywordTok{t}\NormalTok{(}\KeywordTok{simplify2array}\NormalTok{(estimates2$coefficients)) %>%}
\StringTok{    }\KeywordTok{as.data.frame}\NormalTok{() ->}\StringTok{ }\NormalTok{df2}
  \KeywordTok{t}\NormalTok{(}\KeywordTok{simplify2array}\NormalTok{(estimates3$coefficients)) %>%}
\StringTok{    }\KeywordTok{as.data.frame}\NormalTok{() ->}\StringTok{ }\NormalTok{df3}
  \KeywordTok{t}\NormalTok{(}\KeywordTok{simplify2array}\NormalTok{(estimates4$coefficients)) %>%}
\StringTok{    }\KeywordTok{as.data.frame}\NormalTok{() ->}\StringTok{ }\NormalTok{df4}
  \KeywordTok{t}\NormalTok{(}\KeywordTok{simplify2array}\NormalTok{(estimates5$coefficients)) %>%}
\StringTok{    }\KeywordTok{as.data.frame}\NormalTok{() ->}\StringTok{ }\NormalTok{df5}
  \KeywordTok{t}\NormalTok{(}\KeywordTok{simplify2array}\NormalTok{(estimates6$coefficients)) %>%}
\StringTok{    }\KeywordTok{as.data.frame}\NormalTok{() ->}\StringTok{ }\NormalTok{df6}

  \NormalTok{df1 %>%}
\StringTok{    }\KeywordTok{mutate}\NormalTok{(}\DataTypeTok{version =} \KeywordTok{paste}\NormalTok{(}\StringTok{"90 batches,"}\NormalTok{, }\KeywordTok{nrow}\NormalTok{(df1), }\StringTok{" steps"}\NormalTok{)) %>%}
\StringTok{    }\KeywordTok{bind_rows}\NormalTok{(df2 %>%}
\StringTok{                }\KeywordTok{mutate}\NormalTok{(}\DataTypeTok{version =} \KeywordTok{paste}\NormalTok{(}\StringTok{"60 batches,"}\NormalTok{, }\KeywordTok{nrow}\NormalTok{(df2), }\StringTok{" steps"}\NormalTok{))) %>%}
\StringTok{    }\KeywordTok{bind_rows}\NormalTok{(df3 %>%}
\StringTok{                }\KeywordTok{mutate}\NormalTok{(}\DataTypeTok{version =} \KeywordTok{paste}\NormalTok{(}\StringTok{"120 batches,"}\NormalTok{, }\KeywordTok{nrow}\NormalTok{(df3), }\StringTok{" steps"}\NormalTok{))) %>%}
\StringTok{    }\KeywordTok{bind_rows}\NormalTok{(df4 %>%}
\StringTok{                }\KeywordTok{mutate}\NormalTok{(}\DataTypeTok{version =} \KeywordTok{paste}\NormalTok{(}\StringTok{"200 batches,"}\NormalTok{, }\KeywordTok{nrow}\NormalTok{(df4), }\StringTok{" steps"}\NormalTok{))) %>%}
\StringTok{    }\KeywordTok{bind_rows}\NormalTok{(df5 %>%}
\StringTok{                }\KeywordTok{mutate}\NormalTok{(}\DataTypeTok{version =} \KeywordTok{paste}\NormalTok{(}\StringTok{"30 batches,"}\NormalTok{, }\KeywordTok{nrow}\NormalTok{(df5), }\StringTok{" steps"}\NormalTok{))) %>%}
\StringTok{    }\KeywordTok{bind_rows}\NormalTok{(df6 %>%}
\StringTok{                }\KeywordTok{mutate}\NormalTok{(}\DataTypeTok{version =} \KeywordTok{paste}\NormalTok{(}\StringTok{"10 batches,"}\NormalTok{, }\KeywordTok{nrow}\NormalTok{(df6), }\StringTok{" steps"}\NormalTok{))) ->}\StringTok{ }\NormalTok{d2ggplot}

  \KeywordTok{return}\NormalTok{(}\KeywordTok{list}\NormalTok{(}\DataTypeTok{d2ggplot =} \NormalTok{d2ggplot, }\DataTypeTok{est1 =} \NormalTok{estimates, }\DataTypeTok{est2 =} \NormalTok{estimates2,}
              \DataTypeTok{est3 =} \NormalTok{estimates3, }\DataTypeTok{est4 =} \NormalTok{estimates4, }\DataTypeTok{est5 =} \NormalTok{estimates5))}

\NormalTok{\}}
\KeywordTok{simulateCoxSGD}\NormalTok{(dCox, }\DataTypeTok{learningRates =} \NormalTok{function(x)\{}\DecValTok{1}\NormalTok{/(}\DecValTok{100}\NormalTok{*}\KeywordTok{sqrt}\NormalTok{(x))\},}
               \DataTypeTok{max.iter =} \DecValTok{10}\NormalTok{, }\DataTypeTok{epsilon =} \FloatTok{1e-5}\NormalTok{) ->}\StringTok{ }\NormalTok{d2ggplot}
\NormalTok{d2ggplot ->}\StringTok{ }\NormalTok{backpack}
\NormalTok{d2ggplot <-}\StringTok{ }\NormalTok{d2ggplot$d2ggplot}
\NormalTok{beta_0 =}\StringTok{ }\KeywordTok{c}\NormalTok{(}\DecValTok{0}\NormalTok{,}\DecValTok{0}\NormalTok{)}
\NormalTok{solution =}\StringTok{ }\KeywordTok{c}\NormalTok{(}\DecValTok{1}\NormalTok{,}\DecValTok{3}\NormalTok{)}

\KeywordTok{pdf}\NormalTok{(}\DataTypeTok{file =} \StringTok{"b_0_0_iter_10_e-5_100sqrt_878.pdf"}\NormalTok{, }\DataTypeTok{width =} \DecValTok{10}\NormalTok{, }\DataTypeTok{height =} \DecValTok{10}\NormalTok{)}
\KeywordTok{ggplot}\NormalTok{() +}
\StringTok{  }\KeywordTok{stat_contour}\NormalTok{(}\KeywordTok{aes}\NormalTok{(}\DataTypeTok{x=}\NormalTok{outerCox$Var1,}
                   \DataTypeTok{y=}\NormalTok{outerCox$Var2,}
                   \DataTypeTok{z=}\NormalTok{outerCox$value),}
               \DataTypeTok{bins =} \DecValTok{40}\NormalTok{, }\DataTypeTok{alpha =} \FloatTok{0.25}\NormalTok{) +}
\StringTok{  }\KeywordTok{geom_path}\NormalTok{(}\KeywordTok{aes}\NormalTok{(d2ggplot$V1, d2ggplot$V2, }\DataTypeTok{group =} \NormalTok{d2ggplot$version,}
                \DataTypeTok{colour =} \NormalTok{d2ggplot$version), }\DataTypeTok{size =} \DecValTok{1}\NormalTok{) +}
\StringTok{  }\KeywordTok{theme_bw}\NormalTok{(}\DataTypeTok{base_size =} \DecValTok{20}\NormalTok{) +}
\StringTok{  }\KeywordTok{theme}\NormalTok{(}\DataTypeTok{panel.border =} \KeywordTok{element_blank}\NormalTok{(),}
        \DataTypeTok{legend.key =} \KeywordTok{element_blank}\NormalTok{(), }\DataTypeTok{legend.position =} \StringTok{"top"}\NormalTok{) +}
\StringTok{  }\KeywordTok{scale_colour_brewer}\NormalTok{(}\DataTypeTok{palette=}\StringTok{"Dark2"}\NormalTok{, }\DataTypeTok{name =} \StringTok{'Algorithm }\CharTok{\textbackslash{}n}\StringTok{ & Steps'}\NormalTok{) +}
\StringTok{  }\KeywordTok{geom_point}\NormalTok{(}\KeywordTok{aes}\NormalTok{(}\DataTypeTok{x =} \NormalTok{beta_0[}\DecValTok{1}\NormalTok{], }\DataTypeTok{y =} \NormalTok{beta_0[}\DecValTok{2}\NormalTok{]), }\DataTypeTok{col =} \StringTok{"black"}\NormalTok{, }\DataTypeTok{size =} \DecValTok{4}\NormalTok{, }\DataTypeTok{shape =} \DecValTok{17}\NormalTok{) +}
\StringTok{  }\KeywordTok{geom_point}\NormalTok{(}\KeywordTok{aes}\NormalTok{(}\DataTypeTok{x =} \NormalTok{solution[}\DecValTok{1}\NormalTok{], }\DataTypeTok{y =} \NormalTok{solution[}\DecValTok{2}\NormalTok{]), }\DataTypeTok{col =} \StringTok{"black"}\NormalTok{, }\DataTypeTok{size =} \DecValTok{4}\NormalTok{, }\DataTypeTok{shape =} \DecValTok{15}\NormalTok{) +}
\StringTok{  }\KeywordTok{xlab}\NormalTok{(}\StringTok{"X1"}\NormalTok{) +}\StringTok{ }\KeywordTok{ylab}\NormalTok{(}\StringTok{"X2"}\NormalTok{) +}
\StringTok{  }\KeywordTok{guides}\NormalTok{(}\DataTypeTok{col =} \KeywordTok{guide_legend}\NormalTok{(}\DataTypeTok{ncol =} \DecValTok{3}\NormalTok{))}
\KeywordTok{dev.off}\NormalTok{()}
\end{Highlighting}
\end{Shaded}

\begin{Shaded}
\begin{Highlighting}[]
\NormalTok{extractSurvival <-}\StringTok{ }\NormalTok{function(cohorts)\{}
  
  \NormalTok{survivalData <-}\StringTok{ }\KeywordTok{list}\NormalTok{()}
  \NormalTok{for(i in cohorts)\{}
    \KeywordTok{get}\NormalTok{(}\KeywordTok{paste0}\NormalTok{(i, }\StringTok{".clinical"}\NormalTok{), }\DataTypeTok{envir =} \NormalTok{.GlobalEnv) %>%}
\StringTok{                }\KeywordTok{select}\NormalTok{(patient.bcr_patient_barcode,}
                             \NormalTok{patient.vital_status,}
                             \NormalTok{patient.days_to_last_followup,}
                             \NormalTok{patient.days_to_death ) %>%}
\StringTok{                }\KeywordTok{mutate}\NormalTok{(}\DataTypeTok{bcr_patient_barcode =} \KeywordTok{toupper}\NormalTok{(patient.bcr_patient_barcode),}
                       \DataTypeTok{patient.vital_status =} \KeywordTok{ifelse}\NormalTok{(patient.vital_status %>%}
\StringTok{                                                 }\KeywordTok{as.character}\NormalTok{() ==}\StringTok{"dead"}\NormalTok{,}\DecValTok{1}\NormalTok{,}\DecValTok{0}\NormalTok{),}
                   \DataTypeTok{barcode =} \NormalTok{patient.bcr_patient_barcode %>%}
\StringTok{                                     }\KeywordTok{as.character}\NormalTok{(),}
                 \DataTypeTok{times =} \KeywordTok{ifelse}\NormalTok{( !}\KeywordTok{is.na}\NormalTok{(patient.days_to_last_followup),}
                      \NormalTok{patient.days_to_last_followup %>%}
\StringTok{                        }\KeywordTok{as.character}\NormalTok{() %>%}
\StringTok{                        }\KeywordTok{as.numeric}\NormalTok{(),}
               \NormalTok{patient.days_to_death %>%}
\StringTok{                        }\KeywordTok{as.character}\NormalTok{() %>%}
\StringTok{                        }\KeywordTok{as.numeric}\NormalTok{() )}
                     \NormalTok{) %>%}
\StringTok{   }\KeywordTok{filter}\NormalTok{(!}\KeywordTok{is.na}\NormalTok{(times)) ->}\StringTok{ }\NormalTok{survivalData[[i]]}
  
  
  \NormalTok{\}}
  \KeywordTok{do.call}\NormalTok{(rbind,survivalData) %>%}
\StringTok{    }\KeywordTok{select}\NormalTok{(bcr_patient_barcode, patient.vital_status, times) %>%}
\StringTok{    }\NormalTok{unique  }

\NormalTok{\}}


\NormalTok{extractMutations <-}\StringTok{ }\NormalTok{function(cohorts, prc)\{}
  \NormalTok{mutationsData <-}\StringTok{ }\KeywordTok{list}\NormalTok{()}
  \NormalTok{for(i in cohorts)\{}
    \KeywordTok{get}\NormalTok{(}\KeywordTok{paste0}\NormalTok{(i, }\StringTok{".mutations"}\NormalTok{), }\DataTypeTok{envir =} \NormalTok{.GlobalEnv) %>%}
\StringTok{      }\KeywordTok{select}\NormalTok{(Hugo_Symbol, bcr_patient_barcode) %>%}
\StringTok{      }\KeywordTok{filter}\NormalTok{(}\KeywordTok{nchar}\NormalTok{(bcr_patient_barcode)==}\DecValTok{15}\NormalTok{) %>%}
\StringTok{      }\KeywordTok{filter}\NormalTok{(}\KeywordTok{substr}\NormalTok{(bcr_patient_barcode, }\DecValTok{14}\NormalTok{, }\DecValTok{15}\NormalTok{)==}\StringTok{"01"}\NormalTok{) %>%}
\StringTok{      }\NormalTok{unique ->}\StringTok{ }\NormalTok{mutationsData[[i]]}
  \NormalTok{\}}
  \KeywordTok{do.call}\NormalTok{(rbind,mutationsData) %>%}\StringTok{ }\NormalTok{unique ->}\StringTok{ }\NormalTok{mutationsData}
  
  \NormalTok{mutationsData %>%}
\StringTok{    }\KeywordTok{group_by}\NormalTok{(Hugo_Symbol) %>%}
\StringTok{    }\KeywordTok{summarise}\NormalTok{(}\DataTypeTok{count =} \KeywordTok{n}\NormalTok{()) %>%}
\StringTok{    }\KeywordTok{arrange}\NormalTok{(}\KeywordTok{desc}\NormalTok{(count)) %>%}
\StringTok{    }\KeywordTok{mutate}\NormalTok{(}\DataTypeTok{count_prc =} \NormalTok{count/}\KeywordTok{length}\NormalTok{(}\KeywordTok{unique}\NormalTok{(mutationsData$bcr_patient_barcode))) %>%}
\StringTok{    }\KeywordTok{filter_}\NormalTok{(}\KeywordTok{paste0}\NormalTok{(}\StringTok{"count_prc > "}\NormalTok{,prc)) %>%}
\StringTok{    }\KeywordTok{select}\NormalTok{(Hugo_Symbol) %>%}
\StringTok{    }\NormalTok{unlist ->}\StringTok{ }\NormalTok{topGenes}
  
  \NormalTok{mutationsData %>%}
\StringTok{    }\KeywordTok{filter}\NormalTok{(Hugo_Symbol %in%}\StringTok{ }\NormalTok{topGenes) ->}\StringTok{ }\NormalTok{mutationsData_top}
  
  
  \NormalTok{mutationsData_top %>%}
\StringTok{    }\NormalTok{dplyr::}\KeywordTok{group_by}\NormalTok{(bcr_patient_barcode) %>%}
\StringTok{    }\NormalTok{dplyr::}\KeywordTok{summarise}\NormalTok{(}\DataTypeTok{count =} \KeywordTok{n}\NormalTok{()) %>%}
\StringTok{    }\KeywordTok{group_by}\NormalTok{(count) %>%}\StringTok{ }
\StringTok{    }\KeywordTok{summarise}\NormalTok{(}\DataTypeTok{total =} \KeywordTok{n}\NormalTok{()) %>%}
\StringTok{    }\KeywordTok{arrange}\NormalTok{(}\KeywordTok{desc}\NormalTok{(count))}
\CommentTok{#   }
\CommentTok{#   mutationsData_top %>%}
\CommentTok{#     spread(Hugo_Symbol, bcr_patient_barcode) -> mutationsData_top_sp}
  
  \KeywordTok{as.data.table}\NormalTok{(mutationsData_top) ->}\StringTok{ }\NormalTok{mutationsData_top_DT}
  \KeywordTok{dcast.data.table}\NormalTok{(mutationsData_top_DT, bcr_patient_barcode ~}\StringTok{ }\NormalTok{Hugo_Symbol , }\DataTypeTok{fill =} \DecValTok{0}\NormalTok{) %>%}
\StringTok{    }\NormalTok{as.data.frame ->}\StringTok{ }\NormalTok{mutationsData_top_dcasted}
  
  \NormalTok{mutationsData_top_dcasted[,-}\DecValTok{1}\NormalTok{][mutationsData_top_dcasted[,-}\DecValTok{1}\NormalTok{] !=}\StringTok{ "0"}\NormalTok{] <-}\StringTok{ }\DecValTok{1}

  \NormalTok{mutationsData_top_dcasted ->}\StringTok{ }\NormalTok{result}
  \KeywordTok{names}\NormalTok{(result) <-}\StringTok{ }\KeywordTok{gsub}\NormalTok{(}\KeywordTok{names}\NormalTok{(result),}\DataTypeTok{pattern =} \StringTok{"-"}\NormalTok{,  }\DataTypeTok{replacement =} \StringTok{""}\NormalTok{)}
  \NormalTok{result}
\NormalTok{\}}


\NormalTok{extractCohortIntersection <-}\StringTok{ }\NormalTok{function()\{}
  
  \KeywordTok{data}\NormalTok{(}\DataTypeTok{package =} \StringTok{"RTCGA.mutations"}\NormalTok{)$results[,}\DecValTok{3}\NormalTok{] %>%}
\StringTok{    }\KeywordTok{gsub}\NormalTok{(}\StringTok{".mutations"}\NormalTok{, }\StringTok{""}\NormalTok{, }\DataTypeTok{x =} \NormalTok{.) ->}\StringTok{ }\NormalTok{mutations_data}
  \KeywordTok{data}\NormalTok{(}\DataTypeTok{package =} \StringTok{"RTCGA.clinical"}\NormalTok{)$results[,}\DecValTok{3}\NormalTok{] %>%}
\StringTok{    }\KeywordTok{gsub}\NormalTok{(}\StringTok{".clinical"}\NormalTok{, }\StringTok{""}\NormalTok{, }\DataTypeTok{x =} \NormalTok{.) ->}\StringTok{ }\NormalTok{clinical_data}
  
  \KeywordTok{intersect}\NormalTok{(mutations_data, clinical_data)}
\NormalTok{\}}

\NormalTok{prepareCoxDataSplit <-}\StringTok{ }\NormalTok{function(mutationsData, survivalData, groups, }\DataTypeTok{seed =} \DecValTok{4561}\NormalTok{)\{}
  \NormalTok{mutationsData %>%}
\StringTok{  }\KeywordTok{mutate}\NormalTok{(}\DataTypeTok{bcr_patient_barcode =} \KeywordTok{substr}\NormalTok{(bcr_patient_barcode,}\DecValTok{1}\NormalTok{,}\DecValTok{12}\NormalTok{)) %>%}
\StringTok{  }\KeywordTok{left_join}\NormalTok{(survivalData,}
            \DataTypeTok{by =} \StringTok{"bcr_patient_barcode"}\NormalTok{) ->}\StringTok{ }\NormalTok{coxData}

  \NormalTok{coxData <-}\StringTok{ }\NormalTok{coxData[, -}\KeywordTok{c}\NormalTok{(}\DecValTok{1}\NormalTok{,}\DecValTok{2}\NormalTok{)]}
  
  \NormalTok{coxData %>%}
\StringTok{    }\KeywordTok{filter}\NormalTok{(times >}\StringTok{ }\DecValTok{0}\NormalTok{) %>%}
\StringTok{    }\KeywordTok{filter}\NormalTok{(!}\KeywordTok{is.na}\NormalTok{(times)) ->}\StringTok{ }\NormalTok{coxData}
  
  \KeywordTok{apply}\NormalTok{(coxData[,-}\KeywordTok{c}\NormalTok{(}\DecValTok{1092}\NormalTok{, }\DecValTok{1093}\NormalTok{)], }\DataTypeTok{MARGIN =} \DecValTok{2}\NormalTok{,function(x)\{}
    \KeywordTok{as.numeric}\NormalTok{(}\KeywordTok{as.character}\NormalTok{(x))}
  \NormalTok{\}) ->}\StringTok{ }\NormalTok{coxData[,-}\KeywordTok{c}\NormalTok{(}\DecValTok{1092}\NormalTok{, }\DecValTok{1093}\NormalTok{)]}
  
  \KeywordTok{set.seed}\NormalTok{(seed)}
  \KeywordTok{sample}\NormalTok{(groups, }\DataTypeTok{replace =} \OtherTok{TRUE}\NormalTok{, }\DataTypeTok{size =} \DecValTok{6085}\NormalTok{) ->}\StringTok{ }\NormalTok{groups}
  \KeywordTok{split}\NormalTok{(coxData, groups) }\CommentTok{#coxData_split}
\NormalTok{\}}

\NormalTok{prepareForumlaSGD <-}\StringTok{ }\NormalTok{function(coxData)\{}
  \KeywordTok{as.formula}\NormalTok{(}\KeywordTok{paste}\NormalTok{(}\StringTok{"Surv(times, patient.vital_status) ~ "}\NormalTok{,}
                   \KeywordTok{paste}\NormalTok{(}\KeywordTok{names}\NormalTok{(coxData[[}\DecValTok{1}\NormalTok{]])[-}\KeywordTok{c}\NormalTok{(}\DecValTok{1092}\NormalTok{, }\DecValTok{1093}\NormalTok{)],}
                         \DataTypeTok{collapse=}\StringTok{"+"}\NormalTok{), }\DataTypeTok{collapse =} \StringTok{""}\NormalTok{))}
\NormalTok{\}}


\NormalTok{full_cox_loglik_matrix <-}\StringTok{ }\NormalTok{function(beta, x, censored)\{}
  \KeywordTok{order}\NormalTok{(x$times) ->}\StringTok{ }\NormalTok{order2}
  \NormalTok{x[order2, ] ->}\StringTok{ }\NormalTok{xORD}
  \NormalTok{censored[order2] ->}\StringTok{ }\NormalTok{censORD}
  \KeywordTok{sum}\NormalTok{(censORD*(beta%*%x[, -}\KeywordTok{which}\NormalTok{(}\KeywordTok{names}\NormalTok{(x)==}\StringTok{'times'}\NormalTok{)] -}
\StringTok{                       }\KeywordTok{log}\NormalTok{(}\KeywordTok{cumsum}\NormalTok{(}\KeywordTok{exp}\NormalTok{(beta1*}\KeywordTok{rev}\NormalTok{(x1) +}\StringTok{ }\NormalTok{beta2*}\KeywordTok{rev}\NormalTok{(x2))))))}
\NormalTok{\}}



\KeywordTok{library}\NormalTok{(dplyr)}
\end{Highlighting}
\end{Shaded}

\begin{verbatim}

Attaching package: 'dplyr'

The following objects are masked from 'package:stats':

    filter, lag

The following objects are masked from 'package:base':

    intersect, setdiff, setequal, union
\end{verbatim}

\begin{Shaded}
\begin{Highlighting}[]
\KeywordTok{library}\NormalTok{(RTCGA.clinical)}
\end{Highlighting}
\end{Shaded}

\begin{verbatim}
Loading required package: RTCGA
Loading required package: knitr
Welcome to the RTCGA (version: 1.1.10).
\end{verbatim}

\begin{Shaded}
\begin{Highlighting}[]
\KeywordTok{library}\NormalTok{(RTCGA.mutations)}
\KeywordTok{library}\NormalTok{(data.table)}
\end{Highlighting}
\end{Shaded}

\begin{verbatim}

Attaching package: 'data.table'

The following objects are masked from 'package:dplyr':

    between, last
\end{verbatim}

\begin{Shaded}
\begin{Highlighting}[]
\KeywordTok{library}\NormalTok{(coxphSGD)}
\end{Highlighting}
\end{Shaded}

\begin{verbatim}
Loading required package: survival

Attaching package: 'coxphSGD'

The following object is masked _by_ '.GlobalEnv':

    dataCox
\end{verbatim}

Do analizy badającej wpływ występowania mutacji genów na czas przeżycia
wykorzystano dane kliniczne i dane o występujących u pacjentów mutacjach
genetycznych. Starano się wykorzystać dane ze wszystkich 38 dostępnych
kohort nowotworowych z badania \textit{The Cancer Genome Atlas} (TCGA),
jednak nie dla wszystkich kohort umieszczono w badaniu dane o mutacjach.
Częśc wspólną nazw dla kohort zawierających zarówno dane kliniczne oraz
dane o mutacjach wygenerowaną dzięki wywołaniu

\begin{Shaded}
\begin{Highlighting}[]
\NormalTok{(}\KeywordTok{extractCohortIntersection}\NormalTok{() ->}\StringTok{ }\NormalTok{cohorts)}
\end{Highlighting}
\end{Shaded}

\begin{verbatim}
 [1] "ACC"      "BLCA"     "BRCA"     "CESC"     "CHOL"     "COAD"    
 [7] "COADREAD" "DLBC"     "ESCA"     "GBM"      "GBMLGG"   "HNSC"    
[13] "KICH"     "KIPAN"    "KIRC"     "KIRP"     "LAML"     "LGG"     
[19] "LIHC"     "LUAD"     "LUSC"     "OV"       "PAAD"     "PCPG"    
[25] "PRAD"     "READ"     "SARC"     "SKCM"     "STAD"     "STES"    
[31] "TGCT"     "THCA"     "UCEC"     "UCS"      "UVM"     
\end{verbatim}

Następnie dla tak otrzymanych 35 kohort nowotworowych uzyskano dane o
statusie pacjenta (śmierć bądź cenzurowanie) oraz jego czasie spędzonym
pod obseracją dzięki funkcji

\begin{Shaded}
\begin{Highlighting}[]
\KeywordTok{head}\NormalTok{(}\KeywordTok{extractSurvival}\NormalTok{(cohorts) ->}\StringTok{ }\NormalTok{survivalData)}
\end{Highlighting}
\end{Shaded}

\begin{verbatim}
      bcr_patient_barcode patient.vital_status times
ACC.1        TCGA-OR-A5J1                    1  1355
ACC.2        TCGA-OR-A5J2                    1  1677
ACC.3        TCGA-OR-A5J3                    0  1942
ACC.4        TCGA-OR-A5J4                    1   423
ACC.5        TCGA-OR-A5J5                    1   365
ACC.6        TCGA-OR-A5J6                    0  2428
\end{verbatim}

Dane o mutacjach występujących wśród tkanek nowotworowych kolejnych
pacjentów uzyskano za pomocą

\begin{Shaded}
\begin{Highlighting}[]
\KeywordTok{extractMutations}\NormalTok{(cohorts, }\FloatTok{0.02}\NormalTok{) ->}\StringTok{ }\NormalTok{mutationsData}
\end{Highlighting}
\end{Shaded}

\begin{verbatim}
Using 'bcr_patient_barcode' as value column. Use 'value.var' to override
\end{verbatim}

\begin{Shaded}
\begin{Highlighting}[]
\NormalTok{mutationsData[}\DecValTok{1}\NormalTok{:}\DecValTok{6}\NormalTok{, }\KeywordTok{c}\NormalTok{(}\DecValTok{1}\NormalTok{,}\DecValTok{4}\NormalTok{,}\DecValTok{56}\NormalTok{,}\DecValTok{100}\NormalTok{,}\DecValTok{207}\NormalTok{,}\DecValTok{801}\NormalTok{)]}
\end{Highlighting}
\end{Shaded}

\begin{verbatim}
  bcr_patient_barcode A2ML1 ALMS1 ATP2B2 CNTNAP4 PLEC
1     TCGA-02-0003-01     0     1      0       0    0
2     TCGA-02-0033-01     0     0      0       0    0
3     TCGA-02-0047-01     0     0      0       0    0
4     TCGA-02-0055-01     0     0      0       0    0
5     TCGA-02-2470-01     0     0      0       0    0
6     TCGA-02-2483-01     0     0      1       0    0
\end{verbatim}

gdzie wybrano jedynie te geny, których mutacja dotyczyła co najmniej 2
\% pacjentów mających zarówno dane kliniczne jak i dane o występujących
mutacjach w genach.

Dla tak otrzymanych dwóch zbiorów danych połączono dla pacjentów
informacje kliniczne z informacjami o mutacjach dzięki przypisanym do
pacjentów i ich próbek kodów \texttt{bcr\_patient\_barcode}, by
ostatecznie podzielić zbiór pacjentów na 100 losowo utworzonych grup.

\begin{Shaded}
\begin{Highlighting}[]
\KeywordTok{set.seed}\NormalTok{(}\DecValTok{4561}\NormalTok{)}
\KeywordTok{prepareCoxDataSplit}\NormalTok{(mutationsData,survivalData, }\DataTypeTok{groups =} \DecValTok{100}\NormalTok{) ->}\StringTok{ }\NormalTok{coxData_split}
\KeywordTok{head}\NormalTok{(coxData_split[[}\DecValTok{1}\NormalTok{]][}\KeywordTok{c}\NormalTok{(}\DecValTok{1}\NormalTok{,}\DecValTok{10}\NormalTok{), }\KeywordTok{c}\NormalTok{(}\DecValTok{210}\NormalTok{,}\DecValTok{302}\NormalTok{,}\DecValTok{356}\NormalTok{,}\DecValTok{898}\NormalTok{,}\DecValTok{911}\NormalTok{,}\DecValTok{1092}\NormalTok{:}\DecValTok{1093}\NormalTok{)])}
\end{Highlighting}
\end{Shaded}

\begin{verbatim}
     COL14A1 DOCK9 FASN SEMA5A SHPRH patient.vital_status times
81         0     0    0      0     0                    1     7
1068       1     0    0      1     0                    1  1171
\end{verbatim}

Niezbędną formułę modelu potrzebną do sprezycowania, które geny (a
pozostało ich 1091) należy uwzględnić w modelu uzyskano dzięki
pomocniczej funkcji

\begin{Shaded}
\begin{Highlighting}[]
\KeywordTok{prepareForumlaSGD}\NormalTok{(coxData_split) ->}\StringTok{ }\NormalTok{formulaSGD}
\end{Highlighting}
\end{Shaded}

Ostatecznie dla 6085 pacjentów, którzy posiadali informacje o
występujących mutacjach, oraz dla których odnotowano komplet i
poprawność danych klinicznych dotyczących statusu i obserwowanego czasu
przeżycia wyliczono wspólczynniki modelu proporcjonalnych hazardów Coxa
z wykorzystaniem stochastycznego spadku gradientu do estymacji. Model
dopasowano wielokrotnie z różnymi ciągami odpowiadającymi za długość
kroku algorytmu, dodatkowo badano różną ilość epok w algorytmie. Dla tak
powstałych kilku modeli wybrano ten, który dla swoich współczynników
dawał największą wartość funkcji częściowej log-wiarogodności dla
niewykorzystanej do uczenia próbki, zawierającej 2 ostatnie
zaobserwowane podzbiory obserwacji. Dla każdego z ciągów
\(1/t, 1/50*\sqrt(t), 100/5*\sqrt(100)\) odpowiadających długościom
kroków w algorytmie wyznaczono współczynniki modelu dla 5 epok, dzięki
czemu możliwe było rozważanie postępu danego wariantu algorytmu również
po 1, 2, 3 czy 4 epokach.

\begin{Shaded}
\begin{Highlighting}[]
\NormalTok{coxData_split[}\DecValTok{99}\NormalTok{:}\DecValTok{100}\NormalTok{] ->}\StringTok{ }\NormalTok{testCox}
\NormalTok{coxData_split[}\DecValTok{1}\NormalTok{:}\DecValTok{98}\NormalTok{] ->}\StringTok{ }\NormalTok{trainCox}
\KeywordTok{coxphSGD}\NormalTok{(formulaSGD, }\DataTypeTok{data =} \NormalTok{trainCox, }\DataTypeTok{max.iter =} \DecValTok{490}\NormalTok{) ->}\StringTok{ }\NormalTok{model_1_over_t}
\KeywordTok{coxphSGD}\NormalTok{(formulaSGD, }\DataTypeTok{data =} \NormalTok{trainCox, }\DataTypeTok{max.iter =} \DecValTok{490}\NormalTok{,}
         \DataTypeTok{learningRates =} \NormalTok{function(t)\{}\DecValTok{1}\NormalTok{/(}\DecValTok{50}\NormalTok{*}\KeywordTok{sqrt}\NormalTok{(t))\}) ->}\StringTok{ }\NormalTok{model_1_over_50sqrt_t}
\KeywordTok{coxphSGD}\NormalTok{(formulaSGD, }\DataTypeTok{data =} \NormalTok{trainCox, }\DataTypeTok{max.iter =} \DecValTok{490}\NormalTok{,}
         \DataTypeTok{learningRates =} \NormalTok{function(t)\{}\DecValTok{1}\NormalTok{/(}\DecValTok{100}\NormalTok{*}\KeywordTok{sqrt}\NormalTok{(t))\}) ->}\StringTok{ }\NormalTok{model_1_over_100sqrt_t}
\end{Highlighting}
\end{Shaded}

Niemożliwe było sprawdzenie założeń modelu dotyczących proporcjonalności
hazardu, gdyż zakładano napływającą postać danych (stąd podział danych
na 100 grup). Dla takiej postaci pojawiania się danych ciężko także
mówić o jakiejkolwiek diagnostyce poprawności dopasowania modelu i
dokładności otrzymanych wpsółczynników. Nie stworzono teorii
pozwalającej badać istotność statystyczną otrzymanych współczynników w
modelu, jednak założono, że współczynniki dostatecznie odległe od \(0\)
można uznać za istotnie wpływające na czas życia pacjenta. Współczynniki
dodatnie oznaczają zwiększenie hazardu pacjenta posiadającego mutację w
danym genie w stosunku do pacjentów nie posiadających mutacji w danym
genie. Współczynniki ujemne oznaczają zmniejszenie hazardu pacjenta
posiadającego mutację w danym genie w stosunku do pacjentów nie
posiadających mutacji w danym genie. Wzrost proporcji hazardu można
otrzymać dla danego genu poprzez obłożenie współczynnika funkcją
wykładniczą o wykładniku \(e\).

Wyniki estymacji dla genów zawierających największe co do modułu
współczynniki można znaleźć w Tabeli 1.

\newpage

\begin{Shaded}
\begin{Highlighting}[]
\NormalTok{coxphSGD <-}\StringTok{ }\NormalTok{function(formula, data, }\DataTypeTok{learningRates =} \NormalTok{function(x)\{}\DecValTok{1}\NormalTok{/x\},}
                    \DataTypeTok{beta_0 =} \DecValTok{0}\NormalTok{, }\DataTypeTok{epsilon =} \FloatTok{1e-5}\NormalTok{, }\DataTypeTok{max.iter =} \DecValTok{500} \NormalTok{) \{}
  \KeywordTok{checkArguments}\NormalTok{(formula, data, learningRates,}
                  \NormalTok{beta_0, epsilon) ->}\StringTok{ }\NormalTok{beta_start }\CommentTok{# check arguments}
  \NormalTok{n <-}\StringTok{ }\KeywordTok{length}\NormalTok{(data)}
  \NormalTok{diff <-}\StringTok{ }\NormalTok{epsilon +}\StringTok{ }\DecValTok{1}
  \NormalTok{i <-}\StringTok{ }\DecValTok{1}
  \NormalTok{beta_new <-}\StringTok{ }\KeywordTok{list}\NormalTok{()     }\CommentTok{# steps are saved in a list so that they can}
  \NormalTok{beta_old <-}\StringTok{ }\NormalTok{beta_start }\CommentTok{# be traced in the future}
  \CommentTok{# estimate}
  \NormalTok{while(i <=}\StringTok{ }\NormalTok{max.iter &}\StringTok{ }\NormalTok{diff >}\StringTok{ }\NormalTok{epsilon) \{}
    \NormalTok{beta_new[[i]] <-}\StringTok{ }\KeywordTok{coxphSGD_batch}\NormalTok{(}\DataTypeTok{formula =} \NormalTok{formula, }\DataTypeTok{beta =} \NormalTok{beta_old,}
        \DataTypeTok{learningRate =} \KeywordTok{learningRates}\NormalTok{(i), }\DataTypeTok{data =} \NormalTok{data[[}\KeywordTok{ifelse}\NormalTok{(i%%n==}\DecValTok{0}\NormalTok{,n,i%%n)]]) %>%}
\StringTok{      }\NormalTok{unlist}
    \NormalTok{diff <-}\StringTok{ }\KeywordTok{sqrt}\NormalTok{(}\KeywordTok{sum}\NormalTok{((beta_new[[i]] -}\StringTok{ }\NormalTok{beta_old)^}\DecValTok{2}\NormalTok{))}
    \NormalTok{beta_old <-}\StringTok{ }\NormalTok{beta_new[[i]]}
    \NormalTok{i <-}\StringTok{ }\NormalTok{i +}\StringTok{ }\DecValTok{1}  \NormalTok{; }\KeywordTok{cat}\NormalTok{(}\StringTok{"}\CharTok{\textbackslash{}r}\StringTok{ iteration: "}\NormalTok{, i, }\StringTok{"}\CharTok{\textbackslash{}r}\StringTok{"}\NormalTok{)}
  \NormalTok{\}  }\CommentTok{# return results}
  \KeywordTok{list}\NormalTok{(}\DataTypeTok{Call =} \KeywordTok{match.call}\NormalTok{(), }\DataTypeTok{epsilon =} \NormalTok{epsilon, }\DataTypeTok{learningRates =} \NormalTok{learningRates,}
       \DataTypeTok{steps =} \NormalTok{i, }\DataTypeTok{coefficients =} \KeywordTok{c}\NormalTok{(}\KeywordTok{list}\NormalTok{(beta_start), beta_new))}
\NormalTok{\}}

\NormalTok{coxphSGD_batch <-}\StringTok{ }\NormalTok{function(formula, data, learningRate, beta)\{}
  \CommentTok{# collect times, status, variables and reorder samples }
  \CommentTok{# to make the algorithm more clear to read and track}
  \NormalTok{batchData <-}\StringTok{ }\KeywordTok{prepareBatch}\NormalTok{(}\DataTypeTok{formula =} \NormalTok{formula, }\DataTypeTok{data =} \NormalTok{data)}
  \CommentTok{# calculate the log-likelihood for this batch sample}
  \NormalTok{partial_sum <-}\StringTok{ }\KeywordTok{list}\NormalTok{()}
  \KeywordTok{foreach}\NormalTok{(}\DataTypeTok{k =} \DecValTok{1}\NormalTok{:}\KeywordTok{nrow}\NormalTok{(batchData)) %do%}\StringTok{ }\NormalTok{\{}
    \CommentTok{# risk set for current time/observation}
    \NormalTok{risk_set <-}\StringTok{ }\NormalTok{batchData %>%}\StringTok{ }\KeywordTok{filter}\NormalTok{(times >=}\StringTok{ }\NormalTok{batchData$times[k])}
    
    \NormalTok{nominator <-}\StringTok{ }\KeywordTok{apply}\NormalTok{(risk_set[, -}\KeywordTok{c}\NormalTok{(}\DecValTok{1}\NormalTok{,}\DecValTok{2}\NormalTok{)], }\DataTypeTok{MARGIN =} \DecValTok{1}\NormalTok{, function(element)\{}
      \NormalTok{element *}\StringTok{ }\KeywordTok{exp}\NormalTok{(element *}\StringTok{ }\NormalTok{beta)}
    \NormalTok{\}) %>%}\StringTok{ }\KeywordTok{rowSums}\NormalTok{()}
      
    \NormalTok{denominator <-}\StringTok{ }\KeywordTok{apply}\NormalTok{(risk_set[, -}\KeywordTok{c}\NormalTok{(}\DecValTok{1}\NormalTok{,}\DecValTok{2}\NormalTok{)], }\DataTypeTok{MARGIN =} \DecValTok{1}\NormalTok{, function(element)\{}
      \KeywordTok{exp}\NormalTok{(element *}\StringTok{ }\NormalTok{beta)}
    \NormalTok{\}) %>%}\StringTok{ }\KeywordTok{rowSums}\NormalTok{()}
      
    \NormalTok{partial_sum[[k]] <-}\StringTok{ }
\StringTok{      }\NormalTok{batchData[k, }\StringTok{"event"}\NormalTok{] *}\StringTok{ }\NormalTok{(batchData[k, -}\KeywordTok{c}\NormalTok{(}\DecValTok{1}\NormalTok{,}\DecValTok{2}\NormalTok{)] -}\StringTok{ }\NormalTok{nominator/denominator)}
  \NormalTok{\}}
  \KeywordTok{do.call}\NormalTok{(rbind, partial_sum) %>%}
\StringTok{    }\KeywordTok{colSums}\NormalTok{() ->}\StringTok{ }\NormalTok{U_batch}
  
  \KeywordTok{return}\NormalTok{(beta +}\StringTok{ }\NormalTok{learningRate *}\StringTok{ }\NormalTok{U_batch)}
\NormalTok{\}}
\end{Highlighting}
\end{Shaded}

\end{document}
