\documentclass[]{article}
\usepackage{lmodern}
\usepackage{amssymb,amsmath}
\usepackage{ifxetex,ifluatex}
\usepackage{fixltx2e} % provides \textsubscript
\ifnum 0\ifxetex 1\fi\ifluatex 1\fi=0 % if pdftex
  \usepackage[T1]{fontenc}
  \usepackage[utf8]{inputenc}
\else % if luatex or xelatex
  \ifxetex
    \usepackage{mathspec}
    \usepackage{xltxtra,xunicode}
  \else
    \usepackage{fontspec}
  \fi
  \defaultfontfeatures{Mapping=tex-text,Scale=MatchLowercase}
  \newcommand{\euro}{€}
\fi
% use upquote if available, for straight quotes in verbatim environments
\IfFileExists{upquote.sty}{\usepackage{upquote}}{}
% use microtype if available
\IfFileExists{microtype.sty}{%
\usepackage{microtype}
\UseMicrotypeSet[protrusion]{basicmath} % disable protrusion for tt fonts
}{}
\usepackage[margin=1in]{geometry}
\usepackage{color}
\usepackage{fancyvrb}
\newcommand{\VerbBar}{|}
\newcommand{\VERB}{\Verb[commandchars=\\\{\}]}
\DefineVerbatimEnvironment{Highlighting}{Verbatim}{commandchars=\\\{\}}
% Add ',fontsize=\small' for more characters per line
\usepackage{framed}
\definecolor{shadecolor}{RGB}{248,248,248}
\newenvironment{Shaded}{\begin{snugshade}}{\end{snugshade}}
\newcommand{\KeywordTok}[1]{\textcolor[rgb]{0.13,0.29,0.53}{\textbf{{#1}}}}
\newcommand{\DataTypeTok}[1]{\textcolor[rgb]{0.13,0.29,0.53}{{#1}}}
\newcommand{\DecValTok}[1]{\textcolor[rgb]{0.00,0.00,0.81}{{#1}}}
\newcommand{\BaseNTok}[1]{\textcolor[rgb]{0.00,0.00,0.81}{{#1}}}
\newcommand{\FloatTok}[1]{\textcolor[rgb]{0.00,0.00,0.81}{{#1}}}
\newcommand{\CharTok}[1]{\textcolor[rgb]{0.31,0.60,0.02}{{#1}}}
\newcommand{\StringTok}[1]{\textcolor[rgb]{0.31,0.60,0.02}{{#1}}}
\newcommand{\CommentTok}[1]{\textcolor[rgb]{0.56,0.35,0.01}{\textit{{#1}}}}
\newcommand{\OtherTok}[1]{\textcolor[rgb]{0.56,0.35,0.01}{{#1}}}
\newcommand{\AlertTok}[1]{\textcolor[rgb]{0.94,0.16,0.16}{{#1}}}
\newcommand{\FunctionTok}[1]{\textcolor[rgb]{0.00,0.00,0.00}{{#1}}}
\newcommand{\RegionMarkerTok}[1]{{#1}}
\newcommand{\ErrorTok}[1]{\textbf{{#1}}}
\newcommand{\NormalTok}[1]{{#1}}
\ifxetex
  \usepackage[setpagesize=false, % page size defined by xetex
              unicode=false, % unicode breaks when used with xetex
              xetex]{hyperref}
\else
  \usepackage[unicode=true]{hyperref}
\fi
\hypersetup{breaklinks=true,
            bookmarks=true,
            pdfauthor={Marcin Kosinski},
            pdftitle={Untitled},
            colorlinks=true,
            citecolor=blue,
            urlcolor=blue,
            linkcolor=magenta,
            pdfborder={0 0 0}}
\urlstyle{same}  % don't use monospace font for urls
\setlength{\parindent}{0pt}
\setlength{\parskip}{6pt plus 2pt minus 1pt}
\setlength{\emergencystretch}{3em}  % prevent overfull lines
\setcounter{secnumdepth}{0}

%%% Use protect on footnotes to avoid problems with footnotes in titles
\let\rmarkdownfootnote\footnote%
\def\footnote{\protect\rmarkdownfootnote}

%%% Change title format to be more compact
\usepackage{titling}

% Create subtitle command for use in maketitle
\newcommand{\subtitle}[1]{
  \posttitle{
    \begin{center}\large#1\end{center}
    }
}

\setlength{\droptitle}{-2em}
  \title{Untitled}
  \pretitle{\vspace{\droptitle}\centering\huge}
  \posttitle{\par}
  \author{Marcin Kosinski}
  \preauthor{\centering\large\emph}
  \postauthor{\par}
  \predate{\centering\large\emph}
  \postdate{\par}
  \date{25.10.2015}



\begin{document}

\maketitle


dupa sraka \texttt{logitGD()} jasd

\begin{Shaded}
\begin{Highlighting}[]
\NormalTok{logitGD <-}\StringTok{ }\NormalTok{function(y, x, }\DataTypeTok{optim.method =} \StringTok{"GDI"}\NormalTok{, }\DataTypeTok{eps =} \FloatTok{10e-4}\NormalTok{,}
                    \DataTypeTok{max.iter =} \DecValTok{100}\NormalTok{, }\DataTypeTok{alpha =} \NormalTok{function(t)\{}\DecValTok{1}\NormalTok{/t\}, }\DataTypeTok{beta_0 =} \KeywordTok{c}\NormalTok{(}\DecValTok{0}\NormalTok{,}\DecValTok{0}\NormalTok{))\{}
  \KeywordTok{stopifnot}\NormalTok{(}\KeywordTok{length}\NormalTok{(y) ==}\StringTok{ }\KeywordTok{length}\NormalTok{(x) &}\StringTok{ }\NormalTok{optim.method %in%}\StringTok{ }\KeywordTok{c}\NormalTok{(}\StringTok{"GDI"}\NormalTok{, }\StringTok{"GDII"}\NormalTok{, }\StringTok{"SGDI"}\NormalTok{)}
            \NormalTok{&}\StringTok{ }\KeywordTok{is.numeric}\NormalTok{(}\KeywordTok{c}\NormalTok{(max.iter, eps, x)) &}\StringTok{ }\KeywordTok{all}\NormalTok{(}\KeywordTok{c}\NormalTok{(eps, max.iter) >}\StringTok{ }\DecValTok{0}\NormalTok{) &}
\StringTok{              }\KeywordTok{is.function}\NormalTok{(alpha))}
  \NormalTok{iter <-}\StringTok{ }\DecValTok{0}
  \NormalTok{err <-}\StringTok{ }\KeywordTok{list}\NormalTok{()}
  \NormalTok{err[[iter}\DecValTok{+1}\NormalTok{]] <-}\StringTok{ }\NormalTok{eps}\DecValTok{+1}
  \NormalTok{w_old <-}\StringTok{ }\NormalTok{beta_0}

  \NormalTok{res <-}\KeywordTok{list}\NormalTok{()}
  \NormalTok{while(iter <}\StringTok{ }\NormalTok{max.iter &&}\StringTok{ }\NormalTok{(}\KeywordTok{abs}\NormalTok{(err[[}\KeywordTok{ifelse}\NormalTok{(iter==}\DecValTok{0}\NormalTok{,}\DecValTok{1}\NormalTok{,iter)]]) >}\StringTok{ }\NormalTok{eps))\{}

    \NormalTok{iter <-}\StringTok{ }\NormalTok{iter +}\StringTok{ }\DecValTok{1}
    \NormalTok{if (optim.method ==}\StringTok{ "GDI"}\NormalTok{)\{}
      \NormalTok{w_new <-}\StringTok{ }\NormalTok{w_old +}\StringTok{ }\KeywordTok{alpha}\NormalTok{(iter)*}\KeywordTok{updateWeightsGDI}\NormalTok{(y, x, w_old)}
    \NormalTok{\}}
    \NormalTok{if (optim.method ==}\StringTok{ "GDII"}\NormalTok{)\{}
      \NormalTok{w_new <-}\StringTok{ }\NormalTok{w_old -}\StringTok{ }\KeywordTok{as.vector}\NormalTok{(}\KeywordTok{inverseHessianGDII}\NormalTok{(x, w_old)%*%}
\StringTok{                                   }\KeywordTok{updateWeightsGDI}\NormalTok{(y, x, w_old))}
    \NormalTok{\}}
    \NormalTok{if (optim.method ==}\StringTok{ "SGDI"}\NormalTok{)\{}
      \NormalTok{w_new <-}\StringTok{ }\NormalTok{w_old +}\StringTok{ }\KeywordTok{alpha}\NormalTok{(iter)*}\KeywordTok{updateWeightsSGDI}\NormalTok{(y[iter], x[iter], w_old)}
    \NormalTok{\}}
    \NormalTok{res[[iter]] <-}\StringTok{ }\NormalTok{w_new}
    \NormalTok{err[[iter]] <-}\StringTok{ }\KeywordTok{sqrt}\NormalTok{(}\KeywordTok{sum}\NormalTok{((w_new -}\StringTok{ }\NormalTok{w_old)^}\DecValTok{2}\NormalTok{))}

    \NormalTok{w_old <-}\StringTok{ }\NormalTok{w_new}

  \NormalTok{\}}
  \KeywordTok{return}\NormalTok{(}\KeywordTok{list}\NormalTok{(}\DataTypeTok{steps =} \KeywordTok{c}\NormalTok{(}\KeywordTok{list}\NormalTok{(beta_0),res), }\DataTypeTok{errors =} \KeywordTok{c}\NormalTok{(}\KeywordTok{list}\NormalTok{(}\KeywordTok{c}\NormalTok{(}\DecValTok{0}\NormalTok{,}\DecValTok{0}\NormalTok{)),err)))}
\NormalTok{\}}

\NormalTok{updateWeightsGDI <-}\StringTok{ }\NormalTok{function(y, x, w_old)\{}
  \NormalTok{(}\DecValTok{1}\NormalTok{/}\KeywordTok{length}\NormalTok{(y))*}\KeywordTok{c}\NormalTok{(}\KeywordTok{sum}\NormalTok{(y-}\KeywordTok{p}\NormalTok{(w_old, x)), }\KeywordTok{sum}\NormalTok{(x*(y-}\KeywordTok{p}\NormalTok{(w_old, x))))}
\NormalTok{\}}

\NormalTok{updateWeightsSGDI <-}\StringTok{ }\NormalTok{function(y_i, x_i, w_old)\{}
  \KeywordTok{c}\NormalTok{(y_i-}\KeywordTok{p}\NormalTok{(w_old, x_i), x_i*(y_i-}\KeywordTok{p}\NormalTok{(w_old, x_i)))}
\NormalTok{\}}

\NormalTok{p <-}\StringTok{ }\NormalTok{function(w_old, x_i)\{}
  \DecValTok{1}\NormalTok{/(}\DecValTok{1}\NormalTok{+}\KeywordTok{exp}\NormalTok{(-w_old[}\DecValTok{1}\NormalTok{]-w_old[}\DecValTok{2}\NormalTok{]*x_i))}
\NormalTok{\}}

\NormalTok{inverseHessianGDII <-}\StringTok{ }\NormalTok{function(x, w_old)\{}
  \KeywordTok{solve}\NormalTok{(}
    \KeywordTok{matrix}\NormalTok{(}\KeywordTok{c}\NormalTok{(}
      \KeywordTok{sum}\NormalTok{(}\KeywordTok{p}\NormalTok{(w_old, x)*(}\DecValTok{1}\NormalTok{-}\KeywordTok{p}\NormalTok{(w_old, x))),}
      \KeywordTok{sum}\NormalTok{(x*}\KeywordTok{p}\NormalTok{(w_old, x)*(}\DecValTok{1}\NormalTok{-}\KeywordTok{p}\NormalTok{(w_old, x))),}
      \KeywordTok{sum}\NormalTok{(x*}\KeywordTok{p}\NormalTok{(w_old, x)*(}\DecValTok{1}\NormalTok{-}\KeywordTok{p}\NormalTok{(w_old, x))),}
      \KeywordTok{sum}\NormalTok{(x*x*}\KeywordTok{p}\NormalTok{(w_old, x)*(}\DecValTok{1}\NormalTok{-}\KeywordTok{p}\NormalTok{(w_old, x)))}
    \NormalTok{),}
    \DataTypeTok{nrow =}\DecValTok{2} \NormalTok{)}
  \NormalTok{)}
\NormalTok{\}}
\end{Highlighting}
\end{Shaded}

\begin{Shaded}
\begin{Highlighting}[]
                    \CommentTok{# wstępna inicjalizacja parametrów}
\NormalTok{eps =}\StringTok{ }\FloatTok{1e-5}                               \CommentTok{# warunek stopu.}

\NormalTok{n =}\StringTok{ }\KeywordTok{length}\NormalTok{(data)                         }\CommentTok{# data jest listą ramek danych.}

\NormalTok{diff =}\StringTok{ }\NormalTok{eps +}\StringTok{ }\DecValTok{1}                           \CommentTok{# różnice w oszacowaniach parametrów}
                                         \CommentTok{# między kolejnymi krokami.}

\NormalTok{learningRates =}\StringTok{ }\NormalTok{function(x) }\DecValTok{1}\NormalTok{/x          }\CommentTok{# długości kroku algorytmu.}

\NormalTok{beta_old =}\StringTok{ }\KeywordTok{numeric}\NormalTok{(}\DecValTok{0}\NormalTok{, }\DataTypeTok{length =} \NormalTok{k)        }\CommentTok{# punkt startowy dlugosci k,}
                                         \CommentTok{# gdzie k to liczba zmiennych}
                                         \CommentTok{# objaśniających w modelu.}

                              \CommentTok{# estymacja}
\NormalTok{while(i <}\StringTok{ }\NormalTok{n |}\StringTok{ }\NormalTok{diff <}\StringTok{ }\NormalTok{eps) do             }\CommentTok{# do zbieżności lub wyczerpania zbiorów}
  
  \NormalTok{batch =}\StringTok{ }\NormalTok{data[[i]]}
  
  \NormalTok{beta_new =}\StringTok{ }\NormalTok{beta_old -}\StringTok{ }\KeywordTok{learningRates}\NormalTok{(i)*}\KeywordTok{U_Batch}\NormalTok{(batch) }
                                         \CommentTok{# U_Batch to częściowa funkcja}
                                         \CommentTok{# log-wiarogdności dla zaobserwowanego}
                                         \CommentTok{# zbioru `batch`}
  
  \NormalTok{diff =}\StringTok{ }\KeywordTok{euclidean_dist}\NormalTok{(beta_new, beta_old) }\CommentTok{# odległość euklidesowa}
  
  \NormalTok{beta_old =}\StringTok{ }\NormalTok{beta_new }
  
  \NormalTok{i =}\StringTok{ }\NormalTok{i +}\StringTok{ }\DecValTok{1}
  
\NormalTok{end while}

\NormalTok{return beta_new}
\end{Highlighting}
\end{Shaded}

\end{document}
