\chapter*{Podsumowanie}
Praca przedstawia zastosowanie algorytmu numerycznej optymalizacji metodą stochastycznego spadku gradientu do wyznaczenia współczynników w modelu proporcjonalnych hazardów Coxa, opartych na analitycznej metodzie wyznaczania estymatorów metodą największej wiarogodności. Zaprezentowano zastosowanie tego algorytmu do napływających danych, wyliczając kolejne kroki optymalizacji w algorytmie w oparciu o częściową funkcję log-wiarogodności, konstruowaną na bazie obecnie zaobserwowanego podzbioru obserwacji. Zaprezentowane podejście jest nowatorskie. Wcześniej nie wykorzystywano tego podejścia w modelu Coxa.

Opisano matematyczne własności estymatorów największej wiarogodności, przedstawiono model Coxa wraz z najważniejszymi jego cechami, scharakteryzowano algorytm stochastycznego spadku gradientu oraz jego różnice w stosunku go algorytmów spadku gradientu, zaimplementowano algorytm numerycznej optymalizacji metodą stochastycznego spadku gradientu do wyznaczenia współczynników w modelu proporcjonalnych hazardów Coxa oraz przeprowadzono analizę rzeczywistych danych genomicznych, wykorzystując stworzony algorytm. Pracę wzbogacono o symulacje zbieżności procesu optymalizacji funkcji log-wiarogodności w~modelu regresji logistycznej, które przygotowane zostały dla algorytmów spadku gradientu rzędu I, spadku gradientu rzędu II oraz stochastycznego spadku gradientu rzędu I. Dzięki temu możliwe było zobrazowanie różnic w opisywanych metodach numerycznej optymalizacji. Dodatkowo przeprowadzono symulacje przy sztucznie wygenerowanych danych do analizy przeżycia z rozkładu Weibulla, pozwalające zobrazować proces zbieżności nowo wprowadzonego algorytmu numerycznej optymalizacji metodą stochastycznego spadku gradientu do wyznaczenia współczynników w modelu proporcjonalnych  hazardów Coxa.

Proces estymacji w modelu proporcjonalnych hazardów Coxa, z wykorzystaniem metody stochastycznego spadku gradientu, w sytuacji napływających danych wygląda na stabilny i zbiega w okolice bliskie do teoretycznego optimum, niekiedy nawet już po jednej epoce. Przy dobrze dobranych parametrach optymalizacji, proces może być z powodzeniem wykorzystywany do znajdowania współczynników modelu. Kluczowym momentem, mającym wpływ na powodzenie optymalizacji, jest wybór ciągu odpowiedzialnego za dobór długości kroków, więc zaleca się symulacje badające różne ciągi oraz wybranie tego ciągu, który w większości przypadków daje stabilne, jednorodne współczynniki, maksymalizujące częściową funkcję log-wiarogodności, konstruowaną w oparciu o zbiór testowy. 

Do analizy genomicznej wykorzystano środowisko statystyczne $\mathcal{R}$, a wykorzystane funkcje, zaimplementowane algorytmy i przygotowaną dokumentację do stworzonego w trakcie pracy pakietu \texttt{coxphSGD} zamieszczono w Dodatku \ref{docCoxSGD}. W analizie badano wpływ występowania mutacji w poszczególnych genach na czas przeżycia pacjentów, dla których dane kliniczne i dane o występowaniu mutacji zaczerpnięto z badania \textit{The Cancer Genome Atlas}. Zaprezentowano skuteczność badanego modelu oraz wyniki 40 genów o największych, co do modułu, współczynnikach. Dzięki temu wskazano geny odpowiedzialne za zwiększenie ryzyka zgonu w~trakcie choroby nowotworowej.