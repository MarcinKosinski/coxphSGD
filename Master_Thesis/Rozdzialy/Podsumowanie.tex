\chapter*{Podsumowanie}
W pracy przedstawiono zastosowanie algorytmu numerycznej optymalizacji metodą stochastycznego spadku gradientu do wyznaczenia współczynników w modelu proporcjonalnych Coxa opartych na analitycznej metodzie wyznaczania estymatorów metodą największej wiarogodności. Zaprezentowano zastosowanie tego algorytmu do napływających danych wyliczając kolejne kroki optymalizacji w algorytmie w oparciu o częściową funkcję log-wiarogodności konstruowaną dla obecnie zaobserwowanego podzbioru obserwacji.

W pracy opisano matematyczne własności estymatorów największej wiarogodności, przedstawiono model Coxa wraz z najważniejszymi jego własnościami, scharakteryzowano algorytm stochastycznego spadku gradientu oraz jego różnice w stosunku go algorytmów spadku gradientu, zaimplementowano algorytm numerycznej optymalizacji metodą stochastycznego spadku gradientu do wyznaczenia współczynników w modelu proporcjonalnych Coxa opartych na analitycznej metodzie wyznaczania estymatorów metodą największej wiarogodności oraz przeprowadzono analizę rzeczywistych danych genomicznych z wykorzystaniem tego algorytmu. Pracę wzbogacono o symulacje zbieżności procesu optymalizacji funkcji log-wiarogodności dla modelu regresji logistycznej przygotowane dla algorytmów spadku gradientu rzędu I, spadku gradientu rzędu II oraz stochastycznego spadku gradientu rzędu I, dzięki którym możliwe było zobrazowanie różnic w tych metodach. Dodatkowo przeprowadzono symulacje dla sztucznie wygenerowanych danych do analizy przeżycia z rozkładu Weibulla obrazujące proces zbieżności dla nowo wprowadzonego algorytmu numerycznej optymalizacji metodą stochastycznego spadku gradientu do wyznaczenia współczynników w modelu proporcjonalnych Coxa opartych na analitycznej metodzie wyznaczania estymatorów metodą największej wiarogodności.

Do analizy genomicznej wykorzystano środowisko statystyczne $\mathcal{R}$, a wykorzystane funkcje, zaimplementowane algorytmy i przygotowaną dokumentację do stworzonego pakietu \texttt{coxphSGD} można znaleźć w Dodatku \ref{docCoxSGD}.