\chapter*{Podsumowanie}
Praca przedstawia zastosowanie algorytmu numerycznej optymalizacji metodą stochastycznego spadku gradientu do wyznaczenia współczynników w modelu proporcjonalnych Coxa, opartych na analitycznej metodzie wyznaczania estymatorów metodą największej wiarogodności. Zaprezentowano zastosowanie tego algorytmu do napływających danych wyliczając kolejne kroki optymalizacji w algorytmie w oparciu o częściową funkcję log-wiarogodności konstruowaną dla obecnie zaobserwowanego podzbioru obserwacji. Zaprezentowane podejście jest nowatorskie.

W pracy opisano matematyczne własności estymatorów największej wiarogodności, przedstawiono model Coxa wraz z najważniejszymi jego własnościami, scharakteryzowano algorytm stochastycznego spadku gradientu oraz jego różnice w stosunku go algorytmów spadku gradientu, zaimplementowano algorytm numerycznej optymalizacji metodą stochastycznego spadku gradientu do wyznaczenia współczynników w modelu proporcjonalnych Coxa, opartych na analitycznej metodzie wyznaczania estymatorów metodą największej wiarogodności oraz przeprowadzono analizę rzeczywistych danych genomicznych z wykorzystaniem tego algorytmu. Pracę wzbogacono o symulacje zbieżności procesu optymalizacji funkcji log-wiarogodności dla modelu regresji logistycznej, które przygotowane zostały dla algorytmów spadku gradientu rzędu I, spadku gradientu rzędu II oraz stochastycznego spadku gradientu rzędu I. Dzięki temu możliwe było zobrazowanie różnic w opisywanych metodach numerycznej optymalizacji. Dodatkowo przeprowadzono symulacje dla sztucznie wygenerowanych danych do analizy przeżycia z rozkładu Weibulla, pozwalające zobrazować proces zbieżności dla nowo wprowadzonego algorytmu numerycznej optymalizacji metodą stochastycznego spadku gradientu do wyznaczenia współczynników w modelu proporcjonalnych Coxa, opartych na analitycznej metodzie wyznaczania estymatorów metodą największej wiarogodności.

Proces estymacji w modelu proporcjonalnych hazardów Coxa z wykorzystaniem metody stochastycznego spadku gradientu w sytuacji napływających danych wygląda na stabilny i zbiega w okolice bliskie do teoretycznego optimum, niekiedy nawet dla tylko jednej epoki. Dla dobrze dobranych parametrów optymalizacji proces może być z powodzeniem wykorzystywany do znajdowania współczynników modelu. Kluczowym momentem mającym wpływ na powodzenie optymalizacji jest wybór ciągu odpowiedzialnego za dobór długości kroków w algorytmie, więc zaleca się symulacje badające różne ciągi oraz wybranie tego ciągu, który w większości przypadków daje stabilne, jednorodne współczynniki. 

Do analizy genomicznej wykorzystano środowisko statystyczne $\mathcal{R}$, a wykorzystane funkcje, zaimplementowane algorytmy i przygotowaną dokumentację do stworzonego w trakcie pracy pakietu \texttt{coxphSGD} zamieszczono w Dodatku \ref{docCoxSGD}. W analizie badano wpływ występowania mutacji w poszczególnych genach na czas przeżycia pacjentów, dla których dane kliniczne i dane o występowaniu mutacji zaczerpnięto z badania \textit{The Cancer Genome Atlas}.