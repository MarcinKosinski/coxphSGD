\chapter{Analiza danych genomicznych}

\newpage
\section{Genetyczne podstawy nowotworzenia}
Choroby nowotworowe stanowią drugą, po chorobach serca, przyczynę zachorowań i zgonów na całym świecie. Wyniki opracowane przez Polską Unię Onkologiczna wskazują na tendencję wzrostową liczby zachorowań na nowotwory i rokują na utrzymanie się jej do 2020~\cite{zikula}.

Według współczesnej definicji \textbf{nowotwór} jest chorobą cyklu komórkowego i oznacza \cite{zikula2} \\ \ \\ \textit{nieprawidłową tkankę, która powstała z jednej komórki i rośnie jako następstwo zaburzeń dynamizmu i prawidłowego przebiegu cyklu komórkowego oraz zaburzeń różnicowania się komórki i komunikacji wewnątrzkomórkowej, międzykomórkowej i pozakomórkowej jej klonalnego potomstwa}. \\ \ \\
Wyniki badań nad transformacją nowotworową wykazały, że nowotwory powstają jako wynik wielu nielegalnych mutacji w DNA komórki somatycznej, które poprzez kumulację wywołują utratę kontroli nad proliferacją (rozwojem), wzrostem i różnicowaniem~\cite{zikula5}.

Proces tworzenia nowotworu (\textbf{karcynogeneza}) jest wieloczynnikowy i wielostopniowy, a~zmiany w nim nasilają się w miarę pogłębiania się niestabilności genetycznej \cite{zikula2}. Transformacja nowotworowa następuje w wyniku zmian powstałych w obrębie czterech różnych kategorii genów, które wpływają na proliferację i różnicowanie komórek \cite{zikula}. Lista genów wpływających na proliferację i różnicowanie komórek z opisem \cite{zikula4} znajduje się poniżej
\begin{itemize}
\item \textit{geny regulujące naprawę uszkodzonego DNA} - mechanizmy szybkiej naprawy DNA zapobiegają przed mutacjami genów odpowiedzialnyh m. in. za proliferację i rożnicowanie się komórek. Geny biorące udział w naprawie DNA nie są onkogenne, natomiast mutacje w ich obrębie mogą ułatwić transformację nowotworową oraz podwyższają ryzyko utrwalenia się zmian w pozostałych grupach i dlatego mają podstawowe znaczenie dla integracji genomu.
\item \textit{geny supresorowe} (antyonkogeny) - geny działające hamująco na procesy proliferacji komórkowej bądź stabilizująco na procesy utrzymujące stabilność genetyczną komórki,
\item \textit{protoonkogeny} - potencjalnie zdolne do wyzwolenia procesu transformacji nowotworowej. Uwarunkowana mutacją zmiana ich ekspresji sprawia, że przekształcają się w~onkogeny, czyli geny bezpośrednio aktywujące transformację nowotworową,
\item \textit{geny regulujące apoptozę} (naturalny proces zaprogramowanej śmierci komórki w organizmach wielokomórkowych) - zahamowanie procesu apoptozy wydłuża okres przeżycia komórek, zwiększając tym samym liczebność populacji komórek narażonych na działanie karcynogenów i prawdopodobieństwo wystąpienia mutacji w komórce.
\end{itemize}
Zmiany genetyczne w komórkach zachodzą pod wpływem działania czynników mutagennych, do których można zaliczyć: promieniowanie UV (czynniki fizyczne), substancje obecne w dymie papierosowym i spalinach samochodowych, azbest i niektóre metale ciężkie (czynniki chemiczne), wirusy, toksyny bakteryjne i pasożytnicze oraz błędy podczas replikacji czy pośrednie produkty przemiany materii tj. hormony i wolne rodniki (czynniki biologiczne)~\cite{zikula3}.

Jedną z podstawowych cech komórek nowotworowych jest zdolność do endogennej produkcji sygnałów mitogennych bez udziału zewnętrznych czynników wzrostu, dlatego choroby nowotworowe są tak niebezpieczne a ich zwalczanie tak potrzebne.

\section{Opis danych z The Cancer Genome Atlas}
\section{Analiza}