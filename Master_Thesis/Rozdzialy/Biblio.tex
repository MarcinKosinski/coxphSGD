\begin{thebibliography}{99}

\bibitem[1]{aldrich1} Aldrich J., (1997), \textit{R. A. Fisher and the Making of Maximum Likelihood 1912 – 1922}, Statistical Science 1997, Vol. 12, No. 3, 162-176.

\bibitem[2]{assel} Asselain B., Mould R. F., (2010), \textit{Methodology of the Cox proportional hazards model},  Journal of Oncology 2010, volume 60, Number 5,  403–409.

\bibitem[3]{bender} Bender R., Augustin T., Blettner Maria, (2005) \textit{Generating survival times to simulate Cox proportional hazards models}, Statistics in Medicine, Volume 24, Issue 11, 1713–1723.

\bibitem[4]{biecek1} Biecek P., (2011), \textit{Przewodnik po pakiecie R}, Rozprawa doktorska, Oficyna Wydawnicza GiS, wydanie II.

\bibitem[5]{bott1} Bottou L., (2010), \textit{Large-Scale Machine Learning with Stochastic Gradient Descent}.

\bibitem[6]{bottDOD} Bottou L., (1998), \textit{Online Learning and Stochastic Approximations}.

\bibitem[7]{bott2} Bottou L., (2012), \textit{Stochastic Gradient Descent Tricks}.

\bibitem[8]{broad} Broad Institute TCGA Genome Data Analysis Center (2014): Firehose stddata 2015 06 01 run. Broad Institute of MIT and Harvard. DOI:10.7908/C1251HBG

\bibitem[9]{burzyk1} Burzykowski T., (2015?), \textit{Notatki do przedmiotu Biostatystyka}, \texttt{https://e.mini.pw.edu.pl/sites/default/files/biostatystyka.pdf}.

\bibitem[10]{cox0} Cox D. R. (1962), \textit{Renewal Theory. Methuen Monograph on Applied Probability
\& Statistics}, London: Methuen.

\bibitem[11]{cox}  Cox D. R., (1972), \textit{Regression models and life-tables (with discussion)}, Journal of the Royal Statistical Society Series B 34:187-220. 


\bibitem[12]{collett} Collett D., (1994), \textit{Modelling Survival Data in Medical Research}, Chapman and Hall:
London.

\bibitem[13]{czepiel} Czepiel S. A., \textit{Maximum Likelihood Estimation of Logistic Regression Models: Theory and Implementation}, \texttt{http://czep.net/stat/mlelr.pdf}.

\bibitem[14]{dennis} Dennis J. E. Jr., Schnabel R. B., (1983), \textit{Numerical Methods For Un-
constrained Optimization and Nonlinear Equations}, Prentice-Hall.

\bibitem[15]{dobson} Dobson A. J., (2002), \textit{An Introduction to Generalized Linear Models}, Wydanie II, Chapman \& Hall/CRC.

\bibitem[16]{zikula2} Domagała W., (2007), \textit{Molekularne podstawy karcynogenezy i ścieżki sygnałowe niektórych nowotworów ośrodkowego układu nerwowego}, Polski Przegląd Neurologiczny, tom 3: 127-141.

\bibitem[17]{finkel}  Finkel J. R., Kleeman A., Manning C. D., (2008), \textit{Efficient, Feature-based, Conditional Random Field Parsing}, Proc. Annual Meeting of the ACL.

\bibitem[18]{fisher1} Fisher R. A., (1912) \textit{An absolute criterion for fitting frequency curves}. 

\bibitem[19]{fisher2} Fisher R. A., (1922) \textit{On the mathematical foundations of theoretical statistics}, Philos. Trans. Roy. Soc. London Ser. A 222 309-368.

\bibitem[20]{fortuna} Fortuna Z., Macukow B., Wąsowski J., (2006), \textit{Metody Numeryczne}, Wydawnictwa Naukowo-Techniczne.

\bibitem[21]{gauss1} Gauss C. F., (1809,) \textit{Theoria Motus Corporum Coelestium}.

\bibitem[22]{gagol1} Gągolewski M., (2014), \textit{Programowanie w języku R}, Wydawnictwo Naukowe PWN.


\bibitem[23]{hald} Hald A., (1949), \textit{Maximum likelihood estimation of the parameters of a normal distribution which is truncated at a known point}, Skandinavisk Aktuarietidskrift, 119-134.


\bibitem[24]{glmglm} Hastie T. J., Pregibon D., (1992), \textit{Generalized linear models. Chapter 6 of Statistical Models in S}, eds J. M. Chambers and T. J. Hastie, Wadsworth \& Brooks/Cole.


\bibitem[25]{hutch1} Hutchinson J. B., (1928), \textit{The Application of the "Method of Maximum Likelihood" to the Estimation of Linkage}, Genetics. 1929 Nov; 14(6): 519–537.


\bibitem[26]{zikula3} Janik-Papis K., Błasiak J., (2010), \textit{Molekularne wyznaczniki raka piersi. Inicjacja i promocja - część I}, Nowotwory - Journal of Oncology, 60 (3): 236-247.

\bibitem[27]{zikula4} Janik-Papis K., Błasiak J., (2010), \textit{Molekularne wyznaczniki raka piersi. Progresja i nowi kandydaci - część II}, Nowotwory - Journal of Oncology, 60 (4): 341-350.

\bibitem[28]{scoring2} Jennrich R. I., Sampson P. F., (1976), \textit{Newton-Raphson and related algorithms for maximum likelihood variance component estimation}, Technometrics, 18, 11-17.


\bibitem[29]{kenward1} Kenward M. G., Lesaffre E. and Molenberghs G., (1994), \textit{An Application of Maximum Likelihood and Generalized Estimating Equations to the Analysis of Ordinal Data from a Longitudinal Study with Cases Missing at Random}, Biometrics
Vol. 50, No. 4 (Dec., 1994), pp. 945-953.


\bibitem[30]{kosa0} Kosiński M., (2015), \textit{coxphSGD: Stochastic Gradient Descent log-likelihood estimation in Cox proportional hazards model}, R package version 0.0.3,  \texttt{https://github.com/MarcinKosinski/Cox-SGD}.

\bibitem[31]{kosa1} Kosiński M., Biecek P., (2015), \texttt{RTCGA: The Cancer Genome Atlas Data Integration}, R package version 1.1.6, \texttt{https://github.com/RTCGA}.

\bibitem[32]{kosa2} Kosiński M., (2015), \textit{RTCGA.clinical: Clinical datasets from The Cancer Genome Atlas Project}, R package version 20150821.1.2, \\ \texttt{https://bioconductor.org/packages/release/data/experiment/html/RTCGA.clinical.html}

\bibitem[33]{kosa3} Kosiński M., (2015), \textit{RTCGA.mutations: Clinical datasets from The Cancer Genome Atlas Project}, R package version 20150821.1.1, \\ \texttt{https://bioconductor.org/packages/release/data/experiment/html/RTCGA.mutations.html}

\bibitem[34]{kotlowski} Kotłowski W., (2012), Notatki do przedmiotu \textit{Techniki Optymalizacji} prowadzonego na Politechnice Poznańskiej, \\ \texttt{http://www.cs.put.poznan.pl/wkotlowski/teaching/wyklad3b.pdf}

\bibitem[35]{zikula5} Kozłowska J., Łaczmańska I., (2010), \textit{Niestabilność genetyczna - jej znaczenie w procesie powstawania nowotworów oraz diagnostyka laboratoryjna}, Nowotwory - Journal of Oncology, 60 (6): 548-553.

\bibitem[36]{legendre1} Legendre A. M., (1804), \textit{Nouvelles m\`ethods pour la d\`etermination des orbites des com\`etes}.


\bibitem[37]{scoring1} Longford N. T., (1987), \textit{A fast scoring algorithm for maximum likelihood estimation in unbalanced mixed models with nested random effects}, Biometrika 74 (4): 817-827.

\bibitem[38]{milewski} Milewski S., (2006), Konspekt do przedmiotu \textit{Metody Numeryczne} prowadzonego na Politechnice Krakowskiej, \\ \texttt{http://l5.pk.edu.pl/images/skrypty/Metody\_numeryczne\_1}

\bibitem[39]{millar1} Millar R. B., (2011), \textit{Maximum Likelihood Estimation and Inference: With Examples in R, SAS and ADMB, chapter 6. Some Widely Used Applications of Maximum Likelihood}, John Wiley \& Sons, Ltd.

\bibitem[40]{MOOD} Mood A. M., Graybill F. A., Boes D. C., (1974), \textit{Introduction to the Theory of Statistics},
McGraw-Hill: New York.

\bibitem[41]{murata} Murata N., (1998), \textit{A Statistical Study of On-line Learning. In Online Learning
and Neural Networks}, Cambridge University Press.

\bibitem[42]{niemiro} Niemiro W., (2011), Skrypt do przedmiotu \textit{Statystyka} prowadzonego na Uniwersytecie Warszawskim, \\ \texttt{http://www-users.mat.umk.pl/$\sim$wniem/Statystyka/Statystyka.pdf}

\bibitem[43]{norwe} Norwegian Multicentre Study Group, (1981), \textit{Timolol-induced reduction in
mortality and reinfarction}, The New England  Journal of Medicine; 304: 801-7.


\bibitem[44]{mit0} Panchenko D., (2006), Notatki do otwartego kursu MIT \textit{Statistics for Applications, Lecture 2: Maximum Likelihood Estimators.}, \\
\texttt{http://ocw.mit.edu/courses/mathematics/18-443-statistics-for-applications-fall-2006/}

\bibitem[45]{mit1} Panchenko D., (2006), Notatki do otwartego kursu MIT \textit{Statistics for Applications, Lecture 3: Properties of MLE: consistency, asymptotic normality. Fisher information.}, \\
\texttt{http://ocw.mit.edu/courses/mathematics/18-443-statistics-for-applications-fall-2006/}

\bibitem[46]{zikula} Podsiadły K., (2011), \textit{Genetyczne podstawy nowotworzenia}, \texttt{www.e-biotechnologia.pl/Artykuly/Genetyczne-podstawy-nowotworzenia}.

\bibitem[47]{programikr} R Core Team, (2013) \textit{R: A language and environment for statistical computing.} R Foundation for Statistical Computing, Wiedeń , ISBN 3-900051-07-0, \texttt{http://www.R-project.org/}.


\bibitem[48]{robbins} Robbins H. E., Siegmund. D. O., (1971), \textit{A convergence theorem for non negative almost supermartingales and some applications}, In Proc. Sympos. Optimizing Methods in Statistics, pages 233–257, Ohio State
University. Academic Press, New York.

\bibitem[49]{rydl1} Rydlewski J., (2009), \textit{Estymatory Największej Wiarogodności w Uogólnionych Modelach Regresji Nieliniowej}, Rozprawa doktorska.

\bibitem[50]{sokol} Sokołowski A., (2010), \textit{Jak rozumieć i wykonywać analizę przeżycia} \textit{http://www.statsoft.pl/Portals/0/Downloads/Jak\_rozumiec\_i\_wykonac\_analize\_przezycia.pdf}

\bibitem[51]{views} Statistics Views, (2014), \textit{ "I would like to think of myself as a scientist, who happens largely to specialise in the use of statistics"– An interview with Sir David Cox}. 

\bibitem[52]{sgdpkg} Tran D., Lan T., Toulis P., (2015), \textit{sgd: Stochastic Gradient Descent for Scalable Estimation. R package version 0.1}, \texttt{https://github.com/airoldilab/sgd}.

\bibitem[53]{ther} Therneau T. M., Grambsch P. M., (2000), \textit{Modeling Survival Data: Extending the Cox Model}, Springer.

\bibitem[54]{survival} Therneau T. M., (2015), \textit{A Package for Survival Analysis in S. version 2.38}, \texttt{http://CRAN.R-project.org/package=survival}.


\bibitem[55]{ADALINE} Widrow B., (1960), \textit{An adaptive "ADALINE" neuron using chemical "memistors"}, Technical Report No. 1553-2, Stanford University.

\bibitem[56]{ADALINE2} Widrow B., Ho M.E., (1960), \textit{Adaptive switching circuits}, In: IRE WESCON Conv.
Record, Part 4. pp. 96-104.

\bibitem[57]{widrow2} Widrow B., Stearns S. D., (1985), \textit{Adaptive Signal Processing, Prentice Hall}.

\bibitem[58]{wiki1} Wikipedia, encyklopedia wolnego dostępu \texttt{wikipedia.pl}
 
 \bibitem[59]{sfu1} Woodcock S., (2014), Notatki do otwartego kursu Uniwersytetu Simona Frasera \textit{ECON 837, Lecture 11 Asymptotic Properties of Maximum Likelihood Estimators}, \\ \texttt{http://www.sfu.ca/~swoodcoc/teaching/sp2014/econ837/11.mle.pdf}
 
\bibitem[60]{zieli} Zieliński R., (1990), \textit{Siedem wykładów wprowadzających do statystyki matematycznej}, Warszawa, Wydawnictwo Naukowe PWN.


\end{thebibliography}