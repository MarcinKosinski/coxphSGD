\begin{thebibliography}{99}

\bibitem[1]{aldrich1} Aldrich J., (1997), \textit{R. A. Fisher and the Making of Maximum Likelihood 1912 – 1922}, Statistical Science 1997, Vol. 12, No. 3, 162-176.

\bibitem[2]{assel} Asselain B., Mould R. F., (2010), \textit{Methodology of the Cox proportional hazards model},  Journal of Oncology 2010, volume 60, Number 5,  403–409.

\bibitem[3]{bender} Bender R., Augustin T., Blettner Maria, (2005) \textit{Generating survival times to simulate Cox proportional hazards models}, Statistics in Medicine, Volume 24, Issue 11, 1713–1723.

\bibitem[4]{biecek1} Biecek P., (2011), \textit{Przewodnik po pakiecie R}, Rozprawa doktorska, Oficyna Wydawnicza GiS, wydanie II.

\bibitem[5]{bott1} Bottou L., (2010), \textit{Large-Scale Machine Learning with Stochastic Gradient Descent}.

\bibitem[6]{bottDOD} Bottou L., (1998), \textit{Online Learning and Stochastic Approximations}.

\bibitem[7]{bott2} Bottou L., (2012), \textit{Stochastic Gradient Descent Tricks}.

\bibitem[8]{boxst} Box-Steffensmeier J. M., Jones B. S, (2004), \textit{Event History Modeling: A Guide for Social Scientists}, Cambridge University Press.


\bibitem[9]{broad} Broad Institute TCGA Genome Data Analysis Center (2014): Firehose stddata 2015 06 01 run. Broad Institute of MIT and Harvard. DOI:10.7908/C1251HBG

\bibitem[10]{burzyk1} Burzykowski T., (2015?), \textit{Notatki do przedmiotu Biostatystyka}, \texttt{https://e.mini.pw.edu.pl/sites/default/files/biostatystyka.pdf}.

\bibitem[11]{chin1} Chin L., Hahn W.C., Getz G., Meyerson M., (2011), \textit{Making sense of cancer genomic data. Genes and Development}, 25(6): 534-555.

\bibitem[12]{chin2} Chin L., Andersen J.N., Futreal P.A., (2011), \textit{Cancer genomics: from discovery science to personalized medicine}, Nature Medicine, 17(3): 297-303.

\bibitem[13]{cox0} Cox D. R. (1962), \textit{Renewal Theory. Methuen Monograph on Applied Probability
\& Statistics}, London: Methuen.

\bibitem[14]{cox}  Cox D. R., (1972), \textit{Regression models and life-tables (with discussion)}, Journal of the Royal Statistical Society Series B 34:187-220.


\bibitem[15]{collett} Collett D., (1994), \textit{Modelling Survival Data in Medical Research}, Chapman and Hall:
London.

\bibitem[16]{czepiel} Czepiel S. A., \textit{Maximum Likelihood Estimation of Logistic Regression Models: Theory and Implementation}, \texttt{http://czep.net/stat/mlelr.pdf}.

\bibitem[17]{dennis} Dennis J. E. Jr., Schnabel R. B., (1983), \textit{Numerical Methods For Un-
constrained Optimization and Nonlinear Equations}, Prentice-Hall.

\bibitem[18]{dobson} Dobson A. J., (2002), \textit{An Introduction to Generalized Linear Models}, Wydanie II, Chapman \& Hall/CRC.

\bibitem[19]{zikula2} Domagała W., (2007), \textit{Molekularne podstawy karcynogenezy i ścieżki sygnałowe niektórych nowotworów ośrodkowego układu nerwowego}, Polski Przegląd Neurologiczny, tom 3: 127-141.

\bibitem[20]{finkel}  Finkel J. R., Kleeman A., Manning C. D., (2008), \textit{Efficient, Feature-based, Conditional Random Field Parsing}, Proc. Annual Meeting of the ACL.

\bibitem[21]{fisher1} Fisher R. A., (1912) \textit{An absolute criterion for fitting frequency curves}.

\bibitem[22]{fisher2} Fisher R. A., (1922) \textit{On the mathematical foundations of theoretical statistics}, Philos. Trans. Roy. Soc. London Ser. A 222 309-368.

\bibitem[23]{fortuna} Fortuna Z., Macukow B., Wąsowski J., (2006), \textit{Metody Numeryczne}, Wydawnictwa Naukowo-Techniczne.

\bibitem[24]{gauss1} Gauss C. F., (1809,) \textit{Theoria Motus Corporum Coelestium}.

\bibitem[25]{gagol1} Gągolewski M., (2014), \textit{Programowanie w języku R}, Wydawnictwo Naukowe PWN.

\bibitem[26]{goemann} Goeman J. J., (2010), \textit{L1 penalized estimation in the Cox proportional hazards model}, Biom J. Feb; 52(1):70-84.


\bibitem[27]{hald} Hald A., (1949), \textit{Maximum likelihood estimation of the parameters of a normal distribution which is truncated at a known point}, Skandinavisk Aktuarietidskrift, 119-134.


\bibitem[28]{glmglm} Hastie T. J., Pregibon D., (1992), \textit{Generalized linear models. Chapter 6 of Statistical Models in S}, eds J. M. Chambers and T. J. Hastie, Wadsworth \& Brooks/Cole.


\bibitem[29]{heckman} Heckman J. J., Singer B, (1985), \textit{Longitudinal Analysis of Labor Market Data}, Cambridge University Press.

\bibitem[30]{hosmer} Hosmer D. W., Lemeshow S., May S, (2008), \textit{Applied Survival Analysis: Regression Modeling of Time to Event Data}, Wiley Series in Probability and Statistics 2nd edition, Wiley-Interscience.


\bibitem[31]{hutch1} Hutchinson J. B., (1928), \textit{The Application of the "Method of Maximum Likelihood" to the Estimation of Linkage}, Genetics. 1929 Nov; 14(6): 519–537.


\bibitem[32]{zikula3} Janik-Papis K., Błasiak J., (2010), \textit{Molekularne wyznaczniki raka piersi. Inicjacja i promocja - część I}, Nowotwory - Journal of Oncology, 60 (3): 236-247.

\bibitem[33]{zikula4} Janik-Papis K., Błasiak J., (2010), \textit{Molekularne wyznaczniki raka piersi. Progresja i nowi kandydaci - część II}, Nowotwory - Journal of Oncology, 60 (4): 341-350.

\bibitem[34]{scoring2} Jennrich R. I., Sampson P. F., (1976), \textit{Newton-Raphson and related algorithms for maximum likelihood variance component estimation}, Technometrics, 18, 11-17.


\bibitem[35]{kalf} Kalbfleisch J. D., Prentice R. L, (1980), \textit{The Statistical Analysis of Failure Time Data}, New York: John Wiley and Sons. 

\bibitem[36]{kenward1} Kenward M. G., Lesaffre E. and Molenberghs G., (1994), \textit{An Application of Maximum Likelihood and Generalized Estimating Equations to the Analysis of Ordinal Data from a Longitudinal Study with Cases Missing at Random}, Biometrics Vol. 50, No. 4 (Dec., 1994), pp. 945-953.

\bibitem[37]{KIMKIM} Kim J., Kim Y., (2004), \textit{Gradient lasso for feature selection. In Proceedings of the
21th International Conference on Machine Learning}, Morgan Kaufmann, ACM, New York, pp. 473-480.

\bibitem[38]{KIM} Kim J. Kim Y., Kim Y., (2008), \textit{A gradient-based optimization algorithm for lasso},  Journal of Computational and Graphical Statistics 17, 994-1009.

\bibitem[39]{klein} Klein J. P., Moeschberger M. L, (2003), \textit{Survival Analysis: Techniques For Censored and Truncated Data, 2nd edition}, New York: John Willey and Sons.



\bibitem[40]{kosa0} Kosiński M., (2015), \textit{coxphSGD: Stochastic Gradient Descent log-likelihood estimation in Cox proportional hazards model}, R package version 0.0.3,  \texttt{https://github.com/MarcinKosinski/Cox-SGD}.

\bibitem[41]{kosa1} Kosiński M., Biecek P., (2015), \textit{RTCGA: The Cancer Genome Atlas Data Integration}, R package version 1.1.7, \texttt{https://github.com/RTCGA}.

\bibitem[42]{kosa2} Kosiński M., (2015), \textit{RTCGA.clinical: Clinical datasets from The Cancer Genome Atlas Project}, R package version 20150821.1.2, \\ \texttt{https://bioconductor.org/packages/release/data/experiment/html/RTCGA.clinical.html}

\bibitem[43]{kosa3} Kosiński M., (2015), \textit{RTCGA.mutations: Clinical datasets from The Cancer Genome Atlas Project}, R package version 20150821.1.1, \\ \texttt{https://bioconductor.org/packages/release/data/experiment/html/RTCGA.mutations.html}

\bibitem[44]{kotlowski} Kotłowski W., (2012), Notatki do przedmiotu \textit{Techniki Optymalizacji} prowadzonego na Politechnice Poznańskiej, \\ \texttt{http://www.cs.put.poznan.pl/wkotlowski/teaching/wyklad3b.pdf}

\bibitem[45]{zikula5} Kozłowska J., Łaczmańska I., (2010), \textit{Niestabilność genetyczna - jej znaczenie w procesie powstawania nowotworów oraz diagnostyka laboratoryjna}, Nowotwory - Journal of Oncology, 60 (6): 548-553.

\bibitem[46]{legendre1} Legendre A. M., (1804), \textit{Nouvelles m\`ethods pour la d\`etermination des orbites des com\`etes}.


\bibitem[47]{scoring1} Longford N. T., (1987), \textit{A fast scoring algorithm for maximum likelihood estimation in unbalanced mixed models with nested random effects}, Biometrika 74 (4): 817-827.

\bibitem[48]{milewski} Milewski S., (2006), Konspekt do przedmiotu \textit{Metody Numeryczne} prowadzonego na Politechnice Krakowskiej, \\ \texttt{http://l5.pk.edu.pl/images/skrypty/Metody\_numeryczne\_1}

\bibitem[49]{millar1} Millar R. B., (2011), \textit{Maximum Likelihood Estimation and Inference: With Examples in R, SAS and ADMB, chapter 6. Some Widely Used Applications of Maximum Likelihood}, John Wiley \& Sons, Ltd.

\bibitem[50]{MOOD} Mood A. M., Graybill F. A., Boes D. C., (1974), \textit{Introduction to the Theory of Statistics},
McGraw-Hill: New York.

\bibitem[51]{mital} Mittal S., Madigan D., (2014), \textit{High-dimensional, massive sample-size Cox proportional hazards regression for survival analysis}, Biostatistics, 2014 Apr; 15(2): 207-221.

\bibitem[52]{murata} Murata N., (1998), \textit{A Statistical Study of On-line Learning. In Online Learning
and Neural Networks}, Cambridge University Press.

\bibitem[53]{niemiro} Niemiro W., (2011), Skrypt do przedmiotu \textit{Statystyka} prowadzonego na Uniwersytecie Warszawskim, \\ \texttt{http://www-users.mat.umk.pl/$\sim$wniem/Statystyka/Statystyka.pdf}

\bibitem[54]{norwe} Norwegian Multicentre Study Group, (1981), \textit{Timolol-induced reduction in
mortality and reinfarction}, The New England  Journal of Medicine; 304: 801-7.

\bibitem[55]{oakes} Oakes D., (2001), \textit{Biometrika centenary: survival analysis}, Biometrika, 88(1):99-142.


\bibitem[56]{parkm} Park M. Y., Hastie T., (2007), \textit{L1-regularization path algorithm for generalized linear models}, Journal of the Royal Statistical Society, 69(4):659-677.

\bibitem[57]{mit0} Panchenko D., (2006), Notatki do otwartego kursu MIT \textit{Statistics for Applications, Lecture 2: Maximum Likelihood Estimators.}, \\
\texttt{http://ocw.mit.edu/courses/mathematics/18-443-statistics-for-applications-fall-2006/}

\bibitem[58]{mit1} Panchenko D., (2006), Notatki do otwartego kursu MIT \textit{Statistics for Applications, Lecture 3: Properties of MLE: consistency, asymptotic normality. Fisher information.}, \\
\texttt{http://ocw.mit.edu/courses/mathematics/18-443-statistics-for-applications-fall-2006/}

\bibitem[59]{zikula} Podsiadły K., (2011), \textit{Genetyczne podstawy nowotworzenia}, \texttt{www.e-biotechnologia.pl/Artykuly/Genetyczne-podstawy-nowotworzenia}.

\bibitem[60]{programikr} R Core Team, (2013) \textit{R: A language and environment for statistical computing.} R Foundation for Statistical Computing, Wiedeń , ISBN 3-900051-07-0, \texttt{http://www.R-project.org/}.


\bibitem[61]{robbins} Robbins H. E., Siegmund. D. O., (1971), \textit{A convergence theorem for non negative almost supermartingales and some applications}, In Proc. Sympos. Optimizing Methods in Statistics, pages 233–257, Ohio State
University. Academic Press, New York.

\bibitem[62]{rydl1} Rydlewski J., (2009), \textit{Estymatory Największej Wiarogodności w Uogólnionych Modelach Regresji Nieliniowej}, Rozprawa doktorska.

\bibitem[63]{sohn} Sohn I., Kim J., Jung S. H., Park C., (2009) \textit{Gradient lasso for Cox proportional hazards model}, Bioinformatics, Jul 15; 25(14):1775-81.


\bibitem[64]{sokol} Sokołowski A., (2010), \textit{Jak rozumieć i wykonywać analizę przeżycia} \textit{http://www.statsoft.pl/Portals/0/Downloads/Jak\_rozumiec\_i\_wykonac\_analize\_przezycia.pdf}

\bibitem[65]{views} Statistics Views, (2014), \textit{ "I would like to think of myself as a scientist, who happens largely to specialise in the use of statistics"– An interview with Sir David Cox}.

\bibitem[66]{sgdpkg} Tran D., Lan T., Toulis P., (2015), \textit{sgd: Stochastic Gradient Descent for Scalable Estimation. R package version 0.1}, \texttt{https://github.com/airoldilab/sgd}.


\bibitem[67]{future} \textit{The future of cancer genomics}, Nature Medicine, (2015), 21(2): 99.

\bibitem[68]{ther} Therneau T. M., Grambsch P. M., (2000), \textit{Modeling Survival Data: Extending the Cox Model}, Springer.

\bibitem[69]{survival} Therneau T. M., (2015), \textit{A Package for Survival Analysis in S. version 2.38}, \texttt{http://CRAN.R-project.org/package=survival}.


\bibitem[70]{patrycja} Tomczak K., Czerwińska P., Wiznerowicz M., (2015), \textit{The Cancer Genome Atlas (TCGA): an immeasurable source of knowledge}, Contemporary Oncology. 19(1A): A68-A77.


\bibitem[71]{toulis} Toulis P.,Airoldi E., M., (2015), \textit{Implicit stochastic gradient descent}, arXiv:1408.2923v5 [stat.ME] 4 Oct 2015.

\bibitem[72]{ADALINE} Widrow B., (1960), \textit{An adaptive "ADALINE" neuron using chemical "memistors"}, Technical Report No. 1553-2, Stanford University.

\bibitem[73]{ADALINE2} Widrow B., Ho M.E., (1960), \textit{Adaptive switching circuits}, In: IRE WESCON Conv.
Record, Part 4. pp. 96-104.

\bibitem[74]{widrow2} Widrow B., Stearns S. D., (1985), \textit{Adaptive Signal Processing, Prentice Hall}.

\bibitem[75]{wiki1} Wikipedia, encyklopedia wolnego dostępu \texttt{wikipedia.pl}

 \bibitem[76]{sfu1} Woodcock S., (2014), Notatki do otwartego kursu Uniwersytetu Simona Frasera \textit{ECON 837, Lecture 11 Asymptotic Properties of Maximum Likelihood Estimators}, \\ \texttt{http://www.sfu.ca/~swoodcoc/teaching/sp2014/econ837/11.mle.pdf}

\bibitem[77]{zieli} Zieliński R., (1990), \textit{Siedem wykładów wprowadzających do statystyki matematycznej}, Warszawa, Wydawnictwo Naukowe PWN.


\end{thebibliography}
