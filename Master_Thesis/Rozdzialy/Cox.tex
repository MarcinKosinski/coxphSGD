\chapter{Model Coxa}\label{chap2}
%\begin{flushright}
%\textit{The proportion of my life that I spent working on the \\ 
%proportional hazards model is, in fact, very small. I had an \\
% idea of how to solve it but I could not complete the \\
%  argument and so it took me about four years on and off... \\
%Sir David Cox, An interview with Sir David Cox, 2014.
%}
%\end{flushright}
W tym rozdziale zostanie przedstawiony model proporcjonalnych hazardów Coxa. Głównym celem tej pracy jest wykorzystanie numerycznej metody estymacji współczynników metodą stochastycznego spadku gradientu w omawianym modelu. Więcej o estymacji metodą stochastycznego spadku gradientu napisane jest w rozdziale \ref{SGD}. Definicje i twierdzenia w tym rozdziale oparte są o \cite{cox}, \cite{ther}, \cite{assel} i \cite{burzyk1}.

%\section{Nomenklatura i podstawy analizy przeżycia}

%Analiza przeżycia to zbiór metod statystycznych badających procesy, w których interesujący jest czas, jaki upłynie do (pierwszego) wystąpienia pewnego zdarzenia.

\section{Wprowadzenie do modelu Coxa i nomenklatura}

Analiza przeżycia, w której model Coxa znalazł największe zastosowania, polega na modelowaniu wpływu czynników na czas
do wystąpienia pewnego zdarzenia. Zdarzeniem może być np. śmierć pacjenta, awaria urządzenia, zerwanie umowy przez klienta, odejście z pracy pracownika lub deaktywacja pewnej usługi. Analizując czasy do wystąpienia zdarzenia wykorzystuje się funkcję przeżycia bądź niosącą równoważną informację funkcję hazardu.

W poniższych definicjach $T^*$ to czas do wystąpienia zdarzenia. Zakłada się, że wewnątrz każdej grupy $j=1,2,\dots, m$ wyznaczonej przez poziomy zmiennych objaśniających, czasy~$T_{j,i}^*$ dla \ $i=1,\dots,n_j$~to~niezależne~zmienne~losowe~z~tego~samego~rozkładu~o~gęstości~$f_j(t)$.
\begin{definition}
\textbf{Funkcja przeżycia} w grupie $j$, to funkcja, która spełnia
\begin{equation}
S_j(t) = \mathbb{P}(T^*_{j,i} \geq t )  = 1 - F_j(t), t \in \mathbb{R}
\end{equation}
gdzie $F_j(t)$ to dystrybuanta rozkładu zadanego gęstością $f_j(t)$.
\end{definition}
\begin{definition}\label{def:haz}
\textbf{Funkcja hazardu} to funkcja, która wyraża się wzorem
\begin{equation}
\begin{align*}
\lambda_j(t) & =  \lim\limits_{h\rightarrow 0}\dfrac{\mathbb{P}(t \leq T^* \leq t +h | T^* \geq t)}{h} & \ \\
 \ & = \lim\limits_{h\rightarrow 0}\dfrac{\mathbb{P}(t \leq T^* \leq t +h )}{h}\cdot\frac{1}{  \mathbb{P}(T^* \geq t)} & \ \\ \ & = \lim\limits_{h\rightarrow 0} \frac{F_j(t+h) - F_j(t) }{h}\cdot\frac{1}{S_j(t)} & =  \frac{f_j(t)}{S_j(t)}.
\end{align*}
\end{equation}
\end{definition}


\newpage
Wartość funkcji hazardu w momencie $t$ traktuje się jako chwilowy potencjał pojawiającego się zdarzenia (np. śmierci lub choroby), pod warunkiem że osoba dożyła czasu $t$. Funkcja hazardu nazywana jest również funkcją ryzyka,
intensywnością umieralności (\textit{force of mortality}), umieralnością
chwilową (\textit{instantaneous death rate}) lub chwilową
częstością niepowodzeń (awarii) (\textit{failure rate}). Ostatniego
określenia używa się w teorii odnowy \cite{cox0}, w której analizuje
się awaryjność elementów przemysłowych. 


Model proporcjonalnych hazardów Coxa \cite{cox} jest obecnie najczęściej stosowaną procedurą do modelowania relacji pomiędzy zmiennymi objaśniającymi a przeżyciem, lub innym cenzurowanym zdarzeniem. Model ten umożliwia analizę wpływu czynników prognostycznych na przeżycie. Sir David Cox opracował tego typu model dla tabeli przeżyć i zilustrował zastosowanie modelu dla przypadku
białaczki, ale model może być stosowany do obliczania
przeżyć w odniesieniu do innych chorób, jak
w przypadku przeżyć w chorobach nowotworowych lub
kardiologicznych po transplantacji serca lub zawałach
serca \cite{norwe}. 

\begin{definition}\label{defi:koksik}
\textbf{Model Coxa} zakłada postać funkcji hazardu dla i-tej obserwacji $X_i$ jako
\begin{equation}\label{eq:haz}
\lambda_i(t) = \lambda_0(t)e^{X_i(t)'\beta},
\end{equation}
gdzie $\lambda_0$ to niesprecyzowana nieujemna funkcja nazywana \textit{bazowym hazardem}, a $\beta$ to wektor współczynników rozmiaru $p$, co odpowiada liczbie zmiennych objaśniających w modelu Coxa.
\end{definition}

Takie sformułowanie modelu gwarantuje, że funkcja hazardu jest nieujemna. 


Model Coxa, dla wersji kiedy zmienne są stałe w czasie, nazywany jest \textbf{modelem proporcjonalnych hazardów}, gdyż stosunek (proporcja) hazardów dla dwóch obserwacji $X_i$ oraz $X_j$ jest stały w czasie:  
$$\dfrac{\lambda_i(t)}{\lambda_j(t)} = \dfrac{\lambda_0(t)e^{X_i\beta}}{\lambda_0(t)e^{X_j\beta}} = \dfrac{e^{X_i\beta}}{e^{X_j\beta}} = e^{(X_i-X_j)\beta}.$$

Oznacza to, że hazard dla jednej obserwacji można uzyskać poprzez przemnożenie hazardu dla innej obserwacji przez pewną stałą $c_{ij}$:
$$\lambda_i(t) = \dfrac{e^{X_i'\beta}}{e^{X_j'\beta}} \cdot \lambda_j(t) = c_{ij} \cdot \lambda_j(t).$$

W modelu proporcjonalnych hazardów istotnym elementem jest estymacja stałych $c_{ij}$.

\section{Założenia modelu proporcjonalnego ryzyka Coxa}

Model Coxa znalazł szerokie zastosowanie w sytuacjach, gdy analiza wymaga wykorzystania cenzurowanych danych. Model Coxa jest w stanie wykorzystać je do estymacji współczynników w modelu, przekładających się na proporcje hazardów. Z uwagi na aspekt praktyczny podyktowany warunkami technicznymi prób klinicznych i badań biologicznych, zbiory danych klinicznych zawierają cenzurowane czasy zdarzeń. Oznacza to, że w wielu przypadkach niemożliwe jest obserwowanie czasu zdarzeń dla wszystkich obserwacji w zbiorze. Niekiedy jest to uwarunkowane zbyt długim czasem do wystąpienia zdarzenia. Czasem jest to związane z zaplanowanym okresem próby klinicznej, który jest krótszy niż czas do zdarzenia dla pacjentów, którzy mogli zostać włączeni do próby klinicznej pod koniec jej trwania i nie udało się dla nich zaobserwować czasów zdarzeń. W wielu przypadkach pacjenci, traktowani jako obserwacje w zbiorze, znikają z pola widzenia w momencie, gdy np. przestają pojawiać się na wizytach kontrolnych. Może być to spowodowane negatywnymi relacjami z lekarzem prowadzącym lub przeprowadzką. W~takich sytuacjach wykorzystuje się daną obserwację do momentu jej ostatniej kontroli. Nie rezygnuje się z tej obserwacji w analizie i wykorzystuje się o niej informacje w pełni dla czasu, w którym przebywała pod obserwacją. Jest to ogromna zaleta modelu Coxa.

Z przyczyny cenzurowanych danych potrzebne są założenia modelu dotyczące cenzurowania czasów, które opierają się o następujące definicje.

\begin{definition}
\textbf{Cenzuorwanie prawostronne} polega na zaobserwowaniu czasu 
$$T= \min(T^*, C),$$
gdzie $T^*$ to prawdziwy czas zdarzenia, zaś $C$ jest nieujemną zmienną losową.
\end{definition}


\begin{definition}
\textbf{Cenzurowanie jest niezależne} jeśli zachodzi
$$ \lim\limits_{h\rightarrow 0}\dfrac{\mathbb{P}(t \leq T^* \leq t +h | T^* \geq t)}{h} =  \lim\limits_{h\rightarrow 0}\dfrac{\mathbb{P}(t \leq T^* \leq t +h | T^* \geq t, Y(t) = 1)}{h},$$
gdzie $Y(t) = 1$ jeśli do chwili $t$ nie wystąpiło zdarzenie ani cenzurowanie, czyli jednostka pozostaje narażona na ryzyko zdarzenia oraz $Y(t)=0$ w przeciwnym wypadku.
\end{definition}

Interpretacja tej definicji jest następująca: jednostka cenzurowana w chwili $t$ jest reprezentatywna dla wszystkich innych narażonych na ryzyko zdarzenia w chwili $t$. Innymi słowy cenzurowanie nie wybiera z populacji osobników bardziej albo mniej narażonych na zdarzenie. Cenzurowanie działa niezależnie od mechanizmu występowania zdarzenia.

\begin{definition}
\textbf{Cenzurowanie jest nie-informatywne} jeśli zachodzi
\begin{equation}\label{nieinf}
g(t;\theta, \phi) \equiv g(t;\phi),
\end{equation}
gdzie $g(t;\theta, \phi)$ jest funkcją gęstości dla cenzurowań $C_i$ wyrażonych jako niezależne zmienne losowe o jednakowym rozkładzie, zaś prawdzie czasy $T^*_i$ są interpretowane jako niezależne zmienne losowe o jednakowym rozkładzie i funkcji gęstości $f(t;\theta)$, czyli $\theta$ parametryzuje jedynie rozkład czasów zdarzeń.
\end{definition}

Oznacza to, że cenzurowanie nie daje informacji o parametrach rozkładu czasów zdarzeń, ponieaż nie zależy od parametrów od których zależny jest hazard.

\textbf{Model proporcjonalnych hazardów Coxa oparty jest na założeniach:}
\begin{enumerate}
\item[$i$)] Współczynniki modelu $\beta_k, k = 1,\cdots,p$ są stałe w czasie, co przekłada się na to, że stosunek hazardów dla dwóch obserwacji jest stały w~czasie.
\item[$ii$)] Postać funkcjonalna efektu zmiennej niezależnej - postać modelu $\lambda_i(t) = \lambda_0(t)e^{X_i(t)'\beta}$.
\item[$iii)$] Obserwacje są niezależne.
\item[$iv)$] Cenzurowanie czasów jest nie-informatywne.
\item[$v)$] Cenzurowanie czasów jest niezależne (od mechanizmu występowania zdarzenia).
\end{enumerate}

\section{Estymacja w modelu Coxa}

Funkcja hazardu jest wykładniczą funkcją zmiennych objaśniających, nieznana jest natomiast
postać bazowej funkcji hazardu, co bez dalszych założeń uniemożliwia estymację standardową
metodą największej wiarygodności. Rozwiązaniem Cox’a jest maksymalizacja tylko tego fragmentu funkcji wiarygodności, który zależy jedynie od estymowanych parametrów. W modelu Coxa proporcjonalnych hazardów estymacja współczynników $\beta$ oparta jest o częściową funkcję wiarogodności, którą wprowadził Cox w 1972 r. \cite{cox}. 

Dla konkretnego czasu zdarzenia $t_i$, gdzie w zbiorze obserwowanych jest K czasów zdarzeń, prawdopodobieństwo warunkowe ze względu na liczność zbioru ryzyka w czasie $t_i$, że czas zdarzenia dotyczy $i$-tej jednostki spośród wciąż obserwowanych jest równe
\begin{equation}
\dfrac{e^{X_i'\beta}}{\sum\limits_{l\in \mathscr{R}(t_i)}^{}e^{X_l'\beta}},
\end{equation}
gdzie \textit{zbiór ryzyka} $\mathscr{R}(t_i)$, w chwili $t_i$, rozumiany jest jako zbiór indeksów obserwacji, które są w danym czasie $t_i$ pod obserwacją.

Chcąc estymować współczynniki metodą największej wiarogodności należy rozważyć funkcję wiarogodności, która dla niezależnego cenzurowania prawostronnego ma postać:
\begin{equation}
L(\beta,\varphi) = L_p(\beta)\cdot L^{*}(\beta,\varphi),
\end{equation}
gdzie, dla $\lambda(t)$ wprowadzonego w definicji (\ref{def:haz})
\begin{equation}
L_p(\beta) = \prod\limits_{i=1}^{n}f(t_i;\beta)^{\delta_i}S(t_i;\beta)^{1-\delta_i}=\prod\limits_{i=1}^{n}\lambda(t_i;\beta)^{\delta_i}S(t_i;\beta)
\end{equation}
to częściowa funkcja wiarogodności, a $L^{*}(\beta,\varphi)$ zależy od cenzurowania (parametr $\varphi$).

Wtedy dla niezależnego cenzurowania i dla czasów zdarzeń, które nie zaszły jednocześnie \textbf{częściowa funkcja wiarogodności w modelu Coxa} ma postać:
\begin{equation}
L_p(\beta) = \prod\limits_{i=1}^{K}\dfrac{e^{X_i'\beta}}{\sum\limits_{l=1}^{n}Y_l(t_i)e^{X_l'\beta}},
\end{equation}
gdzie $Y_l(t_i)$ = 1, gdy obserwacja $X_l$ jest w zbiorze ryzyka w czasie $t_i$, i $Y_l(t_i)$ = 0 w przeciwnym przypadku, $n$ to liczba obserwacji w zbiorze, a $K$ to wspomniana wyżej liczba zaobserwowanych czasów zdarzeń. Zaletą takiej postaci funkcji częściowej wiarogodności jest to, że w~jej wzorze nie występuje funkcja bazowego hazardu, zatem estymacja współczynników może odbywać się bez znajomości jej postaci.

Jeśli dodatkowo cenzurowanie jest nie-informatywne, to $L_p(\beta)$ jest \textbf{pełną} funkcją wiarogodności, bowiem wówczas $$L^{*}(\beta,\varphi)\propto L^{*}(\varphi)$$ co bierze się z definicji cenzurowania nie-informatywnego (\ref{nieinf}) $$g(t;\theta, \phi) \equiv g(t;\phi).$$

Ponieważ model proporcjonalnych hazardów Coxa zakłada niezależność i nie-informatywność cenzurowania zatem można uważać, że częściowa funkcja wiarogodności daje pełną informację o współczynnikach i wnioskowanie w oparciu o nią jest uzasadnione i poprawne.

W sytuacjach, gdy nie jest spełnione założenie nie-informatywności cenzurowania i częściowa funkcja wiarogodności nie jest funkcją wiarogodności w sensie bycia proporcjonalną do prawdopodobieństwa obserwowanego zbioru, można ją traktować jako funkcję wiarogodności dla celów asymptotycznego wnioskowania o współczynnikach modelu, zobacz \cite{ther}.


\subsubsection{Analityczna estymacja współczynników}
Standardowo w celu znalezienia maximum, aby ułatwić obliczenia, można rozważaną funkcję obłożyć monotoniczną transformacją jaką jest logarytm, tak aby w konsekwencji otrzymać \textbf{częściową~funkcję~log-wiarogodności}
\begin{equation}
\ell_p(\beta) = \sum\limits_{i=1}^{K}X_i'\beta - \sum\limits_{i=1}^{K}\log\Big(\sum\limits_{l\in \mathscr{R}(t_i)}^{}e^{X_l'\beta}\Big).
\end{equation}
Analityczne obliczenia dają $p$-wymiarowy wektor pochodnych, dla $k=1,\dots,p$
\marginnote{\tiny{$X_{ik}$~to~$i$~ta \\ obserwacja \\ i $k$ ta zmienna.}}[1cm]
\begin{equation}\label{score}
U_k(\beta)=\dfrac{\partial\ell_k(\beta)}{\partial\beta_k}=\sum\limits_{i=1}^{K}\Big(X_{ik}-A_{ik}\Big),
\end{equation}
gdzie czynnik
\begin{equation}
A_{ik} = \dfrac{\sum\limits_{l\in \mathscr{R}(t_i)}^{} X_{lk} e^{X_l'\beta}}{\sum\limits_{l\in \mathscr{R}(t_i)}^{} e^{X_l'\beta}}
\end{equation}
to średnia z $X_{.k}$ ($k$-tych zmiennych) po skończonej populacji $\mathscr{R}(t_i)$, z wykorzystaniem \textit{ważonej eksponencjalnie} formy próbkowania.

Z kolei drugie pochodne cząstkowe, jak podaje \cite{cox}, mają postać dla $k_1,k_2=1,\dots,p$
\begin{equation}\label{grad}
\mathscr{I}_{k_1k_2}(\beta) = - \dfrac{\partial^2L_p(\beta)}{\partial\beta_{k_1}\partial\beta_{k_2}}=\sum\limits_{i=1}^{K}C_{ik_1k_2}(\beta),
\end{equation}
gdzie
\begin{equation}
C_{ik_1k_2}(\beta)=\dfrac{\sum\limits_{l\in \mathscr{R}(t_i)}^{} X_{lk_1}X_{lk_2}e^{X_l'\beta}}{\sum\limits_{l\in \mathscr{R}(t_i)}^{} e^{X_l'\beta}} - A_{ik_1}(\beta)A_{ik_2}(\beta)
\end{equation}
to kowariancja pomiędzy $X_{.k_1}$ ($k_1$-tymi zmiennymi) a $X_{.k_2}$ ($k_2$-tymi zmiennymi) przy tej formie ważonego próbkowania.

Estymator największej wiarogodności $\beta$ można uzyskać poprzez przyrównanie (\ref{score}) do $0$, a numerycznie poprzez iteracyjne wykorzystanie (\ref{score}) oraz (\ref{grad}) w algorytmie spadku gradientu rzędu II nazywanego również algorytmem Raphsona-Newtona, który jest opisany w podrozdziale \ref{R-N}. Jest to tradycyjne i szeroko stosowane podejście do estymacji współczynników w modelu proporcjonalnych hazardów Coxa. Niniejsza praca skupia się na wykorzystaniu metody estymacji współczynników jaką jest metoda stochastycznego spadku wzdłuż gradientu, która jest szerzej opisana w następnym rozdziale w podrozdziale \ref{SGD}.

\newpage

\section{Generowanie danych dla modelu Coxa}

W celu skutecznej diagnostyki procesu estymacji w modelu Coxa, należy wiedzieć jak dane dla tego modelu można generować, aby ów model symulować dla znanych współczynników.

Poniższy rozdział przedstawia metodę odwrotnych prawdopodobieństw opisaną szerzej w \cite{bender}, dzięki której można wygenerować czasy zdarzeń dla zadanej z góry funkcji hazardu i zmiennych objaśniających. Ta metoda posłuży w rozdziale \ref{rozdz4} do wygenerowania danych w celu weryfikacji jakości procesu numerycznej estymacji współczynników modelu, w sytuacji gdy wykorzystywany jest algorytm stochastycznego spadku gradientu, który jest opisany w rozdziale \ref{numPAJ}.

Mówiąc o funkcji przeżycia warto wprowadzić skumulowaną funkcję hazardu.
\begin{definition}
\textbf{Skumulowaną funkcją hazardu} nazywa się funkcję spełniającą zależność
\begin{equation}\label{zaradzix}
H(t) = \int\limits_{0}^{t}\lambda(u) du,
\end{equation}
gdzie $\lambda(u)$ to pewna funkcja hazardu, o której mówi definicja (\ref{def:haz}).
\end{definition}

Wtedy dla zdefiniowanej w (\ref{eq:haz}) funkcji hazardu funkcja przeżycia i jej dopełnienie (dystrybuanta) dla modelu Coxa proporcjonalnych hazardów wygląda następująco

\begin{align}
S(t|x)= & e^{-H_0(t)\cdot e^{x'\beta}} \\ 
F(t|x) = & 1 - e^{-H_0(t)\cdot e^{x'\beta}}\label{siema}
\end{align}
gdzie $H_0(t)$ to bazowa skumulowana funkcja hazardu.

Niech $Y$ będzie zmienną losową o dystrybuancie zadanej w (\ref{siema}), wtedy jak wykazano w~\cite{MOOD} zmienna losowa $U=F(Y)$ pochodzi z rozkładu jednostajnego $U \sim \mathcal{U}([0,1])$. Zachodzi to również dla zmiennej losowej $1-U \sim \mathcal{U}([0,1])$. Dodatkowo, niech $T$ będzie czasem przeżycia w modelu Coxa (definicja \ref{defi:koksik}), wtedy z (\ref{siema}) wynika
\begin{equation}
U = e^{-H_0(T)\cdot e^{x'\beta}} \sim \mathcal{U}([0,1]).
\end{equation}

Jeżeli $\lambda_0(t) >0$ dla każdego $t$, to $H_0(t)$ jest dodatnia i można mówić o jej odwrotności, zaś czas przeżycia $T$ dla modelu Coxa może być wyrażony przez 
\begin{equation}\label{eq:czasy}
T = H_0^{-1}(-\log(U)\cdot e^{-x'\beta}),
\end{equation}
gdzie  $U \sim \mathcal{U}([0,1])$.

\subsubsection{Przykład symulacji dla rozkładu Weibulla}\label{symWei}

Równanie (\ref{eq:czasy})  jest odpowiednie do generowania czasów do zdarzenia w modelu Coxa, gdy potrafi się odpowiednio generować zmienne z rozkładu $\mathcal{U}([0,1])$. Jest to możliwe w większości pakietów statystycznych. Poniżej przedstawiony jest kod w języku $\mathcal{R}$ \cite{programikr}, dzięki któremu możliwe jest generowanie czasów zdarzeń pochodzących z rozkładu Weibulla \cite{collett}, dla którego funkcja hazardu ma postać
\begin{equation}
\lambda_0(t)=\lambda\rho t^{\rho-1},
\end{equation}
gdzie $\lambda>0$ to parametr skali, zaś $\rho > 0$ to parametr kształtu.

\newpage

Z (\ref{zaradzix}) wynika, że bazowa skumulowana funkcja hazardu dla rozkładu Weibulla wynosi
\begin{equation}
H_0(t) = \int\limits_{0}^{t} \lambda\rho u^{\rho-1} du = \lambda t^{\rho},
\end{equation}
zaś jej funkcja przeciwna to
\begin{equation}\label{asd}
H_0^{-1}(t)= (\lambda^{-1}t)^{-\rho}.
\end{equation}
Wtedy podstawiając (\ref{asd}) do (\ref{eq:czasy}) można otrzymać
\begin{equation}
T = (\lambda^{-1}\cdot(-\log(U))\cdot e^{-x'\beta})^{-\rho} = \Big(-\frac{\log(U)}{\lambda e^{x'\beta}}\Big)^{-\rho}.
\end{equation}
Wynik ten oznacza, że $T$ czyli odpowiadające czasy zdarzeń pochodzą z warunkowego rozkładu Weibulla (warunkowanego przez $x$) o parametrach kształtu $\rho$ i skali $\lambda  e^{x'\beta}$.

Dzięki tym rozważaniom, możliwe było stworzenie kodu generującego czasy i indykatory zdarzeń dla zadanych z góry współczynników modelu i zmiennych objaśniających. Poniższy kod i jego rezultat posłużą w rozdziale \ref{implemento} do diagnostyki procesu estymacji w modelu Coxa w przypadku wykorzystania algorytmu stochastycznego spadku gradientu.

\newline \ \newline
\begin{Shaded}
\begin{Highlighting}[]
\NormalTok{dataCox <-}\StringTok{ }\NormalTok{function(N, lambda, rho, x, beta, censRate)\{}
  
  \CommentTok{# real Weibull times}
  \NormalTok{u <-}\StringTok{ }\KeywordTok{runif}\NormalTok{(N)}
  \NormalTok{Treal <-}\StringTok{ }\NormalTok{(-}\StringTok{ }\KeywordTok{log}\NormalTok{(u) /}\StringTok{ }\NormalTok{(lambda *}\StringTok{ }\KeywordTok{exp}\NormalTok{(x %*%}\StringTok{ }\NormalTok{beta)))^(}\DecValTok{1} \NormalTok{/}\StringTok{ }\NormalTok{rho)}
  
  \CommentTok{# censoring times}
  \NormalTok{Censoring <-}\StringTok{ }\KeywordTok{rexp}\NormalTok{(N, censRate)}
  
  \CommentTok{# follow-up times and event indicators}
  \NormalTok{time <-}\StringTok{ }\KeywordTok{pmin}\NormalTok{(Treal, Censoring)}
  \NormalTok{status <-}\StringTok{ }\KeywordTok{as.numeric}\NormalTok{(Treal <=}\StringTok{ }\NormalTok{Censoring)}
  
  \CommentTok{# data set}
  \KeywordTok{data.frame}\NormalTok{(}\DataTypeTok{id=}\DecValTok{1}\NormalTok{:N, }\DataTypeTok{time=}\NormalTok{time, }\DataTypeTok{status=}\NormalTok{status, }\DataTypeTok{x=}\NormalTok{x)}
\NormalTok{\}}

\NormalTok{x <-}\StringTok{ }\KeywordTok{matrix}\NormalTok{(}\KeywordTok{sample}\NormalTok{(}\DecValTok{0}\NormalTok{:}\DecValTok{1}\NormalTok{, }\DataTypeTok{size =} \DecValTok{40}\NormalTok{, }\DataTypeTok{replace =} \OtherTok{TRUE}\NormalTok{), }\DataTypeTok{ncol =} \DecValTok{2}\NormalTok{)}

\KeywordTok{head}\NormalTok{(}\KeywordTok{dataCox}\NormalTok{(}\DecValTok{20}\NormalTok{, }\DecValTok{3}\NormalTok{, }\DecValTok{2}\NormalTok{, x, }\DataTypeTok{beta =} \KeywordTok{c}\NormalTok{(}\DecValTok{2}\NormalTok{,}\DecValTok{3}\NormalTok{), }\DecValTok{5}\NormalTok{))}
\end{Highlighting}
\end{Shaded}

\begin{verbatim}
  id       time status x.1 x.2
1  1 0.01193626      0   0   1
2  2 0.03567485      0   1   1
3  3 0.13330012      1   0   1
4  4 0.04358821      1   1   1
5  5 0.03825366      1   1   1
6  6 0.29355955      1   1   0
\end{verbatim}
