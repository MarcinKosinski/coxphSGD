\documentclass[]{mini}
\usepackage[utf8]{inputenc}
\usepackage{pdfpages}
\usepackage{mathtools}
\usepackage{mathrsfs}
\usepackage{marginnote}
\usepackage{geometry}
\setlength{\marginparwidth}{2.21cm}
%\usepackage{mathmode}
% % % % % % % % % % % % % % % % % % % % % % %
\usepackage{lmodern}
\usepackage{amsmath}
\usepackage{amsfonts}
\usepackage{amssymb}
\usepackage{ifxetex,ifluatex}
\usepackage{fixltx2e} % provides \textsubscript
\ifnum 0\ifxetex 1\fi\ifluatex 1\fi=0 % if pdftex
  \usepackage[T1]{fontenc}
  \usepackage[utf8]{inputenc}
\else % if luatex or xelatex
  \ifxetex
    \usepackage{mathspec}
    \usepackage{xltxtra,xunicode}
  \else
    \usepackage{fontspec}
  \fi
  \defaultfontfeatures{Mapping=tex-text,Scale=MatchLowercase}
  \newcommand{\euro}{€}
\fi
% use upquote if available, for straight quotes in verbatim environments
\usepackage{color}
\usepackage{fancyvrb}
\newcommand{\VerbBar}{|}
\newcommand{\VERB}{\Verb[commandchars=\\\{\}]}
\DefineVerbatimEnvironment{Highlighting}{Verbatim}{commandchars=\\\{\}}
% Add ',fontsize=\small' for more characters per line
\usepackage{framed}
\definecolor{shadecolor}{RGB}{248,248,248}
\newenvironment{Shaded}{}{}
\newcommand{\KeywordTok}[1]{\textcolor[rgb]{0.00,0.00,1.00}{{#1}}}
\newcommand{\DataTypeTok}[1]{{#1}}
\newcommand{\DecValTok}[1]{{#1}}
\newcommand{\BaseNTok}[1]{{#1}}
\newcommand{\FloatTok}[1]{{#1}}
\newcommand{\CharTok}[1]{\textcolor[rgb]{0.00,0.50,0.50}{{#1}}}
\newcommand{\StringTok}[1]{\textcolor[rgb]{0.00,0.50,0.50}{{#1}}}
\newcommand{\CommentTok}[1]{\textcolor[rgb]{0.00,0.50,0.00}{{#1}}}
\newcommand{\OtherTok}[1]{\textcolor[rgb]{1.00,0.25,0.00}{{#1}}}
\newcommand{\AlertTok}[1]{\textcolor[rgb]{1.00,0.00,0.00}{{#1}}}
\newcommand{\FunctionTok}[1]{{#1}}
\newcommand{\RegionMarkerTok}[1]{{#1}}
\newcommand{\ErrorTok}[1]{\textbf{{#1}}}
\newcommand{\NormalTok}[1]{{#1}}
%\ifxetex
%  \usepackage[setpagesize=false, % page size defined by xetex
%              unicode=false, % unicode breaks when used with xetex
%              xetex]{hyperref}
%\else
%  \usepackage[unicode=true]{hyperref}
%\fi
%\hypersetup{breaklinks=true,
%            bookmarks=true,
%            colorlinks=true,
%            citecolor=blue,
%            urlcolor=blue,
%            linkcolor=magenta,
%            pdfborder={0 0 0}}
\urlstyle{same}  % don't use monospace font for urls
\setlength{\parindent}{0pt}
\setlength{\parskip}{6pt plus 2pt minus 1pt}
\setlength{\emergencystretch}{3em}  % prevent overfull lines
% % % % % % % % % % % % % % % % % % % % % % % % % % %
%------------------------------------------------------------------------------%
\title{Estymacja w modelu Coxa metodą~stochastycznego~spadku~gradientu z~przykładami~zastosowań~w~analizie~danych z~The~Cancer~Genome~Atlas}
\titleaux{Stochastic gradient descent method in~Cox~models~with~applications to The~Cancer~Genome~Atlas~data}
\author{Marcin Piotr Kosiński}
\supervisor{\text{prof. ndzw. dr hab. inż Przemysław Biecek}}
\type{magisters}
\discipline{matematyka}
\specialty{Statystyka Matematyczna i Analiza Danych}
\monthyear{lipiec 2015}
\date{\today}
\album{265361}
%------------------------------------------------------------------------------%
\begin{document}
\maketitle
\tableofcontents

\chapter*{Wprowadzenie}


+++
Analiza przeżycia.
+++

Najbardziej charakterystyczną cechą typowych danych, jakimi posługuje się w analizie przeżycia, jest obecność obiektów, w których końcowe zdarzenie nastąpiło (wówczas ma się do czynienia z obserwacjami \textit{kompletnymi}), oraz obiektów, w których to zdarzenie (jeszcze) nie nastąpiło (obserwacja \textit{ucięta}). Ta specyficzna postać danych statystycznych doprowadziła do powstania specjalnych metod stosowanych tylko w analizie czasu trwania zjawisk. Jednym z takich modeli jest model proporcjonalnych hazardów Coxa. Jak podaje \cite{assel}, model proporcjonalnych hazardów Coxa jest jednym z najszerzej stosowanych modeli w onkologicznych publikacjach naukowych, ale także jedną z najmniej rozumianych metod statystycznych. Wynika to z łatwego dostępu do pakietów statystycznych zawierających programy do analizy przeżyć, modeli regresji i analiz wielowariantowych, ale prawie nigdy nie zawierających dobrego opisu podstawowych zasad działania modelu Coxa. Dostarczają one wyłącznie instrukcje, jak wprowadzić dane i uruchomić odpowiednie procedury w celu uzyskania wyniku. Poniższy praca zawiera pełny opis metodologii modelu proporcjonalnych hazardów Coxa, w tym wyjaśnienie najważniejszych pojęć. 


++++++++

\chapter{Estymacja metodą największej wiarogodności}
\begin{flushright}
\textit{The making of maximum likelihood was one of \\
the most important developments in 20th century statistics. \\
It was the work of one man but it was no simple process (...). \\
John Aldrich o R. A. Fisher'ze, 1997 \cite{aldrich1}
}
\end{flushright}

\section{Estymacja}
\section{Metoda największej wiarogodności}

Metodę największej wiarogodności wprowadził R. A. Fisher w 1922 r. \cite{fisher2}, dla której po raz pierwszy procedurę numeryczną zaproponował już w 1912 r. \cite{fisher1}. O burzliwym procesie powstawania metody, o zmianach w jej uzasadnieniu, o koncepcjach, które powstały w obrębie tej metody takich jak parametr, statystyka, wiarogodność, dostateczność czy efektywność oraz o podejściach, które Fisher odrzucił tworząc podstawy pod nową teorię można przeczytać w~obszernej pracy dokumentalnej \cite{aldrich1}. 

Metoda ta, jako alternatywa dla metody najmniejszych kwadratów \cite{legendre1}, \cite{gauss1}, była rozwijana i szeroko stosowana później przez wielu statystyków i wciąż znajduje obszerne zastosowania w wielu obszarach estymacji statystycznej, np. \cite{hutch1}, \cite{kenward1}, \cite{millar1}.

Aby zdefiniować estymator oparty o metodę największej wiarogodności, należy najpierw wprowadzić pojęcie funkcji wiarogodności.

\begin{definition}
\textbf{Funkcją wiarogodności} nazywamy funkcję $L : \Theta \rightarrow \mathbb{R}$ daną wzorem $$ L(\Theta|x_1, \dots , x_n) = f(\Theta; x_1, \dots , x_n),$$
którą rozważamy jako funkcję parametru $\theta$ przy ustalonych wartościach obserwacji $x_1, \dots , x_n$, gdzie $$ f(\Theta; x_1, \dots , x_n) = \left\lbrace \begin{align*}
\ & \mathbb{P}_{\Theta}( X_1 = x_1, \dots , X_n = x_n), & \text{dla rozkładów dyskretnych}, & \ \\
\ & f_{\Theta}(x_1, \dots , x_n), & \text{dla rozkładów absolutnie ciągłych.} & \
\end{align*}$$

\end{definition}

Oznacza to, że wiarogodność jest właściwie tym samym, co gęstość prawdopodobieństwa,
ale rozważana jako funkcja parametru $\theta$, przy ustalonych wartościach obserwacji
$x = X(\omega)$.


\begin{definition}
\textbf{Estymatorem największej wiarogodności} parametru $\theta$, oznaczanym ENW($\theta$), nazywamy wartość parametru, w której funkcja
wiarogodności przyjmuje supremum $$L(\hat{\theta}) = \sup_{\theta \in \Theta} L(\theta).$$

\end{definition}

Niektóre pozycje w literaturze, w definicji estymatora największej wiarogodności, supremum zastępują wartością największą \cite{rydl1}, str 14.

\section{Asymptotyczne własności estymatorów największej wiarogodności}

W tym podrozdziale zostanie wykazane, że estymator największej wiarygodności jest
\begin{enumerate}
\item asymptotycznie nieobciążony,
\item zgodny,
\item asymptotycznie normalny.
\end{enumerate}

\subsection{Zgodność estymatorów największej wiarogodności}
\subsection{Asymptotyczna normalność estymatorów największej wiarygodności}
\chapter{Model Coxa}
\begin{flushright}
\textit{The proportion of my life that I spent working on the \\ 
proportional hazards model is, in fact, very small. I had an \\
 idea of how to solve it but I could not complete the \\
  argument and so it took me about four years on and off... \\
Sir David Cox, An interview with Sir David Cox, 2014.
}
\end{flushright}


W tym rozdziale zostanie przedstawiony model proporcjonalnych hazardów Coxa. Głównym celem tej pracy jest wykorzystanie, nietypowej w tym modelu, numerycznej metody estymacji współczynników metodą stochastycznego spadku gradientu. Więcej o estymacji metodą stochastycznego spadku gradientu napisane jest w rozdziale \ref{SGD}. Definicje i twierdzenia w tym rozdziale oparte są o \cite{cox}, \cite{ther} i \cite{assel}.

%\section{Nomenklatura i podstawy analizy przeżycia}

%Analiza przeżycia to zbiór metod statystycznych badających procesy, w których interesujący jest czas, jaki upłynie do (pierwszego) wystąpienia pewnego zdarzenia.

\section{Wprowadzenie do modelu Coxa i nomenklatura}

Model proporcjonalnych hazardów Coxa \cite{cox} jest obecnie najczęściej stosowaną procedurą do modelowania relacji pomiędzy zmiennymi objaśniającymi a przeżyciem lub innym cenzurowanym rezultatem. Model ten umożliwia analizę wpływu czynników prognostycznych na przeżycie. Sir David Cox opracował tego typu model dla tabeli przeżyć i zilustrował zastosowanie modelu dla przypadku
leukemii, ale model może być stosowany do obliczania
przeżyć w odniesieniu do wszystkich innych chorób, jak
w przypadku przeżyć w chorobach nowotworowych lub
kardiologicznych po transplantacji serca lub zawałach
serca \cite{norwe}. 

\begin{definition}
\textbf{Model Coxa} określa funkcję hazardu dla i-tej obserwacji $X_i$ jako
\begin{equation}
\lambda_i(t) = \lambda_0(t)e^{X_i(t)\beta},
\end{equation}
gdzie $\lambda_0$ to niesprecyzowana nieujemna funkcja nazywana \textit{bazowym hazardem}, a $\beta$ to wektor współczynników o rozmiarze $p$, co odpowiada liczbie zmiennych objaśniających w modelu.
\end{definition}

Wspomnianą funkcję hazardu definiuje się jak następuje:
\begin{definition}
\textbf{Funkcja hazardu} to funkcja, która wyraża się wzorem
\begin{equation}
\begin{align*}
\lambda_j(t) & =  \lim\limits_{h\rightarrow 0}\dfrac{\mathbb{P}(t \leq T^* \leq t +h | T^* \geq t)}{h} & \ \\
 \ & = \lim\limits_{h\rightarrow 0}\dfrac{\mathbb{P}(t \leq T^* \leq t +h )}{h}\cdot\frac{1}{  \mathbb{P}(T^* \geq t)} & \ \\ \ & = \lim\limits_{h\rightarrow 0} \frac{F_j(t+h) - F_j(t) }{h}\cdot\frac{1}{S_j(t)} & =  \frac{f_j(t)}{S_j(t)}.
 \end{align*}
\end{equation}
\end{definition}

W powyższej definicji $T^*$ oznacza czas do wystąpienia zdarzenia. Zakłada się, że wewnątrz każdej grupy $j=1,2,\dots, k, k \in \mathbb{N}$ czasy $T_i^*$ to niezależne zmienne losowe z tego samego rozkładu o zadanej gęstości $f_j(t)$, zaś $S_j(t)$ to \textit{funkcja przeżycia} w grupie $j$, która spełnia 
\begin{equation}
S_j(t) = \mathbb{P}(T^* \geq t )  = 1 - F_j(t), 
\end{equation}
gdzie $F_j(t)$ to dystrybuanta rozkładu zadanego gęstością $f_j(t)$.


Wartość funkcji hazardu w momencie $t$ traktuje się jako chwilowy potencjał pojawiającego się zdarzenia (np. śmierci lub choroby), pod warunkiem że osoba dożyła czasu $t$. Funkcja hazardu nazywana jest również funkcją ryzyka,
intensywnością umieralności (\textit{force of mortality}), umieralnością
chwilową (\textit{instantaneous death rate}) lub chwilową
częstością niepowodzeń (awarii) (\textit{failure rate}). Ostatniego
określenia używa się w teorii odnowy \cite{cox0}, w której analizuje
się awaryjność elementów przemysłowych. 


Model Coxa nazywany jest modelem proporcjonalnych hazardów, gdyż stosunek (proporcja) hazardów dla dwóch obserwacji $X_i$ oraz $X_j$, które mają współczynniki stałe w czasie, jest stały w czasie:  
$$\dfrac{\lambda_i(t)}{\lambda_j(t)} = \dfrac{\lambda_0(t)e^{X_i\beta}}{\lambda_0(t)e^{X_j\beta}} = \dfrac{e^{X_i\beta}}{e^{X_j\beta}}$$

\textbf{Założenia modelu proporcjonalnego ryzyka Coxa.}

\section{Estymacja w modelu Coxa}
Estimation of ß is based on the partiallikelihood function introduced by
Cox [36]. For untied failure time data it has the form
P L(ß) = fI rr { Yi(t)ri(ß, t) }dNi(tl ,
i=l t~O Lj Yj(t}rj(ß, t) (3.2)
where ri(ß, t) is the risk score for subject i, ri(ß, t) = exp[Xi(t)ß] == ri(t).
The log partial likelihood can be written as a sum
I(ß) ~ t, f [Y;( t)X;( t)ß -log ( ~ Y, (t)r, (t») 1 dN; (t), (3.3)
from which we can already fore see the counting process structure.
Although the partiallikelihood is not, in general, a likelihood in the sense
of being proportional to the probability of an observed dataset, nonetheless
it can be treated as a likelihood for purposes of asymptotic inference.
Differentiating the log partial likelihood with respect to ß gives the p x 1
score vector, U(ß):
(3.4)
where x(ß, s) is simply a weighted mean of X, over those observations still
at risk at time s,
-(ß ) = L Yi(s)ri(s)Xi(s) x ,s L Yi(s)ri(s) , (3.5)
with Yi (s )ri (s) as the weights.
The negative second derivative is the p x p information matrix
I(ß) = t 100 V(ß, s)dNi(s),
i=l 0
(3.6)
where V (ß, s) is the weighted variance of X at time s:
V(ß, s) = Li Yi(sh(s)[Xi(s) - x(ß, s)]'[Xi(s) - x(ß, s)]. (3.7)
Li Yi(sh(s)
The maximum partiallikelihood estimator is found by solving the partial
likelihood equation:
U(ß) = O.
The solution 13 is consistent and asymptotically normally distributed with
mean ß, the true parameter vector, and variance {EI(ß)} ~I, the inverse 
3.1 Introduction and notation 41
of the expected information matrix. The expectation requires knowledge of
the censoring distribution even for those observations with observed failures;
that information is typically nonexistent. The inverse of the observed
information matrix I-I (13) is available, has better finite sampIe properties
than the inverse of the expected information, and is used as the variance
of ß.
Eoth SAS and S-Plus use the Newton-Raphson algorithm to solve the
partiallikelihood equation. Starting with an initial guess 13(01, the algorithm
iteratively computes
(3.8)
until convergence, as assessed by stability in the log partial likelihood,
I(ß(nH1) ~ l(ß(n1). This algorithm is incredibly robust for the Cox partial
likelihood. Convergence problems are very rare using the default initial
value of 13(01 = 0 and easily addressed by simple methods such as stephalving.
As a result many packages (e.g., SAS) do not even have an option
to choose alternate starting values. 
\section{Płynne przejście do kolejnej sekcji}

%\section{Estymacja analityczna w oparciu o metodę największej wiarogodności dla funkcji pseudo/sub-wiarogodności}
%\section{Estymacja numeryczna w oparciu o metodę stochastycznego spadku gradientu rzędu I dla funkcji pseudo/sub-wiarogodności}
\newpage
Poszukujemy rozwiazan równosci
$$ \delta ln Ln  / \delta \theta =  0.$$
Tym razem w ogólnym przypadku zwykle nie znajdziemy analitycznego rozwiazania. W
zwiazku z tym jestesmy zdani na metody iteracyjne. Poza tym, byc moze rozwiazanie problemu
nie istnieje albo istnieje ich wiele. Zwykle uzywa sie do tego celu, tj. znalezienia
rozwiazania, metody Newtona, zwykle w literaturze statystycznej w zastosowaniu do tego
problemu, nazywanej metoda Newtona-Raphsona. W efekcie w zasadzie dla kazdego modelu
z osobna nalezy badac własnosci asymptotyczne estymatora najwiekszej wiarygodnosci.
\chapter{Numeryczne metody estymacji}

Przez \textbf{numerykę} rozumie się dziedzinę matematyki
zajmującą się rozwiązywaniem przybliżonych zagadnień algebraicznych. Odkąd zjawiska przyrodnicze zaczęto opisywać przy użyciu formalizmu matematycznego,
pojawiła się potrzeba rozwiązywania zadań analizy matematycznej czy algebry. Dopóki były
one nieskomplikowane, dawały się rozwiązywać analitycznie, tzn. z użyciem pewnych
przekształceń algebraicznych prowadzących do otrzymywania rozwiązań ścisłych danych
problemów. Z czasem jednak, przy powstawaniu coraz to bardziej skomplikowanych teorii
opisujących zjawiska, problemy te stawały się na tyle złożone, iż ich rozwiązywanie ścisłe
było albo bardzo czasochłonne albo też zgoła niemożliwe. Numeryka pozwalała znajdywać
przybliżone rozwiązania z żądaną dokładnością. Ich podstawową zaletą była ogólność tak
formułowanych algorytmów, tzn. w ramach danego zagadnienia nie miało znaczenia czy było
ono proste czy też bardzo skomplikowane (najwyżej wiązało się z większym nakładem pracy
obliczeniowej). Natomiast wadą była czasochłonność. Stąd prawdziwy renesans metod
numerycznych nastąpił wraz z powszechnym użyciem w pracy naukowej maszyn cyfrowych,
a w szczególności mikrokomputerów \cite{milewski}. Dziś dziesiątki żmudnych dla człowieka operacji
arytmetycznych wykonuje komputer, jednak złożoność obliczeniowa algorytmów uczących i modeli statystycznych stała się krytycznym czynnikiem ograniczającym w sytuacjach, gdy rozważane są duże zbiory danych. Te ograniczenia spowodowały, że w uczeniu maszynowym i modelowaniu statystycznym wielkiej skali zaczęto wykorzystywać algorytmy \textbf{stochastycznego spadku gradientu}. W poniższym rozdziale przedstawione są klasyczne algorytmy spadku wzdłuż gradientu Cauchy'ego oraz Raphsona-Newtona. Następnie omówiony jest algorytm stochastycznego spadku wzdłuż gradientu, którego wykorzystanie do estymacji współczynników w modelu Coxa jest kluczowym celem tej pracy. Algorytm stochastycznego spadku gradientu to metoda optymalizacji wzdłuż spadku gradientu wykorzystywana w sytuacjach, gdy rozważaną funkcję można zapisać jako sumę różniczkowalnych składników. Ponadto przedstawiono również zalety algorytmów stochastycznego spadku gradientu, które przemawiają za atrakcyjnością i popularnością tego typu rozwiązania. Ostatecznie przedyskutowano asymptotyczną efektywność estymatorów uzyskanych dzięki jednemu przejściu po zbiorze, zwanym \textit{epoką}. Definicje i pojęcia w tym rozdziale pochodzą z~\cite{bott1},~\cite{bott2},~\cite{kotlowski}.

\newpage
\section{Algorytmy spadku wzdłuż gradientu}

t tekst tekst tekst tekst tekst tekst tekst tekst tekst tekst tekst tekst tekst tekst tekst ekst tekst tekst tekst tekst t tekst tekst tekst tekst tekst tekst tekst tekst tekst tekst tekst tekst tekst tekst tekst ekst tekst tekst tekst tekst t tekst tekst tekst tekst tekst tekst tekst t tekst tekst tekst tekst tekst tekst tekst tekst tekst tekst tekst tekst tekst tekst tekst ekst tekst tekst tekst tekst t tekst tekst tekst tekst tekst tekst tekst tekst tekst tekst tekst tekst tekst tekst tekst ekst tekst tekst tekst tekst t tekst tekst tekst tekst tekst tekst tekst

\begin{center}
\textbf{Metoda spadku wzdłuż gradientu I (Cauchy’ego)}
\end{center}
Minimalizacja funkcji $L(w)$:
\begin{itemize}
\item Zaczynamy od wybranego rozwiązania startowego, np. $w_{0} = 0$.
\item Dla $k = 1, 2, \dots$ aż do zbieżności
	\begin{itemize}
	\item Wyznaczamy gradient w punkcie $w_{k-1}, \nabla_{L}(w_{k-1})$.
	\item Robimy krok wzdłuż negatywnego gradientu: $$w_{k} = w_{k-1} - \alpha_{k}\nabla_{L}(w_{k-1}). $$
	\end{itemize}
\end{itemize}

\begin{center}
\textbf{Metoda spadku wzdłuż gradientu II (Newtona-Raphsona)}
\end{center}
Minimalizacja funkcji $L(w)$:
\begin{itemize}
\item Zaczynamy od wybranego rozwiązania startowego, np. $w_{0} = 0$.
\item Dla $k = 1, 2, \dots$ aż do zbieżności
	\begin{itemize}
	\item Wyznaczamy gradient w punkcie $w_{k-1}, \nabla_{L}(w_{k-1})$ \\ i odwrotność Hesjanu $(D_{L}^{2}(w_{k-1}))^{-1}$.
	\item Robimy krok wzdłuż negatywnego gradientu z zadanym krokiem przez Hesjan: $$w_{k} = w_{k-1} - (D_{L}^{2}(w_{k-1}))^{-1}\nabla_{L}(w_{k-1}). $$
	\end{itemize}
\end{itemize}
\section{Algorytm stochastycznego spadku wzdłuż gradientu I}\label{SGD}
\begin{center}
\textbf{Metoda stochastycznego spadku wzdłuż gradientu I}
\end{center}
Minimalizacja funkcji $L(w)$:
\begin{itemize}
\item Zaczynamy od wybranego rozwiązania startowego, np. $w_{0} = 0$.
\item Dla $k = 1, 2, \dots$ aż do zbieżności
	\begin{itemize}
	\item Wylosuj $i \in \{1,\dots,n\}$
	\item Wyznaczamy gradient funkcji $\ell_{i}$ w punkcie $w_{k-1}, \nabla_{\ell_{i}}(w_{k-1})$.
	\item Robimy krok wzdłuż negatywnego gradientu: $$w_{k} = w_{k-1} - \alpha_{k}\nabla_{\ell_{i}}(w_{k-1}).$$
	\end{itemize}
\end{itemize}


Stochastyczny spadek gradientu to popularny algorytm wykorzystywany do estymacji współczynników w szerokiej gamie modeli uczenia maszynowego takich jak maszyny wektorów podpierających (\textit{ang. Support Vector Machines}), regresja logistyczna czy modele graficzne~\cite{finkel}. W~połączeniu z algorytmem propagacji wstecznej jest standardowym algorytmem w~trenowaniu sztucznych sieci neuronowych. Algorytm stochastycznego spadku gradientu był używany już od 1960 przy estymacji współczynników w modelu regresji liniowej, oryginalnie znany pod nazwą \textit{ADALINE} \cite{ADALINE}. Kolejnym znanym popularnym algorytmem wykorzystującym stochastyczny spadek gradientu jest filtr adaptacyjny najmniejszych średnich kwadratów \cite{widrow2} (\textit{ang.~least mean squares (LMS) adaptive filter}), który również został wynaleziony w 1960 przez Bernard Widrowa, który jest także twórcą \textit{ADALINE}.

Idea algorytmu stochastycznego spadku gradientu jest następująca: zamiast obliczać gradient na całej funkcji $L$, w danym kroku oblicz
gradient tylko na pojedynczym elemencie $\ell_{i}$. Nazwa \textit{stochastyczny} bierze się stąd, iż oryginalnie wybiera
się element $\ell_{i}$ losowo. W praktyce zwykle przechodzi się po całym zbiorze danych w losowej kolejności lub, o ile to możliwe, w~kolejności chronologicznej obserwacji.



\subsection{Właściwości stochastycznego spadku wzdłuż gradientu}

\subsubsection{Zalety}
\begin{itemize}
\item \textcolor{orange}{Szybkość}: obliczenie gradientu wymaga wzięcia tylko jednej
obserwacji.
\item \textcolor{orange}{Skalowalność}: cały zbiór danych nie musi nawet znajdować się
w pamięci operacyjnej.
\item \textcolor{orange}{Prostota}: gradient funkcji  $\ell_{i}$ daje bardzo prosty wzór na
modyfikacje wag.
\end{itemize}

\subsubsection{Wady}
\begin{itemize}
\item \textcolor{orange}{Wolna zbieżność}: czasem gradient stochastyczny zbiega wolno
i wymaga wielu iteracji po zbiorze uczącym.
\item \textcolor{orange}{Problem z ustaleniem długości kroku $k$}: wyznaczenie $k$
przez przeszukiwanie liniowe nie przynosi dobrych rezultatów,
ponieważ optymalizujemy oryginalnej funkcji $L$ tylko jej jeden
składnik $\ell_{i}$.
\end{itemize}

\subsubsection{Stochastyczny gradient w praktyce}

\begin{itemize}
\item Zwykle nie losuje się obserwacji, ale przechodzi się po zbiorze
danych w losowej kolejności.
\item Zbieżność wymaga często przejścia parokrotnie po całym
zbiorze danych (jednokrotne przejście nazywa się epoka).
\item Metody ustalania współczynników długości kroku $k$:
\begin{itemize}
\item Ustalamy \textcolor{orange}{stała wartość} $\alpha_{k} = \alpha$ \\ Zwykle tak się robi w praktyce, działa dobrze ale wymaga ustalenia $\alpha_{k}$ metodą prób i błędów.
\item Bierzemy wartość kroku malejącą jak \textcolor{orange}{$\sim \frac{1}{\sqrt{k}}: \alpha_{k} = \frac{\alpha}{\sqrt{k}}$} \\
 Zapewniona zbieżność, ale czasem może zbiegać zbyt wolno.
\end{itemize}
\end{itemize}
\chapter{Estymacja w modelu Coxa metodą stochastycznego spadku gradientu}
\chapter{Analiza danych genomicznych - model Coxa z estymacją metodą stochastycznego spadku gradientu}
\section{Opis i pobranie danych}
\section{Analiza}



%\chapter{Studium przypadku}
%\section{Porównanie wyników algorytmu SGD z R-N}


\appendix

\chapter{Wykorzystane narzędzia}
\chapter{Kody w R}
\section{Implementacje optymalizacji w regresji logistycznej}\label{kody}

\section{Model proporcjonalnych hazardów Coxa}\label{coxKody123}

\begin{Shaded}
\begin{Highlighting}[]
\NormalTok{checkArguments <-}\StringTok{ }\NormalTok{function(formula, data, learningRates,}
                             \NormalTok{beta_0, epsilon) \{}
  \KeywordTok{assert_that}\NormalTok{(}\KeywordTok{is.list}\NormalTok{(data) &}\StringTok{ }\KeywordTok{length}\NormalTok{(data) >}\StringTok{ }\DecValTok{0}\NormalTok{)}
  \KeywordTok{assert_that}\NormalTok{(}\KeywordTok{length}\NormalTok{(}\KeywordTok{unique}\NormalTok{(}\KeywordTok{unlist}\NormalTok{(}\KeywordTok{lapply}\NormalTok{(data, ncol)))) ==}\StringTok{ }\DecValTok{1}\NormalTok{)}
  \CommentTok{# + check names and types for every variables}
  \KeywordTok{assert_that}\NormalTok{(}\KeywordTok{is.function}\NormalTok{(learningRates))}
  \KeywordTok{assert_that}\NormalTok{(}\KeywordTok{is.numeric}\NormalTok{(epsilon))}
  \KeywordTok{assert_that}\NormalTok{(}\KeywordTok{is.numeric}\NormalTok{(beta_0))}
  
    \CommentTok{# check length of the start parameter}
  \NormalTok{if (}\KeywordTok{length}\NormalTok{(beta_0) ==}\StringTok{ }\DecValTok{1}\NormalTok{) \{}
    \NormalTok{beta_0 <-}\StringTok{ }\KeywordTok{rep}\NormalTok{(beta_0, }\KeywordTok{as.character}\NormalTok{(formula)[}\DecValTok{3}\NormalTok{] %>%}
\StringTok{                    }\KeywordTok{strsplit}\NormalTok{(}\StringTok{"}\CharTok{\textbackslash{}\textbackslash{}}\StringTok{+"}\NormalTok{) %>%}
\StringTok{                    }\NormalTok{unlist %>%}
\StringTok{                    }\NormalTok{length)}
  \NormalTok{\}}
  \KeywordTok{return}\NormalTok{(beta_0)}
\NormalTok{\}}
\end{Highlighting}
\end{Shaded}


\begin{Shaded}
\begin{Highlighting}[]
\NormalTok{prepareBatch <-}\StringTok{ }\NormalTok{function(formula, data) \{}
  \CommentTok{# Parameter identification as in  `survival::coxph()`.}
  \NormalTok{Call <-}\StringTok{ }\KeywordTok{match.call}\NormalTok{()}
  \NormalTok{indx <-}\StringTok{ }\KeywordTok{match}\NormalTok{(}\KeywordTok{c}\NormalTok{(}\StringTok{"formula"}\NormalTok{, }\StringTok{"data"}\NormalTok{),}
                \KeywordTok{names}\NormalTok{(Call), }\DataTypeTok{nomatch =} \DecValTok{0}\NormalTok{)}
  \NormalTok{if (indx[}\DecValTok{1}\NormalTok{] ==}\StringTok{ }\DecValTok{0}\NormalTok{) }
      \KeywordTok{stop}\NormalTok{(}\StringTok{"A formula argument is required"}\NormalTok{)}
  \NormalTok{temp <-}\StringTok{ }\NormalTok{Call[}\KeywordTok{c}\NormalTok{(}\DecValTok{1}\NormalTok{, indx)]}
  \NormalTok{temp[[}\DecValTok{1}\NormalTok{]] <-}\StringTok{ }\KeywordTok{as.name}\NormalTok{(}\StringTok{"model.frame"}\NormalTok{)}
  
  \NormalTok{mf <-}\StringTok{ }\KeywordTok{eval}\NormalTok{(temp, }\KeywordTok{parent.frame}\NormalTok{())}
  \NormalTok{Y <-}\StringTok{ }\KeywordTok{model.extract}\NormalTok{(mf, }\StringTok{"response"}\NormalTok{)}
  
  \NormalTok{if (!}\KeywordTok{inherits}\NormalTok{(Y, }\StringTok{"Surv"}\NormalTok{)) }
      \KeywordTok{stop}\NormalTok{(}\StringTok{"Response must be a survival object"}\NormalTok{)}
  \NormalTok{type <-}\StringTok{ }\KeywordTok{attr}\NormalTok{(Y, }\StringTok{"type"}\NormalTok{)}
  
  \NormalTok{if (type !=}\StringTok{ "right"} \NormalTok{&&}\StringTok{ }\NormalTok{type !=}\StringTok{ "counting"}\NormalTok{) }
      \KeywordTok{stop}\NormalTok{(}\KeywordTok{paste}\NormalTok{(}\StringTok{"Cox model doesn't support }\CharTok{\textbackslash{}"}\StringTok{"}\NormalTok{, type, }\StringTok{"}\CharTok{\textbackslash{}"}\StringTok{ survival data"}\NormalTok{, }
          \DataTypeTok{sep =} \StringTok{""}\NormalTok{))}
  
  \CommentTok{# collect times, status, variables and reorder samples }
  \CommentTok{# to make the algorithm more clear to read and track}
  \KeywordTok{cbind}\NormalTok{(}\DataTypeTok{event =} \KeywordTok{unclass}\NormalTok{(Y)[,}\DecValTok{2}\NormalTok{], }\CommentTok{# 1 indicates event, 0 indicates cens}
        \DataTypeTok{times =} \KeywordTok{unclass}\NormalTok{(Y)[,}\DecValTok{1}\NormalTok{],}
        \NormalTok{mf[, -}\DecValTok{1}\NormalTok{]) %>%}
\StringTok{    }\KeywordTok{arrange}\NormalTok{(times) }
\NormalTok{\}}
\end{Highlighting}
\end{Shaded}

%\chapter{Kody w R}\label{kody}
\chapter{Dokumentacja pakietu RTCGA}
\includepdf[pages=-]{RTCGA.pdf}

\begin{thebibliography}{99}

\bibitem[1]{aldrich1} Aldrich J., (1997) \textit{R. A. Fisher and the Making of Maximum Likelihood 1912 – 1922}, Statistical Science
1997, Vol. 12, No. 3, 162-176.

\bibitem[2]{biecek1} Biecek P., (2011) \textit{Przewodnik po pakiecie R}, Rozprawa doktorska, Oficyna Wydawnicza GiS, wydanie II.

\bibitem[3]{bott1} Bottou L., (2010) \textit{Large-Scale Machine Learning with Stochastic Gradient Descent}.

\bibitem[4]{bott1} Bottou L., (2012) \textit{Stochastic Gradient Descent Tricks}.

\bibitem[5]{fisher1} Fisher R. A., (1912) \textit{An absolute criterion for fitting frequency curves}. 

\bibitem[6]{fisher2} Fisher R. A., (1922) \textit{On the mathematical foundations of theoretical statistics}, Philos. Trans. Roy. Soc. London Ser. A 222 309-368.


\bibitem[7]{gauss1} Gauss C. F., (1809) \textit{Theoria Motus Corporum Coelestium}.

\bibitem[8]{gagol1} Gągolewski M., (2014) \textit{Programowanie w języku R}, Wydawnictwo Naukowe PWN.


\bibitem[9]{hutch1} Hutchinson J. B., (1928) \textit{The Application of the "Method of Maximum Likelihood" to the Estimation of Linkage}, Genetics. 1929 Nov; 14(6): 519–537.


\bibitem[10]{kenward1} Kenward M. G., Lesaffre E. and Molenberghs G., (1994) \textit{An Application of Maximum Likelihood and Generalized Estimating Equations to the Analysis of Ordinal Data from a Longitudinal Study with Cases Missing at Random}, Biometrics
Vol. 50, No. 4 (Dec., 1994), pp. 945-953.

\bibitem[11]{legendre1} Legendre A. M., (1804) \textit{Nouvelles m´ethods pour la d´etermination des orbites des com`etes}.

\bibitem[12]{millar1} Millar R. B., (2011) \textit{Maximum Likelihood Estimation and Inference: With Examples in R, SAS and ADMB, chapter 6. Some Widely Used Applications of Maximum Likelihood}, John Wiley \& Sons, Ltd.

\bibitem[13]{rydl1} Rydlewski J., (2009) \textit{Estymatory Największej Wiarogodności w Uogólnionych Modelach Regresji Nieliniowej}, Rozprawa doktorska.

 



\end{thebibliography}


\makestatement
\end{document}
